\begin{abstract}
	%In this thesis we give a brief introduction to the theory of real \Cstar algebras, including a brief review of the classification of real continuous trace algebras. We then develop on the work in \cite{mesrennie} to develop the unbounded Kasparov product for real \Cstar algebras. We show that the construction works almost exactly as in \cite{mesrennie}. 
	In this thesis we construct the unbounded Kasparov product in the real setting, based on the work in the complex setting of \cite{mesrennie}. We recreate the results therein, and where the real and complex settings diverge, we prove analogous results in the real setting. Thereby we create the framework needed for the general lifting construction of the unbounded Kasparov product in \cite[Section 4]{mesrennie}.
	To develop a feel for the theory of real \Cstar algebras, we give the classification of real graded continuous trace from \cite{moutou}.   We also give an introduction to $KK$-theoretic duality, introducing the notion of non-commutative Poincare duality and propose classes implementing Poincare duality for the non-commutative 2-torus in the real setting.
\end{abstract}
\section*{Introduction}
\addcontentsline{toc}{section}{Introduction}
There has been renewed interest in real \Cstar algebras and their topological invariants owing to their new-found applications in condensed matter physics, \cite{bourne}, where the hope is to use the unbounded Kasparov product to put various topological invariants stemming from physical considerations on a firm mathematical footing. Non-commutative differential geometry might be the natural setting in which to study many condensed matter systems, as the dynamics of condensed matter systems may often be described by unbounded non-commuting derivations. The success of this approach is evident in the description of the Integer Quantum Hall Effect, where the only consistent theoretical model is that of \cite{bellissard}.

The study of real \Cstar algebras has a different flavor than the study of complex \Cstar algebras. The difference may essentially be ascribed to the superfically trivial fact that $-1$ and $1$ are not homotopic through unitaries in the reals, so the group of unitaries of $\R$ is $\Z_2$ There is also the essential difference that there are three different real normed division algebras, in contrast to the single complex one. 
The continuous algebras provide us with a class of algebras which is particulary well-suited to illustrate the pecularities of real \Cstar algebras. Continuous trace algebras also serve to illustrate the advantages of the bundle theoretic viewpoint, where the ideas of classical geometry may be brought to bear on non-commutative problems. The classification of continuous trace algebras via cohomological means was completed in the 1980's, as summarized in \cite{raeburncont}. The classification was expanded in \cite{moutou} to real continuous trace algebras with groupoid actions. In this thesis we reproduce the classification of Real Graded continuous trace algebras from \cite{moutou}, showing that the Brauer group of a space $(T,\tau)$ with involution classifies the continuous trace algebras with primitive ideal space $(T,\tau)$ up to stable isomorphism.

One of the most important unifying themes across the thesis is the use of the bundle-theoretic viewpoint as enabling an analysis of \Cstar algebraic properties, both directly for the continuous trace algebras and in the use of the local-global principle in constructing the unbounded Kasparov product. This viewpoint also has computational perspectives in condensed matter physics, as outlined in \cite{prodan}.


Within the past few years there has been substantial progress in the unbounded approach to $KK$-theory, beginning with the PhD Thesis of Bram Mesland as elucidated in \cite{mesland}.
and \cite{unboundkasp}. This development has hinged on several technical developments drawing on ideas from ordinary differential geometry. 
To work with differential structures the appropriate setting is found to be the categories of operator algebras and modules, serving as non-commutative generalization of differential algebras and bundles. 
In \cite{mesland} a crucial ingredient in the construction of the unbounded Kasparov product is found to be the non-commutative analogue of a connection, serving to define a version of $1\tens \D_2$ which is well-defined on the interior tensor product. 
Then to provide a suitable differential structure to accommodate the construction of connections requires a non-commutative analogue of a frame. This analogue is provided by the introduction of projective operator modules, which come with a canonical frame. 
The final technical tool in the construction is the Local-Global principle, developed separately in \cite{pierre} and \cite{locglob} which is used to show that the operator implementing the unbounded Kasparov product is self-adjoint and regular.
The framework in which these ideas come together is the notion of an unbounded $\A-\B$-correspondence, which turns out to be the essential ingredient in the construction of the unbounded Kasparov product. 
It was shown in \cite{mesrennie} that using the tools we have just mentioned, one can recover the entirety of complex $KK$-theory in the unbounded setting.
In the thesis we take these results and generalize them to incorporate the real setting, showing that we can also construct the unbounded Kasparov product for real algebras. 
In order to do so, we show a number of results concerning unbounded multipliers and approximate units. These are essential ingredients in \cite[Section 4]{mesrennie} for showing that given any two composable Kasparov modules, we may lift them to composable unbounded modules. With our results, all the proofs go through smoothly in the real case. 
This ends the line of inquiry investigating unbounded $KK$-theory as vehicle for capturing bounded $KK$-theory. What we show is that unbounded $KK$-theory completely captures bounded $KK$-theory in both the real and complex cases, but it does not lend itself to immediate extension to all cases of interest due to the reliance on bounded approximate units in our differential algebras.
This opens the door to the program begun in \cite{jensmorita}, \cite{kaad} and \cite{mesland} of building an abelian group from the unbounded $KK$-theory. The work of \cite{kaad} and \cite{jensmorita} seems to be promising in that it contains both an equivalence relation and a direct sum, but the theory is still not fully formed.  
An interesting further development in real $KK$-theory, of particular interest to the ongoing investigation of physical consequences thereof, is showing $KK$-theoretic Poincare duality for the real non-commutative torus. We have worked on this, but not been able to come to any useful results. 

%overvej at tilføje reel til summary

In summary, the thesis illustrates how ideas from commutative geometry may be brought to bear on seemingly intractable problems in non-commutative geometry, as well as hinting at how the operator theoretic viewpoint leads to new insights in commutative differential geometry. The thesis also illustrates the added flexibility stemming from working with unbounded operators.
