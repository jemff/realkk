\subsection{Basic properties}
We can now proceed to consider the $K$-theory as well as its corresponding $K$-homology of real \Cstar-algebras in their joint guise of Kasparovs $KK$-theory. There are several reasons for considering unbounded real $KK$-theory. Unbounded real $KK$-theory can detect orientation, and can detect differences between algebras which are equal in the complex case, eg. $\H$ and $M_2(\R)$ which have isomorphic complexifications but are clearly different as real algebras. We start by defining $KKO$-theory. 
\begin{definition}
	Let $A$ and $B$ be separable real \Cstar algebras. We define the Kasparov $(A,B)$ cycles $E(A,B)$ as the triples $E=(E,\pi,F)$, where:
	\begin{enumerate}
		\item
			$E$ is a countably generated $\Z_2$ graded real Hilbert $B$ module 
		\item
			The map $\pi$ is a graded real $^*$-homomorphism $A\to L(E)$. 
		\item
			The operators $[\pi(A),F]$, $\pi(a)(F^2-1)$ and $(F^2-1)\pi(a)$ are in $K(E)$ for all $a$ in $A$.
	\end{enumerate}
	A triple $(E,\pi,F)$ is degenerate if all three operators in the third requirement are identically zero for every $a$ in $A$. We can endow $E(A,B)$ with a binary operation through the direct sum:
	\begin{align*}
		(E_1,\pi_1,F_1)\osum (E_2,\pi_2,F_2)=(E_1\osum E_2,\pi_1\osum \pi_2,F_1\osum F_2)
	\end{align*}
\end{definition}
In order to reduce the unwieldy collection of cycles to a group, we define an equivalence relation akin to the one used to define $K$-homology. 
\begin{definition}
	Two cycles $E_1=(E,\pi,F')$ and $(E,\pi,F)$ are operator homotopic if there is a strictly continuous family $F_t$ giving rise to a path of cycles $(E,\pi,F_t)$ where $F_0=F,F_1=F'$. We define the equivalence relation $\sim_{oh}$ as the equivalence relation stemming from addition of degenerate modules and operator homotopy. 
\end{definition}
\begin{definition}
	Define $KKO(A,B)$ as the group $E(A,B)/\sim_{oh}$. 
	Define the higher $KKO$ groups by the formula 
	\begin{align*}
		K_{p,q}K^{r,s}O(A,B)=KKO(A\tensh Cl_{p,q},B\tensh Cl_{r,s})
	\end{align*}
\end{definition}
For details and proofs, see \cite{kasparov}, where the the equivalence between formal and topological Bott periodicity is explored.
We summarize some of the properties of operator $KKO$ theory in the theorem below, for a proof see \cite{kasparov} or \cite{schroder}.
\begin{theorem}
\begin{enumerate}
\item
	Operator $KO_{*}$-theory is a stable covariant $8$-periodic functor from the category of real graded \Cstar algebras to the category of abelian groups. The functor $KO_*$ takes short exact sequences of real graded \Cstar-algebras to a 24-term cyclic exact sequence of abelian groups. 
\item	Operator $KO^{*}$-homology is a stable $8$-periodic contravariant functor from the category of real \Cstar algebras to the category of abelian groups. If we have a short exact sequence with a completely positive splitting, $KO^{*}$ applied to the sequence gives rise to a 24-term cyclic exact sequence. 
	\item The functors $KO_*$ and $KO^*$ are combined in the stable bifunctor $KKO$, where $KKO(A,\R)\cong KO^{0}(A)$ and $KKO(\R,A)\cong KO_0(A)$. We summarize some of the most important properties of $K^*K_*O$ relating to formal Bott periodicity
	\begin{align*}
		&KO_n(A)\cong KO_0(A\tens Cl_{0,n}) \\
		&KO(A)\cong KKO(\R,A) \\
		&KKO(\R,Cl_{p,q} \tens A) \cong KKO(Cl_{q,p},A) 
	\end{align*}
	\end{enumerate}
	The functor $K^*K_*O$ also satisfies topological Bott-periodicity, ie. it is 8-periodic under taking suspensions. 
\end{theorem} 
\begin{lemma} \label{KTMapping}
	Let $A$ be a real \Cstar algebra then the isomorphism between $K_0(A)$ and $KO(\R,A)$ is given by:
	$[p] \mapsto [(\pi_\R,pA^n,0)]$ for $[p]\in KO_0(A)$. 
\end{lemma}
\begin{proof}
	We start by showing the mapping is well-defined. Let $p\sim_h q$ be unitarily equivalent projections in $\K \tens A$, with the equivalence implemented by some unitary $U$. Then $U(pA^n)U^*=qA^m$, so the mapping is well-defined. To see the mapping is injective, assume that $[(\pi,pA^n,0)]$ is a trivial cycle, implying that $0-Id_{pA^n}=0$, so $p=0$. Given an arbitrary cycle $[(\pi,E,F)]$ with $F$ self-adjoint, consider $[(\pi,\ker(F),0)]$ and $[(\pi,\ker(F)^\perp,F)]$. These are both well-defined as the range of $F$ is closed, it being Fredholm, and by \cite[Theorem 3.2]{lance}, $\ker(F)$ is a complemented submodule. To show surjectivity, it suffices to show that the second cycle is degenerate. This follows from a swindle argument as $F$ is a linear isomorphism in the second cycle, and all commutants with $\pi$ are zero.  
	Additivity is clear, showing the desired. 
\end{proof}
We use the periodicity of $KKO$-theory to derive the following standard results on operator $KO$-theory of the reals, \cite[Section 1]{schroder}. 
\begin{example}
	The $K$-theory of the reals is: 
	\begin{align*}
		&KO_*(\R)=\left \{\begin{array}{c c} 0 & \Z  \\ 1 & \Z_2 \\ 2 & \Z_2 \\  4 & \Z\end{array} \right .
	\end{align*}
	By formal Bott periodicity in $KKO$-theory,  \begin{align*}KO^n(\R)\cong KKO(Cl_{0,n}\tens \R,\R)\cong KO_{-n}(\R), \end{align*} allowing us to derive the following table:
	\begin{align*}
		&KO_*(\R)=\left \{ \begin{array}{c c} KO_0 & \Z  \\ KO_{-1} & \Z_2 \\ KO_{-2} & \Z_2 \\ KO_{-4} & \Z \end{array} \right .
	\end{align*}
\end{example}
As described in the section on real \Cstar-algebras, we have both suspensions and anti-suspensions in real bivariant $K$-theory, which are related to the Clifford algebras via. the following lemma. 
\begin{lemma}
	We have the isomorphisms.
	\begin{align*}
		KKO(A\tens Cl_{0,n},B)\cong KK(A\tens C_0(\R^n,\id),B) \\
		KKO(A\tens Cl_{n,0},B)\cong KK(A\tens C_0(\R^n,-\id),B)
	\end{align*}
	Thus we have the isomorphisms $KO^n(A)=KO(C_0(\R^n)\tens A)$, and $KO^{-n}(A)\cong KO(A\tens C_0(\R,-\id))$. 
\end{lemma}
For a proof, see \cite{kasparov}
We may immediately use these results to calculate the $K$-theory of the algebra $C(S^1,\id)$, giving a flavor of the theory, as well as characterizing how tensoring with $C(S^1,\tau)$ affects the $K$-theory of an algebra. 

\begin{theorem}\label{ktheorys1}
	The $KO$-theory of $C(S^1,\id)$ is given by the following table. 
	\begin{align*}
		KO_n(C(S^1,\id))=\left \{ \begin{array}{c c} \Z\osum \Z_2 & n=0 \\ \Z_2\osum \Z_2 & n=1 \\ \Z_2 & n=2 \\ \Z & n=3,4,7 \\ 0 & n=5,6 \end{array} \right .
	\end{align*}
	For any real \Cstar-algebra $A$ we have the isomorphism 
	\begin{align*}
		KO_n(A\tens C(S^1,\tau_0))\cong KO_n(A)\osum KO_{n-1}(A)
	\end{align*}
	This allows us to compute the $K$-theory of $C(S^1,\tau_0)$.
\end{theorem}
\begin{proof}
	Consider the split short exact sequence, with the split begin given by $x \mapsto (z\mapsto x)$ for $x\in \R$. 
	\begin{align*}
		\xymatrix{
			0\ar[r] & C_0(\R,\id) \ar[r] & C(S^1,\id) \ar[r] & \R \ar[r] & 0 
		}
	\end{align*}
	Thus the 24-periodic exact sequence in $K$-theory reduces to the split short exact sequence 
	\begin{align*}
	\xymatrix{
		0 \ar[r] & KO_{n+1}(\R) \ar[r] & KO(C(S^1,\id)) \ar[r] & KO_n(\R) \ar[r] & 0
		}
	\end{align*}
	giving the desired table. To show the second part of the lemma, consider the split short exact sequence
	\begin{align*}
	\xymatrix{
		0\ar[r] & A\tens C_0(i\R) \ar[r] & A\tens C(S^1,\tau) \ar[r] & A \ar[r] & 0
		}
	\end{align*}
	with the split being given by the map $a\mapsto a\tens 1$.
	We can infer that the 24-periodic long-exact sequence reduces to the split short exact sequence
	\begin{align*}
		\xymatrix{
			0\ar[r] & KO_{n-1}(A) \ar[r] & KO_n(A\tens C(S^1,\tau)) \ar[r] & KO_n(A) \ar[r] & 0
		}
	\end{align*}
	giving $KO_n(A\tens C(S^1,\tau))\cong  KO_{n-1}(A)\osum  KO_n(A)$ as desired. 
\end{proof}
The benefit of real $K$-theory over complex $K$-theory is exhibited in these two cases, where we can see a much finer structure on the $K$-theory of the real circle.

In recent years real $K$-theory has been applied in the realm of topological insulators in order to, once again, explain global behaviors by the presence of these torsion classes. This has also explictly used the construction of the unbounded Kasparov product, \cite{bourne}, in a form which we shall recreate at a later point in the thesis. 
\subsection{$K$-homology of $\R$}
The standard picture of $KK$-theory works very well in purely algebraic context, but in practice the operators that encode the geometry, commutative or otherwise, of a given space are naturally defined as unbounded regular operators. To remedy this shortcoming, it is often preferable to work with unbounded $KK$-theory which is defined from suitable unbounded Kasparov cycles. 
We start by working in the bounded case, and then argue that for suitable operators unbounded operators, the bounded results essentially also hold.  
\begin{definition}
	Let $H$ be a real Hilbert space with the structure of a $Cl_{0,k}$-module, and let $F(H,H)$ denote the odd Fredholm operators on $H$. Define $F_k$ as those elements of $F(H,H)$ which are $Cl_{0,k}$-linear and self-adjoint. 
\end{definition}
We define the Clifford index of operators lying in $F_k$, referring to \Cref{abstheorem} for the notation on $\hat{A}_k$ and $\hat{\M}_k$. 
\begin{definition}
	Let $T$ be an operator in $F_k(H,H)$. Define the Clifford index of $T$ as:
	\begin{align*}
		\ind_k(T)=[\ker T]\in \hat{A}_k\cong KO_{k}(\R)
	\end{align*}
	This is well-defined due to the Atiyah-Bott-Shapiro isomorphism, \Cref{abstheorem}.
\end{definition}
In order to justify calling this construction an index, we check that $\ind_0$ recovers the usual Fredholm index. 
\begin{example}\label{cliffordkernel}
	We see $Cl_0=\R$ and $Cl_{0,1}=\C$. A $\Z_2$ graded $Cl_0$-module is simply a graded $\R$-vector space, $V_0\osum V_1$. We observe that $[V\osum 0]=-[0\osum V]$ in $\hat{A}_0$ since $V\osum V\cong V\tens \C$ may be extended to a graded $Cl_{0,1}$-module. Given $T\in F_k(H,H)$ we get $\ind_0(T)=[\ker T_0\osum \ker T_1]=[\ker T_0 \osum 0]-[\ker T_1 \osum 0]$, recovering the Fredholm Index of $T_0$. 
\end{example}
We wish to show that the index map is well-defined as a map from $K$-homology, so we need it to be homotopy invariant and a group homomorphism. 
\begin{theorem}\label{welldef}
	The Clifford index is constant on connected components of $F_k$, i.e. it is homotopy-invariant. 
\end{theorem}
\begin{proof}
	Let $T\in F_k$, as $0$ is an isolated point in the spectrum we may assume that the non-zero spectrum of $T$ lies outside of $[-2,2]$. Pick a neighborhood $U$ of $T$ in $F_k$ such that for all $S\in U$ we have that $\sigma(S^2)\subset [0,1/2)\cup (1,\infty)$ and $\norm{T^2-S^2}\leq 1/2$. Fix $S\in U$ and let $W$ be the range of the spectral projection of $S^2$ onto $[0,1/2]$. 
	Consider also the orthogonal projection $P:H\to \ker T$. Then we claim that $p:W\cong \ker T$. To do this, pick $v\in W\in (\ker T)^\perp$. We have that the equality 
	\begin{align*}
		\ip{(T^2-S^2)v}{v}\geq \pa{2-\frac{1}{2}}\norm{v}^2\geq \norm{v}^2
	\end{align*}
	As $\norm{T^2-S^2}<\frac{1}{2}$, we get that $\norm{v}<1/2$, and as such we get that that $v=0$. In order to see surjectivity, pick $v\in W^\perp \cap \ker T$ and observe that $\ip{(S^2-T^2)v}{v}\geq \norm{v}^2$ and as such is 0. \todo{Add argument for these inequalities (locally self-adjoint).}
	We see that $W$ is a $\Z_2\Z$ graded submodule of $H$, as well as splitting as $W=\ker S\osum (\ker (S)^\perp \cap W)$. The projection $P$ onto the kernel of $T$ also preserves the graded module structure. Defining $V=(\ker (S)^\perp \cap W)$ we get the equivalence
	\begin{align*}
		\ker T\cong \ker S\osum V
	\end{align*}
	In order to see that the class is the same, we need to give $V$ the structure of $Cl_{k+1}$ module in order to map it to $A_k$.
	Let $S_V=S|_V$. This is a symmetric $\Z_2$ $Cl_{0,k}$-linear graded map. Thus the operator $J=(S^2_V)^{-1/2}S_V$ is as well. We see that $J^2=Id$. Decomposing $V=V_0\osum V_1$, $J=e_{21}J_0+e_{12}J_1$. The graded endomorphism $\tilde{J}=-e_{21}J_0+e_{12}J_1$ squares to $-1$, and as such makes $V$ into a $Cl_{k+1}$-module as desired. Thus $\ind_k T=\ind_k S$ as desired. 
\end{proof}
We refer to the following reuslt: \cite[Page 217]{spingeom}.
\begin{proposition}\label{unboundkernel}
	Given self-adjoint elliptic $Cl_{0,k}$-linear (pseudo)-differential operator $\D$, the class $[\ker(\D)]$ in $\hat{A}_k$ equals $[\ker(\D/(1+\D^2)^{-1/2})]$.
\end{proposition}
To construct cycles representing the generators of the real $K$-homology groups presenting the flavor of the theory, it suffices to construct unbounded elliptic operators with kernels corresponding to the generators of the groups $\hat{A}_k$. Alternatively, we could use the construction in \Cref{KTMapping}.
\begin{theorem}
	The following cycles are generators of the non-trivial real $K$-homology groups of the reals. 
	\begin{enumerate}
	\item
	The group $KO^0(\R)$ is generated by the cycle 
	\begin{align*}
		E=\pa{\R, \ell^2(\N,\R)\osum \ell^2(\N,\R), \begin{pmatrix} 0 & D \\ D^* & 0\end{pmatrix}}
	\end{align*}
	Where the operator $D$ is defined on the core spanned by the standard basis as:
	\begin{align*}
		&D:\ell^2(\N,\R)\to \ell^2(\N,\R) \\
		&e_k\mapsto ke_{k+1}
	\end{align*}
	The adjoint of $D$ also has the standard basis as core, where it is defined as 
	\begin{align*}
		D^*(e_k)=\left \{\begin{array}{c c} 0 & k=1 \\ \frac{1}{k-1}e_{k-1} &  k\geq 2 \end{array} \right .
	\end{align*}

	\item
		The group $KO^{-1}(\R)$ is generated by 
		\begin{align*}
			\pa{Cl_{0,1}, L^2(S^1,\C), \gamma_1 d_\theta}
		\end{align*}
		where $\theta$ is the angular direction on the circle, and $\gamma_1$ is the generator of $Cl_{0,1}\cong \C$. 
	\item
		The group $KO^{-2}(\R)$ is generated by 
		\begin{align*}
			(Cl_{0,2},L^2(S^1\times S^1,\H),\gamma_1 \part_{\theta_1}+\gamma_2 \part_{\theta_2})
		\end{align*}
		where $\theta_i$ are the angular directions on the torus and $\gamma_1,\gamma_2$ are the generators of $Cl_{0,2}\cong \H$.
	\item
		The group $KO^{-4}(\R)$ is generated by:
			\begin{align*}
		(Cl_{0,4},(\ell^2(\N)\osum \ell^2(\N))\tens_\R (\H\osum \H),\D)
	\end{align*}
		Where the representation of $Cl_{0,4}$ is either of the two irreducible graded representations. 


	\end{enumerate}
\end{theorem}
\begin{proof}
We start by noting that for every $n$ the cycles are given by self-adjoint elliptic $Cl_{0,n}$-linear operators. 
To start, we wish to show that: 
\begin{align*}
	\ind_n(T):KO^n(\R)\to \hat{A}_k
\end{align*}
is a homomorphism. 
We start by checking that it is well-defined on equivalence classes. Invariance under unitary isomorphism is clear, and by \Cref{welldef} the map $\ind_n(T)$ is invariant under operator homotopy. Thus it is well-defined as a function $KO^*(\R)\to \hat{A}_*$
To see that $\ind_n$ is a homomorphism, assume that $(\pi,F,H)$ is a degenerate cycle, so it is $Cl_{0,k}$-linear. Furthermore, degeneracy implies that $F^2=1$ and $F=F^*$. Therefore $F$ is a self-adjoint unitary and and thus has trivial kernel. Hence $\ind_n(F)=0$. 
Additivity is clear, hence can infer that $\ind_n: KO^n\to \hat{A}_n$ is homomorphism. As we know what the left-hand side is by Bott periodicity, we only need to show that our cycles in $KO^n(\R)$ have the appropriate index. %Therefore we shall be working directly with unbounded operators, \Cref{unboundkernel}. 
\begin{enumerate}
\item
	Define the operator 
	\begin{align*}
		&D:\ell^2(\N,\R)\to \ell^2(\N,\R) \\
		&e_k\mapsto ke_{k+1}
	\end{align*}
	The operator clearly has the standard basis as core of its minimal closure, and is thereby densely defined. 
	The adjoint of $D$ likewise has the standard as core, where it is given by:
	\begin{align*}
		D^*(e_k)=\left \{\begin{array}{c c} 0 & k=1 \\ \frac{1}{k-1}e_{k-1} &  k\geq 2 \end{array} \right .
	\end{align*}
	Consider the cycle 
	\begin{align*}
		E=\pa{\R, \ell^2(\N,\R)\osum \ell^2(\N,\R), \begin{pmatrix} 0 & D \\ D^* & 0\end{pmatrix}}
	\end{align*}
	By \Cref{cliffordkernel} $\ind_0\pa{\begin{pmatrix} 0 & D \\ D^* & 0\end{pmatrix} }$ is the Fredholm Index of $D$, which is readily seen to be 1. Thus $E$ generates $KO^{0}(\R)$. 
\item
	
	Consider the operator $\D=\gamma_1\tens d_\theta$, in the Fourier basis we can write this on the basis vectors as:
	\begin{align*}
		&\D: \ell^2(\Z)\to \ell^2(\Z) \\
		&e_k\mapsto ke_k
	\end{align*}
	Remark that the considerations above on domains can be reused for this operator. 
	We wish to determine the Clifford index of this operator and thus we take $[\ker(\D)]\in \hat{A}_1\cong \Z_2$. The kernel consists of the constant $Cl_{0,1}$-functions, equipped with the canonical grading. Thus the kernel is $Cl_{0,1}$ viewed as a module over itself. This is is the generator of $\hat{\M}_1$, and thus its image in the quotient $\hat{\M}_1\to \hat{A}_1$ is the generator of $\hat{A}_1$.  
	
%	which by \Cref{spindex} is $\dim_{Cl_{0,2}}(ker(\gamma\tens d_x)) \mod 2$, consisting of the constant real-valued functions and thus is non-zero.
%	\todo{Change well-defined proof to more general, works for all taken from Prop 10.6-Theorem 10.8 in Spin Geometry}

	%which is the quarternionic dimension of the kernel modulo 2, by \Cref{spindex}. As before, the kernel consists of the constant valued functions and as such the class is non-zero. 

\item
	Consider the operator $\D=\gamma_1 \part_{\theta_1}+\gamma_2 \part_{\theta_2}$ on $L^2(S^1 \times S^1,\H)$. If we consider $\D$ as an operator in the Fourier basis, we can write it as	
	\begin{align*}
		\D(k_j e_j,c_n e_n)&=(-\gamma_1\gamma_1 jk_j e_j,-\gamma_2\gamma_1n c_n  e_n) \\
		&=( jk_j e_j,\gamma_1\gamma_2 n  c_ne_n)
	\end{align*}
	where $k_j,c_n\in \H$. Thus $\D f=0$ implies $k_j=c_n=0$ for all non-zero $j,n$. Therefore $f=(k_0,c_0)$ and is thereby a constant $\H$-valued function. We see that the kernel is isomorphic to the constant $\H$-valued functions and as such is the generator of $\hat{\M}_2$, and thereby the generator of $\hat{A}_2$. 
\item
	Let $\pi:M_2(\H)\to \H\tens_{\H\osum \H} M_2(\H)\cong \H\osum \H$ denote either of the two graded irreducible representations of $M_2(\H)$, see \Cref{equivalentcliff}. Define the map $\D:\ell^2(\Z) \tens_\R (\H\osum \H) \to \ell^2(\Z)\tens_\R (\H\osum \H)$ as:
	\begin{align*}
		e_k \tens (h_1,h_2) \mapsto ke_k \tens (h_1,h_2)
	\end{align*}
	The kernel of this map is isomorphic to $\H\osum \H$, which is the generator of $\hat{A}_4$ by construction. 
	Thus the cycle 
	\begin{align*}
		(Cl_{0,4},(\ell^2(\N)\osum \ell^2(\N))\tens_\R (\H\osum \H),\D)
	\end{align*}
	generates $KO^{-4}(\R)$. 
\end{enumerate}
\end{proof}
The above calculations can be seen to be a specific case of the $KO^{*}$-generalization of the analytic index of Clifford-linear operators from spin geometry. 
\begin{remark}
	Let $x\in KO^n(A)$ and $y\in KO_m(A)$. Then we may take the Kasparov product $x\tens y$, to get an element in $KKO(Cl_{0,n},Cl_{0,m})\cong KKO(Cl_{n,m},\R)$. This element is uniquely determined up to operator homotopy by its Clifford index.
\end{remark}
We will give a concrete formula for the calculation of this product in a later section, by applying the unbounded Kasparov product. 