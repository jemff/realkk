\begin{theorem}
	The following cycles are generators of the non-trivial real $K$-homology groups of the reals. 
	\begin{enumerate}
	\item
	Define the operator 
	\begin{align*}
		&D:\ell^2(\N,\R)\to \ell^2(\N,\R) \\
		&e_k\mapsto ke_{k+1}
	\end{align*}
	with formal adjoint 
	\begin{align*}
		D^*(e_k)=\left \{\begin{array}{c c} 0 & k=1 \\ \frac{1}{k-1}e_{k-1} &  k\geq 2 \end{array} \right .
	\end{align*}
	The group $KO^0(\R)$ is generated by the cycle 
	\begin{align*}
		E=\pa{\R, \ell^2(\N,\R)\osum \ell^2(\N,\R), \begin{pmatrix} 0 & D \\ D^* & 0\end{pmatrix}}
	\end{align*}
	\item
		The group $KO^{-1}(\R)$ is generated by 
		\begin{align*}
			\pa{Cl_{0,1}, L^2(S^1,\C), \gamma_1 }
		\end{align*}
	\item
		The group $KO^{-2}(\R)$ is generated by 
		\begin{align*}
			(Cl_{0,2},L^2(S^1\times S^1,\H),\gamma_1 \part_{\theta_1}+\gamma_2 \part_{\theta_2})
		\end{align*}
		where $\theta_i$ are the angular directions on the torus and $\gamma_1,\gamma_2$ are the generators of $Cl_{0,2}\cong \H$.
	\item
		The group $KO^4(\R)$ is generated by 
		\begin{align*}
			(Cl_4,L^2(F_4))\tensh S,\DO,\psi,\pi_{\spin_r})
		\end{align*}
		where $F_4$ is the Fermat quartic ie. hypersurface defined as 
		\begin{align*}
			\{(z_0,\dots,z_3)\in \C P^3 | \sum_{i=0}^3 z_i^4=0\}
		\end{align*}
		and $S$ is the canonical spin bundle, with Dirac operator $\DO$. 
	\end{enumerate}
\end{theorem}
\begin{proof}
We start by noting that for every $n$ the cycles are given by $Cl_{0,n}$-linear operators. Given an operator $T$ we may take $\ind_n(T)$ and it is this we shall use to establish the isomorphism. Invariance under unitary isomorphism is clear, and by \Cref{welldef} the map $\ind_n(T)$ is well-defined as a map $KO^*(\R)\to \hat{A}_k)$ as this invariance under operator homotopy.
To see that it is a homomorphism, assume that $(\pi,F,H)$ is a degenerate cycle, so it is $Cl_{0,k}$-linear. Furthermore, this implies that $F^2=1$ and $F=F^*$. Therefore $F$ is a self-adjoint unitary and thereby has trivial kernel, giving that the image under $\ind_n$ is trivial. 
As such, we have argued that $\ind_n: KO^n\to \hat{A}_k$ is homomorphism, and as we know what the left-hand side is by Bott periodicity, we simply need to find an element of $KO^n(\R)$ that has the appropriate index. Therefore we shall be working directly with unbounded operators, \Cref{unboundkernel}. 
\begin{enumerate}
\item
	Consider the cycle 
	%\begin{align*}
	%	E=\pa{C_0((-1,1))\tensh Cl_{0,1},L^2([0,1])\tensh \C, \gamma_1 dx,(f,\gamma)g\mapsto (f(x)g\gamma)}
	%\end{align*}
	%which we consider as a $Cl_0$-module. If we take $\ind_0(\gamma_1 dx)$ of this operator we recover the Fredholm index of the bounded transform of the operator, which written in Fourier basis is 
	%\begin{align*}
	%e_k\mapsto \frac{k}{\sqrt{k^2+1}}e_k
	%\end{align*}
	%and the Fredholm index of this operator is 1. 
	Define the operator 
	\begin{align*}
		&D:\ell^2(\N,\R)\to \ell^2(\N,\R) \\
		&e_k\mapsto ke_{k+1}
	\end{align*}
	with formal adjoint 
	\begin{align*}
		D^*(e_k)=\left \{\begin{array}{c c} 0 & k=1 \\ \frac{1}{k-1}e_{k-1} &  k\geq 2 \end{array} \right .
	\end{align*}
	Consider the cycle 
	\begin{align*}
		E=\pa{\R, \ell^2(\N,\R)\osum \ell^2(\N,\R), \begin{pmatrix} 0 & D \\ D^* & 0\end{pmatrix}}
	\end{align*}
	By \Cref{cliffordkernel} $\ind_0\pa{\begin{pmatrix} 0 & D \\ D^* & 0\end{pmatrix} }$ is the Fredholm Index of $D$, which is readily seen to be 1. Thus $E$ generates $KO^{0}(\R)$. 
\item
	Consider the operator $\D=\gamma_1\tens d_\theta=e_{12}d_\theta-e_{21}d_\theta$ on $L^2(S^1,\R)\osum L^2(S^1,\R)$. We wish to determine the analytic index of this operator, and thus we take $[\ker(D_\theta)]\in \hat{A}_1\cong \Z_2$. The kernel consists of the constant complex-valued  $Cl_{0,1}$-valued functions viewed as a module over itself, ie. $\ker(D_\theta)$ is the generator of $\hat{\M}_1$, and thus its image is non-trivial under the quotient map $\hat{\M}_1\to \hat{A}_1$, as in \cite[Section 5]{abs}. 
	
%	which by \Cref{spindex} is $\dim_{Cl_{0,2}}(ker(\gamma\tens d_x)) \mod 2$, consisting of the constant real-valued functions and thus is non-zero.
%	\todo{Change well-defined proof to more general, works for all taken from Prop 10.6-Theorem 10.8 in Spin Geometry}

	%which is the quarternionic dimension of the kernel modulo 2, by \Cref{spindex}. As before, the kernel consists of the constant valued functions and as such the class is non-zero. 

\item
	Consider the operator $\D=\gamma_1 \part_{\theta_1}+\gamma_2 \part_{\theta_2}$ on $L^2(S^1 \times S^1,\H)$. If we consider $\D$ as an operator in the Fourier basis, we can write 	
	\begin{align*}
		(\D)(k_j e_j,c_n e_k)=(\gamma_1 jk_j e_j,\gamma_2 c_n n e_n)
	\end{align*}
	where $k_j,c_n\in \H$. Thus $\D f=0$ implies $k_j=c_n=0$ for all non-zero $j,n$. Therefore $f=(k_0,c_0)$ and is thereby a constant $\H$-valued function. We see that the kernel is isomorphic to the constant $\H$-valued functions and as such is the generator of $\hat{\M}_2$, and thereby the generator of $\hat{A}_k$. 
\item
	Let $\pi:M_2(\H)\to (\H\osum \H)$ denote either of the two graded irreducible representations of $M_2(\H)$. Define the map $(\ell^2(\N)\osum \ell^2(\N))\tens_\R (\H\osum \H) \to \ell^2(\Z)\tens_\R (\H\osum \H)$ given as 
	\begin{align*}
		(e_j,e_k)\tens (h_1,h_2) \mapsto k e_k, \tens (h_1,h_2)
	\end{align*}
	The kernel of this map is isomorphic to $\H\osum \H$, which is the generator of $\hat{A}_4$ by construction. 
	Thus we can define the cycle 
	\begin{align*}
		(Cl_{0,4},\ell^2(\Z)
	\end{align*}
	Alternatively we could invoke the results of spin geometry in order to show the claim, which we have hitherto abstained from. We see by \Cref{spindex} that $\ind_4(\DO)=1/2\hat{A}(F_4)$, which by \cite[Example 2.14]{spingeom} is 1 in this case.  We do not delve further into these results and instead refer to \cite{spingeom} for further details. 
	
\end{enumerate}
\end{proof}