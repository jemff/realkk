\documentclass[10pt,a4paper,twoside]{article}
\usepackage{ulem,amsmath,amssymb,ulem,fancyhdr,setspace,lastpage,comment,graphics,wallpaper}
\usepackage{mathrsfs}
\usepackage[obeyFinal,final]{todonotes}
\usepackage[utf8x]{inputenc}
\usepackage[all]{xy}
\usepackage[outer=3cm,inner=4cm]{geometry}
\setlength{\parindent}{0pt}
\usepackage{amsthm}
\theoremstyle{definition}
\usepackage{cleveref}


\pagestyle{fancy}
%\onehalfspacing
\linespread{1.12}
\cfoot{\vspace{3em} {\thepage} of \pageref{LastPage}} 
\lhead{\today}
\newcommand{\authorz}[1]{\author{#1}\lhead{#1}}
\renewcommand{\footrulewidth}{0.4pt}
%\renewcommand{\labelenumi}{(\roman{enumi})}
%\renewcommand{\labelenumii}{(\arabic{enumii})}

\DeclareMathOperator{\id}{id}
\DeclareMathOperator{\Nil}{Nil}
\DeclareMathOperator{\ran}{Ran}
\DeclareMathOperator{\conv}{conv}
\DeclareMathOperator{\cont}{cont}
\DeclareMathOperator{\sign}{sign}
\DeclareMathOperator{\diag}{diag}
\DeclareMathOperator{\colim}{colim}
\DeclareMathOperator{\sing}{sing}
\DeclareMathOperator{\coker}{Coker}
\DeclareMathOperator{\im}{Im}
\DeclareMathOperator{\loc}{loc}
\DeclareMathOperator{\comp}{comp}
\DeclareMathOperator{\findex}{index}
\DeclareMathOperator{\dist}{dist}
\DeclareMathOperator{\op}{op}
\DeclareMathOperator{\spin}{Spin}
\DeclareMathOperator{\pin}{Pin}
\DeclareMathOperator{\SO}{SO}
\DeclareMathOperator{\dom}{Dom}
\DeclareMathOperator{\lip}{Lip}
\DeclareMathOperator{\supp}{supp}
\DeclareMathOperator{\GL}{GL}
\DeclareMathOperator{\Cg}{C^*_{\lambda}(G)}
\DeclareMathOperator{\elset}{\text{ else }}
\DeclareMathOperator{\ift}{\text{ if }}
\DeclareMathOperator{\red}{\text{red}}
\DeclareMathOperator{\Lie}{Lie}
\DeclareMathOperator{\End}{End}
\DeclareMathOperator{\spn}{span}
\DeclareMathOperator{\ind}{Ind}



%i\DeclareMathOperator{\exp}{exp}
\newcommand{\sta}[1]{\text{#1}^{\text{\sout{o}}}}
\newcommand{\de}[2]{\frac{\text{d} #1}{\text{d} #2}}
\newcommand{\titlez}[1]{\title{#1}\rhead{#1}}
\newcommand{\Lr}{\Leftrightarrow}
\newcommand{\pa}[1]{\left ( #1 \right )}
\newcommand{\nintd}[4]{\int_{#3}^{#4} \!#1 \,\text{d}#2}
\newcommand{\nint}[2]{\int \!#1 \,\text{d}#2}
\newcommand{\refz}[1]{(\ref{#1})}
\newcommand{\dpart}[2]{\frac{\partial #1}{\partial #2}}
\newcommand{\dt}[1]{d{#1}}
\newcommand{\df}[1]{\ensuremath{\mathop{\makebox[0pt]{\hspace{10.5pt}{\(^{\text{-}}\)}}d}} #1}
\newcommand{\abs}[1]{\left | #1 \right |}
\newcommand{\for}[0]{\text{ for }}
\newcommand{\nin}[0]{\notin}
\newcommand{\ol}[0]{\overline}
\newcommand{\comm}[1]{\left [ #1 \right ]}
\newcommand{\F}{\mathcal{F}}
\newcommand{\AS}{\mathfrak{S}}
\newcommand{\osum}{\oplus}
\newcommand{\G}{\mathcal{G}}
\newcommand{\g}{\mathfrak{g}}
\newcommand{\K}{\mathbb{K}}
\newcommand{\V}{\mathcal{V}}
\newcommand{\Adel}{\mathbb{A}}
\newcommand{\De}{\mathbb{D}}
\newcommand{\Cstar}{$C^*~$}
\newcommand{\DO}{\mathfrak{D}}
\newcommand{\E}[0]{\mathcal{E}}
\newcommand{\lin}[0]{\mathcal{L}}
\newcommand{\pd}[0]{$\Psi$DO }
\newcommand{\dlim}{\lim\limits_{\longrightarrow}}
\newcommand{\overbar}[1]{\mkern 1.5mu\overline{\mkern-1.5mu#1\mkern-1.5mu}\mkern 1.5mu}
\newcommand{\exre}[0]{\overbar{\boldsymbolb{R}}}
\newcommand{\lsp}[1]{\mathcal{L}^{ #1 }}
\newcommand{\ip}[2]{\left \langle #1,#2 \right \rangle}
\newcommand{\brak}[1]{\langle #1 \rangle}
\newcommand{\N}[0]{\mathbb{N}}
\newcommand{\C}[0]{\mathbb{C}}
\newcommand{\Z}[0]{\mathbb{Z}}
\newcommand{\T}[0]{\mathcal{T}}
\newcommand{\R}[0]{\mathbb{R}}
\newcommand{\Q}[0]{\mathbb{Q}}
\newcommand{\B}[0]{\mathcal{B}}
\newcommand{\Cl}[0]{\C l}
\newcommand{\M}[0]{\mathfrak{M}}
\newcommand{\tensh}[0]{\hat{\otimes}}
\newcommand{\optens}{\widetilde{\otimes}}
\newcommand{\A}[0]{\mathcal{A}}
\newcommand{\Cu}{\mathcal{O}}
\newcommand{\bwed}[1]{\bigwedge\nolimits^{\!#1}}
\newcommand{\Zred}{\hat{\mathbb{Z}}}
\newcommand{\norm}[1]{\left \Vert #1 \right \Vert }
\newcommand{\D}[0]{\mathscr{D}}
\newcommand{\tens}[0]{\otimes}
\newcommand{\bra}[1]{\langle #1 \rvert}
\newcommand{\ket}[1]{\lvert #1 \rangle}


\renewcommand{\H}{\mathbb{H}}
\renewcommand{\S}[0]{\mathcal{S}}
\renewcommand{\part}[0]{\partial}
\renewcommand{\Re}{\text{Re}}


\let\oldphi\phi \let\phi\varphi \let\varphi\oldphi
\let\oldphi\phi \let\epsilon\varepsilon
\author{Emil Frølich}
\titlez{Unbounded $KK$-theory for real \Cstar-algebras}

\newtheorem{theorem}{Theorem}[section]
\newtheorem{remark}[theorem]{Remark}
\newtheorem{corollary}[theorem]{Corollary}
\newtheorem{lemma}[theorem]{Lemma}
%\newtheorem{theorem}[theorem]{Theorem}
\newtheorem{example}[theorem]{Example}
%\newtheorem{definition}{Definition}%[section]
\newtheorem{definition}[theorem]{Definition}
\newtheorem{proposition}[theorem]{Proposition}
\newtheorem{assumption}[theorem]{Assumption}
\newtheorem{conjecture}[theorem]{Conjecture}

\begin{document}
\tableofcontents
\newpage
\setcounter{page}{1}
%\section{Introduction}
\begin{abstract}
	%In this thesis we give a brief introduction to the theory of real \Cstar algebras, including a brief review of the classification of real continuous trace algebras. We then develop on the work in \cite{mesrennie} to develop the unbounded Kasparov product for real \Cstar algebras. We show that the construction works almost exactly as in \cite{mesrennie}. 
	In this thesis we construct the unbounded Kasparov product in the real setting, based on the work in the complex setting of \cite{mesrennie}. We recreate the results therein, and where the real and complex settings diverge, we prove analogous results in the real setting. Thereby we create the framework needed for the general lifting construction of the unbounded Kasparov product in \cite[Section 4]{mesrennie}.
	To develop a feel for the theory of real \Cstar algebras, we give the classification of real graded continuous trace from \cite{moutou}.   We also give an introduction to $KK$-theoretic duality, introducing the notion of non-commutative Poincare duality and propose classes implementing Poincare duality for the non-commutative 2-torus in the real setting.
\end{abstract}
\section*{Introduction}
\addcontentsline{toc}{section}{Introduction}
There has been renewed interest in real \Cstar algebras and their topological invariants owing to their new-found applications in condensed matter physics, \cite{bourne}, where the hope is to use the unbounded Kasparov product to put various topological invariants stemming from physical considerations on a firm mathematical footing. Non-commutative differential geometry might be the natural setting in which to study many condensed matter systems, as the dynamics of condensed matter systems may often be described by unbounded non-commuting derivations. The success of this approach is evident in the description of the Integer Quantum Hall Effect, where the only consistent theoretical model is that of \cite{bellissard}.

The study of real \Cstar algebras has a different flavor than the study of complex \Cstar algebras. The difference may essentially be ascribed to the superfically trivial fact that $-1$ and $1$ are not homotopic through unitaries in the reals, so the group of unitaries of $\R$ is $\Z_2$ There is also the essential difference that there are three different real normed division algebras, in contrast to the single complex one. 
The continuous algebras provide us with a class of algebras which is particulary well-suited to illustrate the pecularities of real \Cstar algebras. Continuous trace algebras also serve to illustrate the advantages of the bundle theoretic viewpoint, where the ideas of classical geometry may be brought to bear on non-commutative problems. The classification of continuous trace algebras via cohomological means was completed in the 1980's, as summarized in \cite{raeburncont}. The classification was expanded in \cite{moutou} to real continuous trace algebras with groupoid actions. In this thesis we reproduce the classification of Real Graded continuous trace algebras from \cite{moutou}, showing that the Brauer group of a space $(T,\tau)$ with involution classifies the continuous trace algebras with primitive ideal space $(T,\tau)$ up to stable isomorphism.

One of the most important unifying themes across the thesis is the use of the bundle-theoretic viewpoint as enabling an analysis of \Cstar algebraic properties, both directly for the continuous trace algebras and in the use of the local-global principle in constructing the unbounded Kasparov product. This viewpoint also has computational perspectives in condensed matter physics, as outlined in \cite{prodan}.


Within the past few years there has been substantial progress in the unbounded approach to $KK$-theory, beginning with the PhD Thesis of Bram Mesland as elucidated in \cite{mesland}.
and \cite{unboundkasp}. This development has hinged on several technical developments drawing on ideas from ordinary differential geometry. 
To work with differential structures the appropriate setting is found to be the categories of operator algebras and modules, serving as non-commutative generalization of differential algebras and bundles. 
In \cite{mesland} a crucial ingredient in the construction of the unbounded Kasparov product is found to be the non-commutative analogue of a connection, serving to define a version of $1\tens \D_2$ which is well-defined on the interior tensor product. 
Then to provide a suitable differential structure to accommodate the construction of connections requires a non-commutative analogue of a frame. This analogue is provided by the introduction of projective operator modules, which come with a canonical frame. 
The final technical tool in the construction is the Local-Global principle, developed separately in \cite{pierre} and \cite{locglob} which is used to show that the operator implementing the unbounded Kasparov product is self-adjoint and regular.
The framework in which these ideas come together is the notion of an unbounded $\A-\B$-correspondence, which turns out to be the essential ingredient in the construction of the unbounded Kasparov product. 
It was shown in \cite{mesrennie} that using the tools we have just mentioned, one can recover the entirety of complex $KK$-theory in the unbounded setting.
In the thesis we take these results and generalize them to incorporate the real setting, showing that we can also construct the unbounded Kasparov product for real algebras. 
In order to do so, we show a number of results concerning unbounded multipliers and approximate units. These are essential ingredients in \cite[Section 4]{mesrennie} for showing that given any two composable Kasparov modules, we may lift them to composable unbounded modules. With our results, all the proofs go through smoothly in the real case. 
This ends the line of inquiry investigating unbounded $KK$-theory as vehicle for capturing bounded $KK$-theory. What we show is that unbounded $KK$-theory completely captures bounded $KK$-theory in both the real and complex cases, but it does not lend itself to immediate extension to all cases of interest due to the reliance on bounded approximate units in our differential algebras.
This opens the door to the program begun in \cite{jensmorita}, \cite{kaad} and \cite{mesland} of building an abelian group from the unbounded $KK$-theory. The work of \cite{kaad} and \cite{jensmorita} seems to be promising in that it contains both an equivalence relation and a direct sum, but the theory is still not fully formed.  
An interesting further development in real $KK$-theory, of particular interest to the ongoing investigation of physical consequences thereof, is showing $KK$-theoretic Poincare duality for the real non-commutative torus. We have worked on this, but not been able to come to any useful results. 

%overvej at tilføje reel til summary

In summary, the thesis illustrates how ideas from commutative geometry may be brought to bear on seemingly intractable problems in non-commutative geometry, as well as hinting at how the operator theoretic viewpoint leads to new insights in commutative differential geometry. The thesis also illustrates the added flexibility stemming from working with unbounded operators.

\newpage

\newpage
\section{A brief introduction to real \Cstar algebras}
\subsection{Definitions and basic properties}
In order to define unbounded $KK$-theory in the real setting, it is necessary for us to define graded real \Cstar and explore some of their representation theory. In this section we shall give such a basic exposition, for a more in-depth review of the theory of real \Cstar algebras, we refer to \cite{goodearl}. 
\begin{definition}
A real \Cstar-algebra $A$ is a Banach $^*$-algebra $A$ over $\R$ $^*$-isomorphic to a subset of $B(H)$ for a real Hilbert space $H$. 
\end{definition}
\begin{remark}
	A \Cstar algebra or Hilbert space is real if it has $\R$ as scalar field. 
\end{remark}
\noindent As in the complex case, it can be shown that the concrete definition is equivalent to a purely algebraic definition:
\begin{theorem}
A real Banach $^*$-algebra is a real \Cstar algebra if and only if it satisfies: 
\begin{enumerate}
\item
	$\norm{x^*x}=\norm{x}^2$
\item
	$1+x^*x$ is invertible in $A^+$, the unitization of $A$,. 
\end{enumerate}
\end{theorem}
For a proof, see \cite{goodearl}.
\begin{remark}
	The assumption that $1+x^*x$ is invertible is necessary and does not follow from the other axioms. For instance we could consider the complex numbers as a real \Cstar-algebra with trivial involution, and thereby we would have $0<\norm{1+x^*x}=\norm{1+i^2}=0$. Henceforth all \Cstar-algebras considered will be real, unless stated otherwise. 
\end{remark}
The motivating geometric example for the construction of real \Cstar is viewing them as non-commutative generalizations of Atiyah's real spaces. 
\begin{definition}[Atiyah's Real spaces]
	A Real space is a pair $(X,\tau)$ consisting of a locally compact Hausdorff space $X$ with a continuous involution $\tau$. The set of continuous Real functions from $(X_1,\tau_1)$ to $(X_2,\tau_2)$ is defined as 
	\begin{align*}
		C_R(X_1,X_2)=\{f\in C(X_1,X_2) \mid f(\tau_1(x))=\tau_2(f(x))\}
	\end{align*}
	When considering morphisms sets between real spaces, we shall always suppress the $R$ and often suppress the explicit involutions.  
\end{definition}
\noindent Given a real space we can define a real \Cstar algebra.
\begin{definition}[\Cstar version of Real spaces]
	Let $(X,\tau)$ be a Real space. Define the algebra:
	\begin{align*}
		C_0(X,\tau)=\{f\in C(X,\C) | f(\tau(x))=\overline{f(x)}\}
	\end{align*}
\end{definition} 
\begin{example}
	\begin{enumerate}
	\item 
		Given a Real space, the algebra $C_0(X,\tau)$ is a real \Cstar-algebra, with $(f)^*(x)=\overline{f(x)}$. 
	\item
		The field $\R$ is itself a real \Cstar-algebra, equipped with the star $x^*=x$. The complex numbers are a real \Cstar algebra, with involution $z^*=\overline{z}$. The quarternions $\H$ are a real \Cstar algebra, with involution defined on generators as $i^*=-i,j^*-j,k^*=-k$ and $1^*=1$. 
		
		The algebras $M_n(\R$, $M_n(\C)$ and $M_n(\H)$ are all real \Cstar-algebras, with the $^*$-operation stemming from the entrywise-$^*$-operation composed with the transpose. 
	\item
		There is a real version of the multiplier algebra, which may be defined exactly as in the complex case, viewing $A$ as a Hilbert \Cstar-module over itself, i.e. $\ip{a}{a'}=a^*a$. Then we define the multiplier algebra of $A$ as the adjointables on $A$, $M(A)=L(A)$. 
	\end{enumerate}
\end{example}
\noindent As in the complex case, we have a direct classification of the finite dimensional real \Cstar-algebras 
\begin{theorem}
	Every finite dimensional real \Cstar-algebra is a direct sum of matrix algebras over the quarternions, the reals and the complex numbers. 
\end{theorem}
Thus we naturally also have a notion of real AF and UHF algebras but with a slightly more complicated classification theory, \cite{giordano}. 
\begin{definition}
	Given a real \Cstar algebra $A$, define $A_\C=A\tens_\R \C$. This is a complex \Cstar algebra with $^*$-operation given by: 
	\begin{align*}
		(a+ib)^*=a^*-ib^*
	\end{align*}
	and norm 
	\begin{align*}
		\norm{a+ib}_{A_\C}^2=\norm{a^*a+b^*b}_A
	\end{align*}
\end{definition}
The continuous functional calculus is an essential element in the toolkit of operator algebras, and we have a version of it for real \Cstar-algebras: 
\begin{definition}
	Suppose $A$ is a real \Cstar algebra. Define the spectrum of an element $a\in A$ to be the spectrum of $a$ in $A_\C$.
\end{definition}
We can define the state space and irreducible representations of real \Cstar algebras analogous to the complex case:
\begin{definition}
Let $A$ be a real \Cstar algebra. 
\begin{enumerate}
\item  A state on $A$ is a real positive linear functional $\rho:A\to \R$ with $\norm{\rho}=1$. A state which cannot be written as a convex combination of other states is a pure state. 
\item 	A representation of $A$ \Cstar algebra is a $^*$-homomorphism $A\to B(H)$, where $H$ is a real Hilbert space. A representation is irreducible if $\pi(A)$ has no invariant subspaces except for $H$. 
\item Let $\hat{A}$ denote the space of unitary equivalence classes of irreducible representations of $A$. 
\item Define $\mathcal{I}(A)$ as the set of ideals of $A$. For $I\in \mathcal{I}$, define $U(I)=\{J\in \mathcal{I}| J\setminus I\neq \emptyset\}$. This is a subbasis for a topology on $\mathcal{I}$ and gives rise to a topology on the space of representations through the map $\pi\mapsto \ker \pi$. The subspace topology on $\hat{A}$ is the Jacobson topology. 
\end{enumerate}
\end{definition}
\begin{remark}
	Let $A$ be a real \Cstar algebra. By the $GNS$-construction every state corresponds to a representation of $A$. Likewise, the pure states correspond to irreducible representations. For a proof of this, see \cite{goodearl}. 
\end{remark}
One of the nice features of real \Cstar algebras is that we have additional structure on $\hat{A}$.
\begin{lemma}
	Let $\pi$ be an irreducible representation of a real \Cstar-algebra $A$ on a real Hilbert space $H$. Then the commutant of $\pi(A)$ is $\R,\C$ or $\H$. 
\end{lemma}
\begin{proof}
	As $\pi(A)'$ is a real \Cstar-algebra, it suffices to show that it is a real division algebra since the only real division algebras are $\R,\C$ and $\H$, \cite{divisionalgebra}. 
	Let $p\in \pi(A)'$ be a spectral projection of a self-adjoint element $x\in \pi(A)'$, this exists by \cite[Remark 20.18]{meise}. Then $pH$ and $(1-p)H$ are both invariant subspaces for $\pi(A)$. As $\pi$ was assumed irreducible, either $p=1$ or $1-p=1$. As all non-trivial spectral projections agree the spectrum of $x$ must be a one-point set, ie. an eigenvalue. Therefore $x$ is of the form $x=\lambda \cdot 1$, for $\lambda \in \R$. If $y\in \pi(A)'$, then $y^*y=\lambda \cdot 1$ for $\lambda \in \R$, and likewise for $yy^*$. The spectra of $yy^*$ and $y^*y$ coincide except possibly for the set $\{0\}$ it follows $yy^*=y^*y=\lambda \cdot 1$ and as such either $y=0$, or $y$ is invertible with inverse $y^{-1}=\norm{y}^{-2}y^*$. This implies that $\pi(A)'$ is a division algebra as desired, showing the theorem. 
\end{proof}
\begin{definition}
	Let $\pi$ be an irreducible representation of a real \Cstar algebra $A$. 
\begin{enumerate}
\item
	$\pi$ is of real type if $\pi(A)'\cong \R$, 
\item
	$\pi$ is of complex type if $\pi(A)'\cong \C$, 
\item
	$\pi$ is of quartenionic type if $\pi(A)'\cong \H$, 
\end{enumerate}	
\end{definition}
\begin{definition}
	A Real \Cstar-algebra is a pair $(A,\tau)$ where $A$ is a complex \Cstar-algebra and $\tau$ is a complex-linear involution such that:
	\begin{align*}
		\tau(ab)&=\tau(b)\tau(a) \\
		\tau(a^*)&=\tau(a)^*
	\end{align*}
	Such a map is a $^*$-anti-automorphism. 
\end{definition}
\begin{remark}\label{equivalent}
	The categories of real and Real \Cstar algebras are equivalent via. the following construction:
	Given a real \Cstar algebra $A$ with complexification $A_\C$, define the anti-linear $^*$-automorphism: 
	\begin{align*}
		&\sigma:A_\C\to \A_\C \\
		&\sigma(a+ib)=a-ib
	\end{align*}
	From this, define the $^*$-anti-automorphism $\tau$:
	\begin{align*}
		\tau(a+ib)&=(\sigma(a+ib))^*=a^*+ib^*
	\end{align*}
	Then it is easy to see that $(A_\C,\tau)$ defines a Real \Cstar algebra.
	Conversely, given a Real \Cstar algebra $(A,\tau)$, define: 
	\begin{align*}
		&\sigma:A\to A \\
		&\sigma(a)=(\tau(a))^* \\
		&A_{real}=\{a\in A: \sigma(a)=a\}
	\end{align*}
	Then the algebra $A_{real}$ is a real \Cstar algebra, and the two constructions are inverses of each other. 
	The construction shows that a Real \Cstar algebra $(A,\tau)$ is equivalent to $(A,\sigma)$ where $A$ is a complex \Cstar algebra and $\sigma$ is an anti-linear involution commuting with the adjoint. 
\end{remark}

\begin{definition}\label{realhilbertspace}
	A Real Hilbert space is a pair $(H,\sigma)$, where $H$ is a complex Hilbert space and $\sigma$ is anti-linear involution on $H$. 
\end{definition}
\begin{remark}
	The categories of real and Real Hilbert spaces are equivalent via. the following construction:
	Given a real Hilbert space $H$, define $(H,\sigma)$ as $H\tens \C$ with involution $\sigma(\xi\tens z)=\xi \tens \overline{z}$. 
	Conversely, given a Real Hilbert space $(H,\sigma)$ define $H_{real}=\{\xi \in H \mid \sigma(\xi)=\xi \}$.
\end{remark}
The definition of Real Hilbert spaces allows us to define the Real version of $B(H)$
\begin{definition}\label{realhilbert}
	Suppose $(H,\sigma)$ is a Real Hilbert space. Define $B(H)$ the $^*$-anti-automorphism $\theta$ on $B((H,\sigma))$ for $x\in B(H)$ as: 
\begin{align*}
	\theta(x)=\sigma(x^*)\sigma
\end{align*}
\end{definition}
\begin{definition}
	Suppose $(A,\tau)$ is a Real \Cstar algebra, and $\pi:A\to B(H)$ is a complex $^*$-representation of $A$ on a Real Hilbert space. Then $B(H)$ comes with a $^*$-anti-automorphism $\theta$, see \Cref{realhilbert}. 
	If $\pi$ satisfies
	\begin{align*}
		&\pi(\tau(a))=\theta(\pi(a))
	\end{align*}
	$\pi$ is a Real representation. 
\end{definition}
\begin{definition}
	Given a $^*$-algebra $A$ over $k$, define the set $A^{op}$ as the set $A$. We equip $A^{op}$ with the structure of a Real algebra. Let $a^{op}, b^{op}\in A^{op}$ and define the operations: 
	\begin{align*}
		&(ab)^{op}=b^{op}a^{op} \\
		&\lambda a^{op}=(\lambda a)^{op} \\
		&(a^{op})^*=(a^*)^{op} \\
		&a^{op}+b^{op}=(a+b)^{op} \\
		&\tau(a^{op})=(\tau(a))^{op}
	\end{align*}
\end{definition}
We can see that a Real \Cstar algebra $(A,\tau)$ satisfies $A\cong A^{op}$ as complex algebras, with the isomorphism implemented by $\tau$. Given a representation of $(A,\tau)$ we  define a representation of the opposite algebra. 
\begin{definition}\label{oppositerep}
	Let $\pi$ be a representation of a Real \Cstar algebra $(A,\tau)$ on $(H,\sigma)$. Define $\pi^{op}:A^{op}\to B(H)^{op}$:
	\begin{align*}
		\pi^{op}(a^{op})=\pi(\tau(a))
	\end{align*}
\end{definition}
An interesting property of Real \Cstar-algebras is that $\tau$ allows us to define an involution on the space of irreducible representations of a Real algebra. 
\begin{proposition}\label{involution}
	If $(A,\tau)$ is a Real \Cstar-algebra, $\tau$ induces an involution on $\hat{A}$. 
\end{proposition}
\begin{proof}
	Let $\pi$ be an irreducible $^*$-representation of a Real algebra $(A,\tau)\to (H,\sigma)$. Define the map $\pi^{op}: A^{op} \to B(H)^{op}$ as in \Cref{oppositerep}. We have the $^*$-anti-automorphism $\theta$ on $B(H)$, $\theta:B(H)^{op}\to B(H)$ from \Cref{realhilbert}. This allows us to define a new representation $A\to B(H)$:
	\begin{align*}
		\theta_*(\pi)=\theta\pi^{op}(\tau))
	\end{align*}
	We wish to show the map $\theta_*$ is an involution on $\hat{A}$. To this end consider $\theta_*(\theta_*(\pi))$. Unravelling this expression we get:
	\begin{align*}
		\theta_*(\theta_*(\pi))&=\theta_*(\theta(\pi^{op}(\tau))) \\
		&=\theta(\theta(\pi(\tau(\tau)))) \\
		&=\pi
	\end{align*}
%	and the diagram below spelling out all the compositions:
%	\begin{align*}
%		\xymatrix{
%			A\ar[r]^{\tau} & A^{\op} \ar[r]^{\tau} & A \ar[r]^{\pi} & B(H) \ar[r]^{\theta_H}& B(H)^{\op} \ar[r]^{\theta_H} & B(H)
%		}
%	\end{align*}
%	Following the diagram gives us the identity, as $\tau$ and $\theta_H$ are both involutions. 
	Hence $\theta_*$ is an involution on $\hat{A}$. 
\end{proof}
\begin{theorem}[Gelfand-Naimark]\label{gelfandnaimark}
	The category of real commutative \Cstar-algebras with non-degenerate $^*$-homomorphisms as the morphisms is equivalent to the category of locally compact spaces with proper continuous maps as the morphisms. Given a commutative real \Cstar algebra $A$ we define the space $X$ of $\R$-linear characters of $A$, so $A \cong C_0(X,\tau)$ where $\tau$ is the involution on the character space of $A$, defined in \Cref{involution}.
\end{theorem}
\begin{remark}
%	Let $A$ be a commutative \Cstar algebra, so $A\cong C_0(X,\tau)$.  $\theta_*$  the way one recovers the involution $\tau$ on a Real space. Defining $A=C_0(X,\tau)$, we get $(\hat{A},\theta_*)=(X,\tau)$. 
	The space $X$ is homeomorphic to the space of $\C$-linear characters of $A_\C$. 
\end{remark}
These statements combine to give a real version of the continuous functional calculus. 
\begin{proposition}
	Assume $A$ is a real \Cstar algebra. Let $a\in A$ be a normal element in $A$. Then the algebra $C^*(\{x\})$ is a commutative algebra real \Cstar algebra, by \Cref{gelfandnaimark} $C^*(\{x\})\cong C_0(Spec(x),\tau$. Thus $f(x)\in C^*(\{x\})$ for $f \in C_0(Spec(x),\tau)$. 
\end{proposition}
For a proof of these results see \cite{goodearl}. 
The three possible commutants of an irreducible representation is reflected in the involution on the irreducible representations, as encapsulated in the following theorem. 
\begin{theorem}
	Let $\pi$ be an irreducible representation of a real \Cstar algebra $A$ on a real Hilbert space $H$. Let $(A_\C,\tau)$ be the complexification of $A$ with the involution $\tau$ defined as in \Cref{equivalent}.
	\begin{enumerate}
	\item If $\pi$ is of real type, the Real representation $\pi_\C:A_\C\to B(H\tens \C)$ of $A_\C$ induced by $\pi$ is fixed under the involution on $\hat{A}_\C$ defined in \Cref{involution}. 
	\item Suppose that $\pi(A)'\cong \C$. This gives $H$ the structure of a complex Hilbert space, $H$ acquires two different structures of a complex Hilbert space: One by multiplication and one by multiplication with the conjugate. 
	
	The representation $\pi$ is complex-linear with respect to both  of these structures. These structures give two representations $\pi_\C',\pi_\C''$ of $A_\C$ which are inequivalent and interchanged by the involution in \Cref{involution}. 
	\item Assume that $\pi(A)'\cong \H$, and view $H$ as a quartenionic Hilbert space, $H^h$ through the action of $\pi(A)'$. The complexified representation $\pi_\C$ is fixed under the involution in \Cref{involution}.
\end{enumerate}	
We let $\sigma$ denote the complex conjugation in the proof of this theorem, and let $\theta$ be the $^*$-antiautomorphism on $B(H)$ as defined in \Cref{realhilbert}. 
\end{theorem}
\begin{proof}
\begin{enumerate}
\item 
	Assume that $\pi$ is of real type, the commutant of $\pi$ is $\R$ and thus the commutant of the complexification $\pi_\C$ is $\C$. To see that $\pi_\C$ is fixed by the involution, consider the following calculation for $a\in A$ where $^t$ denotes the transpose:
	\begin{align*}
		\theta_*(\pi_\C)(a)=\tau(\pi_\C^{\op}(\theta(a)))=\tau(\pi_\C^{\op}(a^*))=\tau(\pi(a)^t)=(\pi(a)^t)^t
	\end{align*}
	This shows that $\pi_\C$ is fixed. 
	\item
		Consider the case where $\pi$ is of complex type, then it suffices to show that $\theta_*(\pi_\C)$ and $\pi_\C$ are inequivalent representations. We may thus assume for contradiction that they are equivalent. Viewing $\pi$ as an irreducible representation on a complex Hilbert space $H^c$, we may expand it to an irreducible complex representation of $A_\C$, which is equivalent to $\theta_*(\pi^c)$. If we have the element $a+ib\in A_\C$ then $\sigma(a+ib)=a-ib,\theta(a+ib)=a^*+ib^*$. Thus, by our assumption $\theta_*(\pi^c)(a+ib)=\tau(\pi_\C^{\op})(a^*+ib^*)=\overline{\pi(a)}+\overline{\pi(b)}$, where $\overline{\cdot}$ denotes the complex conjugation. 


	We have a canonical identification of $H_\C\ \cong H^c \osum \overline{H^c}$, implying that the complexification of $\pi$ may be canonically identified with $\pi^c\osum \theta_*(\pi^c)$ by our previous calculations. For this to be equivalent to $\pi^c\osum \pi^c$, would require the commutant of its image to be isomorphic to $M_2(\C)$. However, the commutant of $\pi_\C$ must be isomorphic to $\pi(A)'\tens \C$, leading to the desired contradiction. 
\item 
	The last case we need to consider is the case where $\pi$ is of quartenionic type. In this case the commutant of $\pi_\C$ commutant is $\H\tens \C\cong M_2(\C)$, with the isomorphism given by mapping 
	\begin{align*}
	&i\tens i \mapsto e_{12}+e_{21} \\
	&j\tens i \mapsto e_{11}-e_{22} \\
	&k\tens i \mapsto -ie_{12} +ie_{21} 
	\end{align*}
	That $\pi_\C(A_\C)'\cong M_2(\C)$ implies that $\pi_\C: A\to H\tens \C$ is equivalent to $\tilde{\pi}_\C\osum \tilde{\pi}_\C :A\to V \osum V$, where $V$ is a complex Hilbert space and $\tilde{\pi}_\C$ is irreducible on $V$. %We may thus write $\pi_\C$ as the matrix:
%	\begin{align*}
%		\pi_\C=\begin{pmatrix} \tilde{\pi}_\C & 0 \\ 0 & \tilde{\pi}_\C \end{pmatrix}
%	\end{align*}
	This reduces our problem to showing that $\tilde{\pi}_\C \osum \tilde{\pi}_\C$ is fixed by the involution. To see this, start by noting that the involution is $\sigma \osum \sigma$. We note that the projections onto $e_{11}$ and $e_{22}$ corresponds to $\frac{1}{2}(1+i)$ and $\frac{1}{2}(1-i)$, under the identification $\C\cong \H \tens M_2(\C)$. Thus in a slight abuse of notation:
	\begin{align*}
		\pi_\C=\frac{1}{2}(1\tens 1+j\tens i)\tilde{\pi}_\C+\frac{1}{2}(1\tens 1-j\tens i)\tilde{\pi}_\C
	\end{align*}
	It is clear that $\sigma\osum \sigma$ interchanges these, showing that $\pi_\C$ is fixed under the involution up to unitary equivalence. 
	\end{enumerate}  
\end{proof}
Before we proceed, we need to introduce the notion of a graded \Cstar algebra and the graded commutator as well as the graded tensor product. 
\begin{definition}
\begin{enumerate}
\item Suppose that $\Gamma$ is a discrete group and $A$ is a $k$-algebra. The algebra $A$ is $\Gamma$-graded if there is a decomposition of $A$ into $k$-vector spaces $A^{(g)}$ such that:
\begin{align*}
	&A=\bigoplus_{g \in \Gamma} A^{(g)} \\
	&A^{(g)} A^{(g')}\subset A^{(g g')}
\end{align*}
In the case where $\Gamma$ is $\Z_2$, the grading can equivalently be given by an automorphism $\gamma:A\to A$ such that $\gamma(A^{(0)})=A^{(0)}$, and $\gamma(A^{(1)})=-A^{(1)}$. From now on we will restrict ourselves to the case $\Gamma=\Z_2$.  
%\item	Let $A$ be a \Cstar-algebra with a $\Z_2$ action $\gamma$. A grading on $A$ is a decomposition $A=A^{(0)}\osum A^{(1)}$, where $A^{(0)}$ is the eigenspace of $1$ and $A^{(1)}$ is the eigenspace of $-1$.  
\item	Suppose $A$ is a graded \Cstar algebra. An element $x\in A^{(i)}$ is homogeneous, and every element in $A$ may be written as a finite sum of homogeneous elements. Given a homogeneous element $x$, we denote its degree as $\deg x$. 
\item	The grading is called inner if the grading automorphism, is implemented by a self-adjoint unitary $g\in M(A)$ such that $\gamma(a)=gag^*=(-1)^n a$ for every $a\in A^{(n)}$. This operator is known as the grading operator. In case the grading is not implemented by a self-adjoint unitary in $M(A)$, we shall say that the grading is outer. 
\item 	 A $^*$-subalgebra is called a graded $^*$-subalgebra if it is invariant under the grading operator.
\item	 An ungraded \Cstar-algebra $A$ can be viewed as a graded algebra by setting $A^{(1)}=0$.  
\item 	 Let $B$ be a graded \Cstar algebra. A Hilbert $B$-module $E_B$ is graded if there is a decomposition of $E_B$ such that:
	\begin{align*}
		&E_B=E_B^{(0)}\osum E_B^{(1)} \\
		&B^{(i)}E_B^{(j)}\subset E_B^{(j+i)} \\
		&\ip{}{}_B:E_B^{(i)}\times E_B^{(j)}\to B^{(i+j)}
	\end{align*}
	As in the \Cstar-algebraic case, the grading can equivalently be given by an automorphism $\gamma:E_B\to E_B$ such that $\gamma(E_B^{(0)})=E_B^{(0)}$, and $\gamma(E_B^{(1)})=-E_B^{(1)}$. From now on we will restrict ourselves to the case $\Gamma=\Z_2$. 
\item 	 Let $B$ be a graded \Cstar algebra and let $E_B,F_B$ be graded Hilbert modules. Define $L(E_B,F_B)$:
	\begin{align*}
		&L^{(i)}(E_B,F_B)=\bigoplus_{k,j \in \Z_2,~k+j=i} L(E_B^{(k)},F_B^{(j)}) \\
		&L(E_B,F_B)=L^{(0)}(E_B,F_B)\osum L^{(1)}(E_B,F_B)
	\end{align*}
\end{enumerate}
\end{definition}
\begin{definition}
	Given two graded \Cstar-algebras $A$ and $B$ a $^*$-homomorphism $\phi:A\to B$, $\phi$ is graded if $\phi(A^{(i)})\subset B^{(i)}$. 
\end{definition}
\begin{definition}
	Let $x,y$ be homogeneous elements. We define the graded commutator as 
	\begin{align*}
		[x,y]=xy-(-1)^{\deg(x)\deg(y)}yx
	\end{align*}
\end{definition}
\begin{proposition}
The graded commutator satisfies the following relations on homogeneous elements. 
\begin{enumerate}
	\item
		$[x,y]+(-1)^{\deg x\deg y}[y,x]=0$
	\item
		$[x,yz]=[x,y]z-(-1)^{\deg x \deg y}y[x,z]$
	\item
		$(-1)^{\deg x \deg z}[[x,y],z]+(-1)^{\deg x \deg y}[[y,z],x]+(-1)^{\deg y \deg z}[[z,x],y]=0$
\end{enumerate}
\end{proposition}
\begin{proof}
We shall only show the second equality as the calculations become progressively longer and the nature of the relevant calculations is illustrated in this calculation. 
%\begin{enumerate}
%\item	
%	Let $x,y$ be homogeneous elements. Then 
%	\begin{align*}
%		[x,y]+(-1)^{\deg x \deg y}[y,x]&=xy-(-1)^{\deg x \deg y}yx+((-1)^{\deg x \deg y}yx -(-1)^{\deg x \deg y+\deg x \deg y}xy) \\
%		&=xy-(-1)^{\deg x \deg y+\deg x \deg y}xy \\
%		&=0
%	\end{align*}
%\item	
Let $x$, $y$ and $z$ be homogeneous elements, and remark that $yz\in A^{(\deg y+\deg z)}$. Thus we get
\begin{align*}
	xyz-(-1)^{\deg x (\deg y+\deg z)}yzx&=xyz-(-1)^{\deg x (\deg y+\deg z)}y((-1)^{\deg x\deg z} xz+[z,x]) \\
	&=[x,y]z-(-1)^{\deg x (\deg y+\deg z)}y[x,z]
\end{align*} 
as desired. 
%\end{enumerate}
\end{proof}
\begin{definition}
	Given graded \Cstar-algebras $A$ and $B$, we may form their algebraic tensor product $A\odot B$. We equip it with the grading defined on homogeneous elements as: $\deg(x\tens y)=\deg(x)+\deg(y)$. We define the product on homogeneous elements by 
	\begin{align*}
		(a_1\tens b_1)(a_2\tens b_2)=(-1)^{\deg b_1 \deg a_2}(a_1a_2\tens b_1b_2)
	\end{align*}
	and we extend it by linearity to the entirety of the algebraic tensor product.
	Suppose $\pi,\rho$ be graded faithful representations of $A$ and $B$ on $H$ and $K$ respectively, where $H$ and $K$ are graded Hilbert spaces. We define $\pi \tens \rho$ on $B(H\tens  K)$. 
	Completing in the norm derived from this representation gives us the minimal graded tensor product of $A$ and $B$. The proof of the fact that this is well-defined is essentially the same as for ungraded \Cstar algebras, and may be found in \cite{kasparovhilbert}. 
\end{definition}
\textbf{All tensor products in this thesis are minimal.}
\begin{definition}
	Suppose that $A$ and $B$ are graded \Cstar algebras, and consider their tensor product $A\tens B$. Let $E_A$ and $E_B$ be graded Hilbert modules over $A$ and $B$. 
	Consider the algebraic tensor product $E_A\odot E_B$, with the grading defined on homogeneous elements as: $\deg(x\tens y)=\deg(x)+\deg(y)$.
	This is an $A\tens B$ module through the action of $a\tens b\in A\tens B$ on $x\tens y\in E_A\odot E_B$ as:
	\begin{align*}
		(x\tens y)(a\tens b)=(-1)^{\deg y \deg a} (xa\tens yb)
	\end{align*}
	We define the exterior tensor product $E_A\tens E_B$ as the completion of $E_A\odot E_B$ with respect to the inner product:
	\begin{align*}
		\ip{(x_1\tens x_2)}{(y_1\tens y_2)}=(-1)^{\deg x_2(\deg x_1+\deg x_2)}\ip{x_1}{y_1}\tens \ip{x_2}{y_2}
	\end{align*}
\end{definition}
\begin{definition}
	Let $A$ and $B$ be graded \Cstar-algebras, and let $E_A$ and $E_B$ be $A$ and $B$-Hilbert modules respectively. Assume that we have a representation $\pi: A\to L(E)$, and define the sesquilinear map: 
	\begin{align*}
		&\ip{\cdot}{\cdot}_{int}: E_A\odot E_B \to B \\
		&\ip{x_1\tens y_1}{x_2\tens y_2}_{int}=\ip{y_1}{\pi(\ip{x_1}{x_2}_A)y_2}_B
	\end{align*}
	Define the nullspace of the semi-norm $\norm{\ip{\cdot}{\cdot}_{int}}$:
	\begin{align*}
		\mathcal{N}=\{z\in E_A\odot E_B : \ip{z}{z}_{int}=0\}
	\end{align*}
	We define the interior tensor product:
	\begin{align*}
		E_A\tens_{\pi} E_B=\overline{(E_A\odot E_B)/\mathcal{N}}^{\norm{\ip{}{}_{int}}}
	\end{align*}
	This is a Hilbert $B$-module with inner product $\ip{}{}_{int}$. We shall suppress the subscript ${}_{int}$ from the notation from now on. Often we will write $\tens_A$ instead of $\pi$ when the representation is unambigious.
\end{definition}
For a proof that these tensor-products are well-defined, we refer to \cite{lance95}. Throughout the thesis all tensor products will be graded, unless stated otherwise . 
To bring back the concreteness of what we are working with, we have some examples of real \Cstar algebras: 
\begin{example}
\begin{enumerate}
\item
	We define the rotation algebras as $C(S^1,\R)\rtimes \Z$ with $\Z$ acting by rotation by $n\theta$. If $\Z$ is acting by rotation with an irrational angle, we get the real irrational algebra.% an analogous twisted crossed products with $\Z$ have attracted a substantial amount of attention in recent years as models for a materials acting as topological insulators. 
\item	
We may also consider a commutative torus algebra, namely $C(S^1,\R)\tens C(S^1,\tau_0)$, with $\tau_0$ the complex conjugation on the circle. One should notice that this is just of one three different choices we could have taken for a real algebra reflecting the torus, showing once again the wide range of choices afforded by choice of involution. 	
	This is contrast to the complex case, where we get $C(S^1)\tens C(S^1)$ no matter what, leading to a different geometry as reflected in the $K$-theory of these algebras. %We will return to these algebras again later in the thesis.
\item
	Consider the space $\ell^2(\N,\R)$ and the operator $Se_k=e_{k+1}$. We define $\T=C^*(\{S\})$. This fits into an extension as in the complex case, see \cite[Chapter 1]{schroder}:
	\begin{align*}
	\xymatrix{
		0\ar[r] & \K \ar[r] & \mathcal{T} \ar[r] & C(S_1,\tau_0) \ar[r] & 0
	}
	\end{align*}
	as in the complex case. 
\item
	We construct higher-dimensional non-commutative analogues of the Toeplitz algebra which also fit into extensions. Let $S_1,\dots,S_n$ be a family of unilateral shifts with orthogonal ranges. As in the complex case this gives rise to the extension, see \cite[Chapter 1]{schroder}: 
	\begin{align*}
	\xymatrix{
		0\ar[r] & \K(H) \ar[r] & C^*(\{S_1,\dots,S_n\}) \ar[r]  & O_n \ar[r] & 0
		}
	\end{align*}
	This defines $O_n$, the $n$'th real Cuntz algebra. 
\item
	The prototypical example of graded real \Cstar-algebras is given by the Clifford algebras $Cl_{p,q}$, as in the appendix on Clifford algebras.
\item
	We define the real suspension and anti-suspension as the tensor product with:
	\begin{align*}
		&C_0(\R,\id) \\
		&C_0(\R,-\id) 
	\end{align*}
\end{enumerate}
\end{example}
For $K$-theory the algebras encoding the suspension are also essential and provide good examples of how different involutions on the underlying topological space $X$ induce entirely different \Cstar-algebras. 
For further examples, and an introduction to the construction of crossed products in the real case, we refer to \cite[Ch. 1]{schroder}. We end our brief exposition of the fundamental theory of real \Cstar algebras with a theorem which will allow us to switch seamlessly between the complex and the real case. For notation and results regarding unbounded operators on Hilbert modules, we refer to \cite[Chapter 9, Chapter 10]{lance95}.
\begin{theorem}\label{complextoreal}
	Let $B$ be real \Cstar-algebra, and let $E_B$ be a Hilbert $B$-module.
\begin{enumerate}
		\item
			The operator $\D:\dom \D \to E_B$ is self-adjoint and regular if and only if $\D\tens 1$ is self-adjoint and regular as an operator $\dom \D \tens \C \to E_{B \tens \C}\cong E_{B}\tens \C$. 
		\item Let $\D$ be self-adjoint and regular on $E_B$. Then the operator $(\D^2+1)^{-1}$ is compact if and only if $(\D+i)^{-1}$ and $(\D-i)^{-1}$ are compact.
		\item The map sending $T\mapsto T\tens 1:$ is an isometry $L(E_B)\tens \C \to L(E_B \tens \C)$. 
\end{enumerate}
\end{theorem}
\begin{proof}
	Remark that all tensor products in this proof are ungraded. 
	\begin{enumerate}
	\item
		Start by noting that $E_B\tens_B B \cong E_B$, thus $E_B \tens \C \cong E_B \tens_B B \tens \C \cong E_{B\tens \C}$.  
	
		Assume that $\D$ is self-adjoint and regular as a map $\dom \D \to E_B$. Then consider the map $\D\tens 1:(\dom \D)\tens \C$. As we have assumed that the the mapping $1+\D^*\D$ is surjective, it follows that the mapping $1\tens 1+\D^* \D \tens 1$ is also surjective, giving regularity. To see self-adjointness, consider any element $x\in \dom ((\D\tens 1)^*)$. This element must satisfy that $\ip{x}{(\D \tens 1) y}$ is well-defined for all $y\in \dom \D \tens \C$. However, $x$ must be of the form $\xi_1\tens 1$+ $\xi_2\tens i$, and we may assume $y=\eta \tens 1$, for $\eta \in \dom \D$, as the domain is a linear space. Thus we have 
		\begin{align*}
			\ip{x}{(\D \tens 1 )y}=\ip{\xi_1\tens 1}{(\D\tens 1) \eta_1} -1\tens i\ip{\xi_2\tens 1}{(\D\tens 1)(\eta_1\tens 1)} 
		\end{align*}
		This gives an element of $B\osum iB$ and in order for this element to be well-defined, both terms must lie in $B$. Thus we get that $\xi_i\in \dom \D^*$, and by self-adjointness of $\D$ they must lie in $\dom \D$. 
		
		Conversely, assume that $\D\tens 1:\dom \D \tens \C \to E_B\tens \C$ is self-adjoint and regular. Start by assuming $\D\tens 1$ is self-adjoint, and consider $\D\tens 1$ as a map $E_B\osum iE_B \to E_B\osum iE_B$, ie $\D\tens 1=\D\osum \D$. Then self-adjointness implies that $(\D \osum \D)^*=\D \osum \D$. Therefore $\D^*=\D$, and thus $\D$ is self-adjoint. 
		
		For regularity, let $x\in E_B$, then there is an element $\xi\tens z \in E_{B}\tens \C$ such that $((\D^*\D +1)\tens 1) \xi\tens z=x\tens 1$. Then 
		\begin{align*}
			((\D^*\D +1)\tens 1) (\xi \tens \overline{z}+\xi \tens z)&=x\tens 1 \\
			((\D^*\D+1) \tens 1) \xi' \tens 1= x\tens 1
		\end{align*}
		Thus $(\D^*\D+1)\xi'=x$, showing regularity.
	\item
		It is clear by the $C_0(\R)$-functional calculus for self-adjoint regular operators that if $(\D\tens 1\pm 1\tens i)^{-1}$ are compact, then $(\D^2\tens 1+1\tens 1)^{-1}=(\D\tens 1\pm 1\tens i)^{-1}(\D\tens 1\mp 1\tens i)^{-1}$ is compact. For the converse, assume that $(\D^2+1)^{-1}$ is compact. Consider the operator 
		\begin{align*}
			\D(\D^2+1)^{-1/2}
		\end{align*}
		This is bounded, as it is the bounded transform of a self-adjoint regular operator, see \cite[Chapter 9]{lance95}. 
		By the functional calculus, $(\D^2+1)^{-1/2}$ is also compact, and $(\D-i)^{-1}=(i+\D)(\D^2+1)^{-1}$. Applying the $C_0(\R)$-functional for self-adjoint regular operators, we get:
		\begin{align*}
		&~(\D-i)^{-1} \\
		&=\D(\D^2+1)^{-1}+i(\D^2+1)^{-1} \\
		&~=\underbrace{\D(\D^2+1)^{-1/2}}_{bounded}\underbrace{(\D^2+1)^{-1/2}}_{compact}+i\underbrace{(\D^2+1)^{-1}}_{compact} \\
		&=\underbrace{(\D(\D^2+1)^{-1/2}+i(\D^2+1)^{-1/2})}_{bounded}\underbrace{(\D^2+1)^{-1/2}}_{compact} 
		\end{align*}
		Thus $(\D-i)^{-1}$ is compact. Similarly, $(\D+i)^{-1}$ is compact. 
	\item
		Clear.
	\end{enumerate}
\end{proof}
Henceforth we shall usually suppress the tensor product when considering complexifications and let it be implicit in the notation as to whether we are working with a complexified algebra or not. 
 We now turn our attention to a brief exposition of the theory of real continuous trace algebras, which highlights some of the essential differences between real and complex \Cstar-algebras. 
%Apart from condensed matter theory and the associated coarse geometric setup, one of the places where real \Cstar-algebras come into their own is in the realm of  geometry. That this should happen is maybe not that surprising given the reliance on elliptic operators and harmonic spinors in various essential constructions such as the hodge decomposition and the Atiyah-Singer index theorem. 
%The concrete example we shall be looking at here which explores the interplay between algebraic geometry and real \Cstar-algebras is the class of \Cstar-algebras with continuous trace.

\newpage
\section{Real $K$ and $KK$-theory}
\subsection{Basic properties}
We can now proceed to consider the $K$-theory as well as its corresponding $K$-homology of real \Cstar-algebras in their joint guise of Kasparovs $KK$-theory. There are several reasons for considering unbounded real $KK$-theory. Unbounded real $KK$-theory can detect orientation, and can detect differences between algebras which are equal in the complex case, eg. $\H$ and $M_2(\R)$ which have isomorphic complexifications but are clearly different as real algebras. We start by defining $KKO$-theory. 
\begin{definition}
	We define the Kasparov $(A,B)$ cycles $E(A,B)$ as the triples $E=(E,\pi,F)$ satisfying the following properties
	\begin{enumerate}
		\item
			$E$ is a countably generated $\Z_2$ graded real Hilbert $B$ module 
		\item
			The map $\pi$ is a graded real $^*$-homomorphism $A\to L(E)$. 
		\item
			The operator $F$ satisfies that $[\pi(A)F]\subset K(E)$, as well as $\pi(A)(F^2-1)\subset K(E),~(F^2-1)\pi(A)\subset K(E)$. 
	\end{enumerate}
	A cycle is degenerate if all commutators in the definition are identically zero. The collection $E(A,B)$ comes with a natural binary operation in the direct sum. 
\end{definition}
\begin{definition}
	Two cycles $E_1=(E,\pi,F')$ and $(E,\pi,F)$ are operator homotopic if there is a strictly continuous family $F_t$ giving rise to a path of cycles $(E,\pi,F_t)$ where $F_0=F,F_1=F'$. We define the equivalence relation $\sim_{oh}$ as the equivalence relation stemming from addition of degenerate modules and operator homotopy. 
\end{definition}
\begin{definition}
	Define $KKO(A,B)$ as the group $E(A,B)/\sim_{oh}$. 
	Define the higher $KKO$ groups by the formula 
	\begin{align*}
		K_{p,q}K^{r,s}O(A,B)=KKO(A\tensh Cl_{p,q},B\tensh Cl_{r,s})
	\end{align*}
\end{definition}
For details and proofs, see \cite{kasparov}. 
We summarize some of the properties of operator $KKO$ theory in the theorem below, for a proof see \cite{kasparov} or \cite{schroder}.
\begin{theorem}
\begin{enumerate}
\item
	Operator $KO_{*}$-theory is a covariant $8$-periodic functor from the category of real \Cstar algebras to the category of abelian groups. The functor $KO$ takes short exact sequences of real \Cstar-algebras to a 24-term cyclic exact sequence. 
\item	Operator $KO^{*}$-homology is an $8$-periodic contravariant functor from the category of real \Cstar algebras to the category of abelian groups. If we have a short exact sequence with a completely positive splitting, $KO^{*}$ applied to the sequence gives rise to a 24-term cyclic exact sequence. 
	\item These two are combined in the stable bifunctor $KKO$, where $KKO(A,\R)\cong KO^{0}(A)$ and $KKO(\R,A)\cong KO_0(A)$. This functor satisfies the following properties.
	\begin{align*}
		&KO_n(A)\cong KO_0(A\tens Cl_{0,n}) \\
		&KO(A)\cong KKO(\R,A) \\
		&KKO(\R,Cl_{p,q} \tens A) \cong KKO(Cl_{q,p},A) 
	\end{align*}
	\end{enumerate}
\end{theorem}
We use the periodicity of $KKO$-theory to derive the following standard results on operator $KO$-theory of the reals, \cite[Section 1]{schroder}. 
\begin{example}
	The $K$-theory of the reals is: 
	\begin{align*}
		&KO_*(\R)=\left \{\begin{array}{c c} 0 & \Z  \\ 1 & \Z_2 \\ 2 & \Z_2 \\  4 & \Z\end{array} \right .
	\end{align*}
	By the isomorphisms in $KKO$-theory, we get \begin{align*}KO^n(\R)\cong KKO(Cl_{0,n}\tens \R,\R)\cong KO_{-n}(\R), \end{align*} so we derive the following table 
	become clear
	\begin{align*}
		&KO_*(\R)=\left \{ \begin{array}{c c} 0 & \Z  \\ -1 & \Z_2 \\ -2 & \Z_2 \\ -4 & \Z \end{array} \right .
	\end{align*}
\end{example}
As described in the section on real \Cstar-algebras, we have both suspensions and anti-suspensions in real bivariant $K$-theory, the actions of which are encapsulated in the following standard result. %One should remark that the proof in \cite{kasparov} foreshadows unbounded $KK$-theory, as all cycles are bounded transforms of unbounded cycles. 
\begin{lemma}
	We have the isomorphisms.
	\begin{align*}
		KKO(A\tens Cl_{0,n},B)\cong KK(A\tens C_0(\R^n,\id),B) \\
		KKO(A\tens Cl_{n,0},B)\cong KK(A\tens C_0(\R^n,-\id),B)
	\end{align*}
	Thus we have the isomorphisms $KO^n(A)=KO(C_0(\R^n)\tens A)$, and $KO^{-n}(A)\cong KO(A\tens C_0(\R,-\id))$. 
\end{lemma}
For a proof, see \cite{kasparov}
We may immediately use these results to calculate the $K$-theory of the algebra $C(S^1,\id)$, giving a flavor of the theory, as well as characterizing how tensoring with $C(S^1,\tau)$ affects the $K$-theory of an algebra. 

\begin{theorem}\label{ktheorys1}
	The $KO$-theory of $C(S^1,\id)$ is given by the following table. 
	\begin{align*}
		KO_n(C(S^1,\id))=\left \{ \begin{array}{c c} \Z\osum \Z_2 & n=0 \\ \Z_2\osum \Z_2 & n=1 \\ \Z_2 & n=2 \\ \Z & n=3,4,7 \\ 0 & n=5,6 \end{array} \right .
	\end{align*}
	For any real \Cstar-algebra $A$ we have the isomorphism 
	\begin{align*}
		KO_n(A\tens C(S^1,\tau_0))\cong KO_n(A)\osum KO_{n-1}(A)
	\end{align*}
	This allows us to compute the $K$-theory of $C(S^1,\tau_0)$.
\end{theorem}
\begin{proof}
	Consider the split short exact sequence, with the split begin given by $x \mapsto (z\mapsto x)$ for $x\in \R$. 
	\begin{align*}
		\xymatrix{
			0\ar[r] & C_0(\R,\id) \ar[r] & C(S^1,\id) \ar[r] & \R \ar[r] & 0 
		}
	\end{align*}
	Thus the 24-periodic exact sequence in $K$-theory reduces to the split short exact sequence 
	\begin{align*}
	\xymatrix{
		0 \ar[r] & KO_{n+1}(\R) \ar[r] KO(C(S^1,\id)) \ar[r] & KO_n(\R) \ar[r] & 0
		}
	\end{align*}
	giving the desired table. 
	Start by considering the split short exact sequence, with the split being given by the map $a\mapsto a\tens 1$.
	\begin{align*}
	\xymatrix{
		0\ar[r] & A\tens C_0(i\R) \ar[r] & A\tens C(S^1,\tau) \ar[r] & A \ar[r] & 0
		}
	\end{align*}
	thus the 24-periodic long-exact sequence reduces to the split short exact sequence
	\begin{align*}
		\xymatrix{
			0\ar[r] & KO_{n-1}(A) \ar[r] & KO_n(A\tens C(S^1,\tau)) \ar[r] & KO_n(A) \ar[r] & 0
		}
	\end{align*}
	giving $KO_n(A\tens C(S^1,\tau))\cong  KO_{n-1}(A)\osum  KO_n(A)$ as desired. 
\end{proof}
The benefit of real $K$-theory over complex $K$-theory is exhibited in these two cases, where we can see a much finer structure on the $K$-theory of the real circle. %The study of classes in $KO_1,KO_2$ was one of the reasons for the interest in real $K$-theory, as they provide topological obstructions to positive scalar curvature of manifolds, which cannot be detected with complex $K$-theory. 

In recent years real $K$-theory has been applied in the realm of topological insulators in order to, once again, explain global behaviors by the presence of these torsion classes. This has also explictly used the construction of the unbounded Kasparov product, \cite{bourne}, \todo{do} in a form which we shall recreate at a later point in the thesis. 
\subsection{$K$-homology of $\R$}
The standard picture of $KK$-theory works very well in purely algebraic context, but in practice the operators that encode the geometry, commutative or otherwise, of a given space are naturally defined as unbounded regular operators. To remedy this shortcoming, it is often preferable to work with unbounded $KK$-theory which is defined from suitable unbounded Kasparov cycles. 
We start by working in the bounded case, and then argue that for suitable operators unbounded operators, the bounded results essentially also hold.  
\begin{definition}
	Let $H$ be a real Hilbert space with the structure of a $Cl_{0,k}$-module, and let $F(H,H)$ denote the odd Fredholm operators on $H$. Define $F_k\subset F(H,H)$ as the subset of $F(H,H)$ which are $Cl_{0,k}$-linear and self-adjoint. 
\end{definition}
We can then define the Clifford index of operators lying in $F_k$, we refer to \Cref{abstheorem} for the notation on $\hat{A}_k$ and $\hat{\M}_k$. 
\begin{definition}
	Given an operator $T\in F_k(H,H)$, we define its Clifford index as the residue class:
	\begin{align*}
		\ind_k(T)=[\ker T]\in \hat{A}_k\cong KO_{k}(\R)
	\end{align*}
	This is well-defined due to the Atiyah-Bott-Shapiro isomorphism, \Cref{abstheorem}.
\end{definition}
In order to justify this definition of the index, we check that $\ind_0$ recovers the usual Fredholm index. 
\begin{example}\label{cliffordkernel}
	We see $Cl_0=\R$ and $Cl_{0,1}=\C$. A $\Z_2$ graded $Cl_0$-module is simply a graded $\R$-vector space, $V_0\osum V_1$. We observe that $[V\osum 0]=-[0\osum V]$ in $\hat{A}_0$ since $V\osum V\cong V\tens \C$ may be extended to a graded $Cl_{0,1}$-module. Given $T\in F_k(H,H)$ we get $\ind_0(T)=[\ker T_0\osum \ker T_1]=[\ker T_0 \osum 0]-[\ker T_1 \osum 0]$, thus recovering the Fredholm Index of $T_0$. 
\end{example}
We wish to show that the index map is well-defined as a map from $K$-homology, so we need it to be homotopy invariant and a group homomorphism. 
\begin{theorem}\label{welldef}
	The Clifford index is constant on connected components of $F_k$, i.e. it is homotopy-invariant. 
\end{theorem}
\begin{proof}
	Let $T\in F_k$, as $0$ is an isolated point in the spectrum we may assume that the non-zero spectrum of $T$ lies outside of $[-2,2]$. Pick a neighborhood $U$ of $T$ in $F_k$ such that for all $S\in U$ we have that $\sigma(S^2)\subset [0,1/2)\cup (1,\infty)$ and $\norm{T^2-S^2}\leq 1/2$. Fix $S\in U$ and let $W$ be the range of the spectral projection of $S^2$ onto $[0,1/2]$. 
	Consider also the orthogonal projection $P:H\to \ker T$. Then we claim that $p:W\cong \ker T$. To do this, pick $v\in W\in (\ker T)^\perp$. We have that the equality 
	\begin{align*}
		\ip{(T^2-S^2)v}{v}\geq \pa{2-\frac{1}{2}}\norm{v}^2\geq \norm{v}^2
	\end{align*}
	As $\norm{T^2-S^2}<\frac{1}{2}$, we get that $\norm{v}<1/2$, and as such we get that that $v=0$. In order to see surjectivity, pick $v\in W^\perp \cap \ker T$ and observe that $\ip{(S^2-T^2)v}{v}\geq \norm{v}^2$ and as such is 0. \todo{Add argument for these inequalities (locally self-adjoint).}
	We see that $W$ is a $\Z_2\Z$ graded submodule of $H$, as well as splitting as $W=\ker S\osum (\ker (S)^\perp \cap W)$. The projection $P$ onto the kernel of $T$ also preserves the graded module structure. Defining $V=(\ker (S)^\perp \cap W)$ we get the equivalence
	\begin{align*}
		\ker T\cong \ker S\osum V
	\end{align*}
	In order to see that the class is the same, we need to give $V$ the structure of $Cl_{k+1}$ module in order to map it to $A_k$.
	Let $S_V=S|_V$. This is a symmetric $\Z_2$ $Cl_{0,k}$-linear graded map. Thus the operator $J=(S^2_V)^{-1/2}S_V$ is as well. We see that $J^2=Id$. Decomposing $V=V_0\osum V_1$, $J=e_{21}J_0+e_{12}J_1$. The graded endomorphism $\tilde{J}=-e_{21}J_0+e_{12}J_1$ squares to $-1$, and as such makes $V$ into a $Cl_{k+1}$-module as desired. Thus $\ind_k T=\ind_k S$ as desired. 
\end{proof}
\begin{comment}
To show that the Clifford index actually gives a homomorphism, ie. that two operators are homotopic if and only if they have the same index, we have the following theorem due to Atiyah and Singer. 
\begin{theorem}
	The Clifford index induces a bijection $\pi_0(\tilde{\F}_k) \to KO^{-k}(\{*\})$
\end{theorem}
\end{comment}
We refer to the following reuslt: \cite[Page 217]{spingeom}.
\begin{proposition}\label{unboundkernel}
	Given an elliptic $Cl_{0,k}$-linear differential operator $\D$, $[\ker(\D)]\in \hat{A}_k=[\ker(\D/(1+\D^2)^{-1/2})]$.
\end{proposition}
Thus it suffices to construct unbounded elliptic operators with the appropriate kernels. 
\begin{theorem}
	The following cycles are generators of the non-trivial real $K$-homology groups of the reals. 
	\begin{enumerate}
	\item
	Define the operator 
	\begin{align*}
		&D:\ell^2(\N,\R)\to \ell^2(\N,\R) \\
		&e_k\mapsto ke_{k+1}
	\end{align*}
	with formal adjoint 
	\begin{align*}
		D^*(e_k)=\left \{\begin{array}{c c} 0 & k=1 \\ \frac{1}{k-1}e_{k-1} &  k\geq 2 \end{array} \right .
	\end{align*}
	The group $KO^0(\R)$ is generated by the cycle 
	\begin{align*}
		E=\pa{\R, \ell^2(\N,\R)\osum \ell^2(\N,\R), \begin{pmatrix} 0 & D \\ D^* & 0\end{pmatrix}}
	\end{align*}
	\item
		The group $KO^{-1}(\R)$ is generated by 
		\begin{align*}
			\pa{Cl_{0,1}, L^2(S^1,\C), \gamma_1 d_\theta}
		\end{align*}
		where $\theta$ is the angular direction on the circle, and $\gamma_1$ is the generator of $Cl_{0,1}\cong \C$. 
	\item
		The group $KO^{-2}(\R)$ is generated by 
		\begin{align*}
			(Cl_{0,2},L^2(S^1\times S^1,\H),\gamma_1 \part_{\theta_1}+\gamma_2 \part_{\theta_2})
		\end{align*}
		where $\theta_i$ are the angular directions on the torus and $\gamma_1,\gamma_2$ are the generators of $Cl_{0,2}\cong \H$.
	\item
		The group $KO^{-4}(\R)$ is generated by:
			\begin{align*}
		(Cl_{0,4},(\ell^2(\N)\osum \ell^2(\N))\tens_\R (\H\osum \H),\D)
	\end{align*}
		Where the representation of $Cl_{0,4}$ is either of the two irreducible graded representations. 

	%	\begin{align*}
	%		(Cl_4,L^2(F_4))\tensh S,\DO,\psi,\pi_{\spin_r})
	%	\end{align*}
	%	where $F_4$ is the Fermat quartic ie. hypersurface defined as 
	%	\begin{align*}
	%		\{(z_0,\dots,z_3)\in \C P^3 | \sum_{i=0}^3 z_i^4=0\}
	%	\end{align*}
	%	and $S$ is the canonical spin bundle, with Dirac operator $\DO$. 
	\end{enumerate}
\end{theorem}
\begin{proof}
We start by noting that for every $n$ the cycles are given by self-adjoint Elliptic $Cl_{0,n}$-linear operators. 
To start, we wish to show that: 
\begin{align*}
	\ind_n(T):KO^n(\R)\to \hat{A}_k
\end{align*}
is a homomorphism. 
We start by checking that it is well-defined on equivalence classes. Invariance under unitary isomorphism is clear, and by \Cref{welldef} the map $\ind_n(T)$ is invariant under operator homotopy. Thus it is well-defined as a function $KO^*(\R)\to \hat{A}_k$
To see that $\ind_n$ is a homomorphism, assume that $(\pi,F,H)$ is a degenerate cycle, so it is $Cl_{0,k}$-linear. Furthermore, degeneracy implies that $F^2=1$ and $F=F^*$. Therefore $F$ is a self-adjoint unitary and and thus has trivial kernel. Hence $\ind_n(F)=0$. 
Additivity is clear. 

We can infer that $\ind_n: KO^n\to \hat{A}_k$ is homomorphism. As we know what the left-hand side is by Bott periodicity, we only need to show that our cycles in $KO^n(\R)$ have the appropriate index. %Therefore we shall be working directly with unbounded operators, \Cref{unboundkernel}. 
\begin{enumerate}
\item
	Consider the cycle 
	%\begin{align*}
	%	E=\pa{C_0((-1,1))\tensh Cl_{0,1},L^2([0,1])\tensh \C, \gamma_1 dx,(f,\gamma)g\mapsto (f(x)g\gamma)}
	%\end{align*}
	%which we consider as a $Cl_0$-module. If we take $\ind_0(\gamma_1 dx)$ of this operator we recover the Fredholm index of the bounded transform of the operator, which written in Fourier basis is 
	%\begin{align*}
	%e_k\mapsto \frac{k}{\sqrt{k^2+1}}e_k
	%\end{align*}
	%and the Fredholm index of this operator is 1. 
	Define the operator 
	\begin{align*}
		&D:\ell^2(\N,\R)\to \ell^2(\N,\R) \\
		&e_k\mapsto ke_{k+1}
	\end{align*}
	with formal adjoint 
	\begin{align*}
		D^*(e_k)=\left \{\begin{array}{c c} 0 & k=1 \\ \frac{1}{k-1}e_{k-1} &  k\geq 2 \end{array} \right .
	\end{align*}
	Consider the cycle 
	\begin{align*}
		E=\pa{\R, \ell^2(\N,\R)\osum \ell^2(\N,\R), \begin{pmatrix} 0 & D \\ D^* & 0\end{pmatrix}}
	\end{align*}
	By \Cref{cliffordkernel} $\ind_0\pa{\begin{pmatrix} 0 & D \\ D^* & 0\end{pmatrix} }$ is the Fredholm Index of $D$, which is readily seen to be 1. Thus $E$ generates $KO^{0}(\R)$. 
\item
	Consider the operator $\D=\gamma_1\tens d_\theta$, in the Fourier basis we can write this on the basis vectors as:
	\begin{align*}
		&\D: \ell^2(\Z)\to \ell^2(\Z) \\
		&e_k\mapsto ke_k
	\end{align*}
	We wish to determine the Clifford index of this operator and thus we take $[\ker(\D)]\in \hat{A}_1\cong \Z_2$. The kernel consists of the constant $Cl_{0,1}$-functions, equipped with the canonical grading. Thus the kernel is $Cl_{0,1}$ viewed as a module over itself. This is is the generator of $\hat{\M}_1$, and thus its image in the quotient $\hat{\M}_1\to \hat{A}_1$ is the generator of $\hat{A}_1$.  
	
%	which by \Cref{spindex} is $\dim_{Cl_{0,2}}(ker(\gamma\tens d_x)) \mod 2$, consisting of the constant real-valued functions and thus is non-zero.
%	\todo{Change well-defined proof to more general, works for all taken from Prop 10.6-Theorem 10.8 in Spin Geometry}

	%which is the quarternionic dimension of the kernel modulo 2, by \Cref{spindex}. As before, the kernel consists of the constant valued functions and as such the class is non-zero. 

\item
	Consider the operator $\D=\gamma_1 \part_{\theta_1}+\gamma_2 \part_{\theta_2}$ on $L^2(S^1 \times S^1,\H)$. If we consider $\D$ as an operator in the Fourier basis, we can write it as	
	\begin{align*}
		\D(k_j e_j,c_n e_n)&=(-\gamma_1\gamma_1 jk_j e_j,-\gamma_2\gamma_1n c_n  e_n) \\
		&=( jk_j e_j,\gamma_1\gamma_2 n  c_ne_n)
	\end{align*}
	where $k_j,c_n\in \H$. Thus $\D f=0$ implies $k_j=c_n=0$ for all non-zero $j,n$. Therefore $f=(k_0,c_0)$ and is thereby a constant $\H$-valued function. We see that the kernel is isomorphic to the constant $\H$-valued functions and as such is the generator of $\hat{\M}_2$, and thereby the generator of $\hat{A}_2$. 
\item
	Let $\pi:M_2(\H)\to \H\tens_{\H\osum \H} M_2(\H)\cong \H\osum \H$ denote either of the two graded irreducible representations of $M_2(\H)$, see \Cref{equivalentcliff}. Define the map $\D:\ell^2(\Z) \tens_\R (\H\osum \H) \to \ell^2(\Z)\tens_\R (\H\osum \H)$ as:
	\begin{align*}
		e_k \tens (h_1,h_2) \mapsto ke_k \tens (h_1,h_2)
	\end{align*}
	The kernel of this map is isomorphic to $\H\osum \H$, which is the generator of $\hat{A}_4$ by construction. 
	Thus the cycle 
	\begin{align*}
		(Cl_{0,4},(\ell^2(\N)\osum \ell^2(\N))\tens_\R (\H\osum \H),\D)
	\end{align*}
	generates $KO^{-4}(\R)$. 
	
	%Alternatively we could invoke the results of spin geometry in order to show the claim, which we have hitherto abstained from. We see by \Cref{spindex} that $\ind_4(\DO)=1/2\hat{A}(F_4)$, which by \cite[Example 2.14]{spingeom} is 1 in this case.  We do not delve further into these results and instead refer to \cite{spingeom} for further details. 
	
\end{enumerate}
\end{proof}
The above calculations can be seen to be a general case of the $KO^{*}$-generalization of the analytic index of Clifford-linear operators from spin geometry. 
\begin{definition}
	Let $x\in KO^n(A)$ and $y\in KO_m(A)$. Then we may take the Kasparov product $x\tens y$ thereby ending in $KKO(Cl_{0,n},Cl_{0,m})\cong KKO(Cl_{n,m},\R)$. Here we may take the Clifford index of the resulting product product operator as a $Cl_{n,m}$-linear Fredholm operator.
\end{definition}
\begin{remark}
	The result above gives a pairing between $KO^*$ and $KO_*$-theory: Let $(A\tens Cl_{*,0},F,H)$ be an element of $KO^*(A)$ and let $p\in KO_0(A)$. Then the operator $\pi(p)\tens 1_{Cl_{0,*}}(F\tens 1_n)\pi(p)\tens 1_{Cl_{0,*}}$ is a self-adjoint $Cl_{0,*}$-linear Fredholm operator on $\pi(p) H^n$.
	For an example of this, see eg. \cite{bourne} where it is also shown this recovers the Kasparov product. 
\end{remark}
\newpage
\section{Construction of the unbounded Kasparov product over real \Cstar-algebras}
The development of unbounded KK-theory is foreshadowed in Kasparov's original paper \cite{kasparov} where all Bott elements are bounded transforms of naturally defined unbounded cycles. The study of unbounded $KK$-theory begins with the paper by Baaj and Julg, \cite{baajjulg} where they define unbounded $KK$-cycles:
\begin{definition}[Unbounded $KK$-cycles]
	We define the set $\Psi(A,B)$ of unbounded Kasparov $A-B$ cycles as the set of quadruples $(E,\pi,\A,\D)$ where $E$ is a graded Hilbert $B$-module and $\A$ is a dense subalgebra, $\pi:A\to L(E)$ is a $^*$-homomorphism, and $\D$ is a degree one unbounded self-adjoint regular densely defined operator on $E$ satisfying that 
	\begin{align*}
		&(1+\D^2)^{-1}\in \{ T\in L(E):\pi(A)T,T\pi(A)\subset K(E)\} \\
		&\pi(\A)\dom(\D)\subset \dom(\D) 
	\end{align*}
	and for all $a\in \A$, the operator $[\D,\pi(a)]$ extends uniquely to an operator in $L(E)$. 
\end{definition}
We recognize the cycles defining the real $K$-homology of $\R$ as unbounded Kasparov cycles.
\begin{remark}
	The commutators and gradings on $A$ and $E_B$ are related: 
	\begin{align*}
		&[\D,\pi(a)]=\D\pi(a)-\pi(\gamma(a))\D \\
		&[\D,\pi(a)]^*=-[\D,\pi(\gamma_A(a^*))] \\
		&[\pi(\gamma_A(a))]=\gamma_{E_B}\pi(a)\gamma_{E_B}
	\end{align*}
\end{remark}
The utility of unbounded Kasparov modules was immediate, as it allowed \cite{baajjulg} to write the product operator in the exterior Kasparov product simply as 
\begin{align*}
	D_1\tens 1+1\tens D_2
\end{align*}
It was an open problem for many years to construct an unbounded version of the interior Kasparov product, but since the work of \cite{mesland} this problem has come ever closer to being solved in full generality through the work of \cite{kaad}, \cite{suijlekom} and various others. In this section, we shall give the hitherto most general construction of the unbounded interior Kasparov product by expanding the results  \cite{mesrennie}  to the real setting. We generally follow their methods closely, only occasionally adapting them to the real case when necessary. 


As it will turn out the natural setting for the unbounded Kasparov product is the category of operator spaces, not \Cstar-algebras so we need to introduce the framework of operator spaces and algebras, in particular complete differential algebras.
\subsection{Operator spaces and algebras}
\begin{definition}
\begin{enumerate}
\item 
	An operator algebra $\A$ is a closed subalgebra of a \Cstar algebra $B$. As we may represent $B$ isometrically on $B(H)$, we may assume $\A\subset B(H)$. 
\item
	An operator $^*$-algebra is an operator algebra $\A\subset B(H)$ with a completely bounded involution $^*:\A\to \A$. This involution will in general not coincide with the involution on $B$.
\item	
	An operator space $X$ is a closed subspace of a \Cstar algebra.
\item 
	Let $\A$ be an operator algebra, and $X$ be an operator space. If there is a continuous (left or right)-action $\A\times X\to X$, $X$ is a (left or right)-$\A$-module. 
\item
	Let $A$ and $B$ be \Cstar-algebras. Suppose that $X\subset A$ and $Y\subset B$ are operator spaces, and let $\psi:X\to Y$ be a linear map. There are unique norms $\norm{\cdot}_n$ on $M_n(X)$ and $M_n(Y)$ stemming from the unique norms on $M_n(A)$ and $M_n(B)$ respectively. 
	Define $\phi_n=\phi \tens 1_n:M_n(X)\to M_n(Y)$. If 
	\begin{align*}
		\sup_{n\in \N} \sup_{\norm{a}_n\leq 1}\norm{\phi_n(a)}_{n}<\infty
	\end{align*}
	Then $\phi$ is said to be completely bounded, with 
	\begin{align*} 
		\norm{\phi}_{cb}=	\sup_{n\in \N} \sup_{\norm{a}_n\leq 1}\norm{\phi_n(a)}_{n}. 
	\end{align*} 
	Likewise, $\phi$ is completely isometric if $\phi_n$ is an isometry for all $n$, and likewise for contractiveness. 
\end{enumerate}
\end{definition}



\begin{remark}
	There are two \Cstar algebras associated to a given operator algebra:
	\begin{enumerate}
	\item
	$C^*(\A)\subset A$, the smallest subalgebra of $A$ containing $\A$. 
	\item
	$A$: The \Cstar-closure of $A$, which stems from viewing $\A$ as a Banach $^*$-algebra and completing in the norm defined from the spectral radius of $a^*a$. 
	\end{enumerate}
\end{remark}
\begin{definition}
Given an unbounded Kasparov module $(\A,E_B,\D)$ with grading $\gamma$ define the algebra $\A_D=\{a\in A| a\dom(\D) \subset \dom(\D), [\D,a]\in L(E_B)\}$. We may equip this with the structure of an operator $^*-$algebra via. the representation 
\begin{align*}
	&\pi_{\D}:\A\to L((E_B)\osum E_B) \\
	&a\mapsto \begin{pmatrix} \pi(a) & 0 \\ [\D,\pi(a)] & \pi(\gamma(a)) \end{pmatrix}
\end{align*} 
With the $^*$-operation given by $\pi_{\D}(a)^*=\pi(\gamma(a))$.  We shall always assume $\A_\D$ to be equipped with the topology coming from $\norm{\pi_\D(a)}_{L(E_B\osum E_B)}$. 
\end{definition}
\begin{assumption}
	The representation $\pi$ will be assumed to be faithful throughout the thesis, as else we would have to consider the the representation 
	\begin{align*}
		a\mapsto a\osum \pi_\D(a) \in A\osum L_B(E_B\osum E_B)
	\end{align*}
	which does not change the results, but clutters the calculations. 
\end{assumption}
As motivation for the definition of the Lipschitz representation, remark that if $a\in C_0^1(\R)$ and $\D$ is the derivative on $\R$, $[\pi(a),\D]=a'$. In general, one should think of the commutators with $\D$ as derivatives with respect to $\D$. 
An important generalization of this is the full Lipschitz algebra associated to a closed symmetric regular operator on a \Cstar module $E_B$, as well as the notion of a differentiable algebra. 
\begin{definition}
We define the Lipschitz algebra associated to a self-adjoint regular operator $\D$:
\begin{align*}
	\lip(\D)=\{T\in L(E_B) \mid T\dom(\D^*)\subset \dom(\D), [\D^*,T]\in L(E_B)\}
\end{align*}
\end{definition}
We can use this to define the general framework of differentiable algebras, which are algebras where the operators behave like differentiable functions. 
\begin{definition}
We define a differentiable algebra $\A$ as a separable operator $^*$-subalgebra of $\lip(\D)$ closed in the topology stemming from $\pi_{\D}$. Projecting onto the first coordinate of $\pi_{\D}(\A)$, the \Cstar closure of $\A$ algebra coincides with the closure of $\A$ as a subalgebra of $L(E_B)$, thereby giving a subalgebra of $A$. 
\end{definition}
We need to define a tensor product on the category of operator spaces, so we may proceed with our constructions in the setting of operator spaces rather than Hilbert \Cstar modules. 
\begin{definition}
	Given two operator spaces $X,Y$ we may define their algebraic tensor product $X\odot Y$. Define the Haagerup norm for $z\in M_n(X\odot Y)$
	\begin{align*}
		\norm{z}^2=\inf \left \{ \norm{\sum_{i=1}^m  x_ix_i^*} \norm{\sum_{i=1}^m y_i^*y_i} : z=\sum_{i=1}^m x_i\tens y_i \right \}
	\end{align*}
	We shall denote the Haagerup norm as $\norm{\cdot}_{\optens}$. Let $\A$ be an operator algebra. If $X$ is a left and $Y$ is a right operator module, define the Haagerup module tensor product:
	\begin{align*}
		X\optens_\A Y
	\end{align*}
	as the quotient of $X\optens Y$ by the closed linear spans of expressions of the form $x\tens ay-xa\tens y$. The Haagerup tensor product serves to make multiplication $X\optens \A\to X$ continuous. 
\end{definition}
\begin{remark}
	In case $E_B$ and $F_C$ are Hilbert modules, with a representation $\pi:B\to F_C$, we have the cb. isomorphism $E_B\optens_B F_C \cong E_B\tens_B F_C$, see \cite{blecher}.
\end{remark}
To see this is well-defined, we refer to \cite{blecher}. 
Likewise, we would like to define the notion of inner product operator modules. 
\begin{definition}
	An inner product operator module $\E$ is a normed right operator module over an operator $^*$-algebra $\mathcal{B}$ with a sesquilinear pairing $\E\times \E\to \mathcal{B}$ such that 
	\begin{align*}
	&\ip{e_1}{e_2b}=\ip{e_1}{e_2}b \\
	&\ip{e_1}{e_2}^*=\ip{e_2}{e_1} \\
	&\ip{e}{e}\geq 0 \text{ in } B\\
	&\ip{e}{e}=0 \Lr e=0
	\end{align*}
	Further we require that $\ip{\cdot}{\cdot}$ satisfies a weak version of the Cauchy-Schwarz inequality for $C>0$: \begin{align*} \norm{\ip{e_1}{e_2}}\leq C \norm{e_1}_{\E}\norm{e_2}_{\E} \end{align*} for all matrix norms. 
\end{definition}
\begin{remark}
	We do not require $\E$ to be complete in the inner product $\ip{}{}_{\E}$, and in general the topology induced from the $\B$-valued inner product will differ from the norm topology on $\E$. 
\end{remark}

The operator-algebraic analogue of having an exhausting net of subspaces in a locally compact Hausdorff space is the following 
\begin{definition}
	Let $\A$ be an operator algebra, then a bounded approximate unit for $\A$ is a net $(u_{\lambda})_{\lambda\in I}$ such that $\sup_{\lambda \in I}\norm{u_{\lambda}}<\infty$ and $\lim_{\lambda\in I}\norm{u_\lambda a-a }=\lim_{\lambda \in I}\norm{au_{\lambda}-a}=0$. We say that it is commutative if $u_{\lambda}u_{\mu}=u_{\mu}u_{\lambda}$. The approximate unit is said to be sequential if the index set is the natural numbers with the usual ordering. 
\end{definition}
As we are working over operator algebras and operator spaces, we slightly need to expand the notion of a representation so we may analyze $\A$ via. its representation theory as for \Cstar-algebras. 
\begin{definition}
	A completely bounded (cb) representation of an operator algebra $\A$ is a completely bounded homomorphism $\pi:\A\to B(H)$. An representation is non-degenerate if $\pi(A)H$ is dense in $H$. 
\end{definition}
Bounded approximate units of an operator algebra converge strongly to idempotents under completely bounded representations. 
\begin{lemma}
	Let $\pi:A\to B(H)$ be a completely bounded representation. Then
	\begin{align*}
		H=\overline{\pi(\A)H}\osum (\pi(\A)H)^\perp=\overline{\pi(A)^*H}\osum (\pi(\A)^* H)^\perp
	\end{align*}
	Defining $\Nil(\pi(\A))=\{h\in H: \pi(a)h=0, \quad \forall a\in A\}$, it satisfies $\Nil(\pi(\A))=\overline{(\pi(\A)^*)}^{\perp}$. 
\end{lemma}
\begin{proof}
	Let $a\in \A$ and $\xi,\eta\in H$, then the decomposition of $H$ follows from the identity: $\ip{\pi(a)\xi}{\eta}=\ip{\xi}{\pi(a)^*\eta}$. 
	For the identity on $\Nil(\pi(A))$, let $h\in \Nil(\pi(\A)),v\in H$ and $a\in \A$. Then 
	\begin{align*}
		\ip{h}{\pi(a)^*v}=\ip{\pi(a)h}{v}=0
	\end{align*}
	So $\Nil(\pi(\A))\subset \overline{\pi(A)^*H}^\perp$. For the converse, let $h\in  \overline{\pi(A)^*H}^\perp$, $v\in H$, and $a\in \A$. Then 
	\begin{align*}
		\ip{\pi(a)h}{v}=\ip{h}{\pi(a)^*v}=0
	\end{align*}
	hence $\pi(a)h=0$, and $h\in \Nil(\pi(\A))$.  
\end{proof}
With this minor remark out of the way, we can show the claimed strong convergence. 
\begin{proposition}\label{mesrennie17}
	Let $\A$ be an operator algebra with bounded approximate unit $(u_{\lambda})_{\lambda \in I}$ and a completely bounded representation $\pi:\A \to B(H)$. Then $\pi(u_\lambda)$ converges in the strong operator topology to an idempotent $q\in B(H)$ satisfying the following 
	\begin{enumerate}
		\item
			$q\pi(a)=\pi(a)q=\pi(a)$
		\item
			$qH=\overline{\pi(\A)H}$. 
		\item
			$(1-q)H=\Nil(\pi(\A))$. 
		\item
			$\norm{q}\leq \norm{\pi}\sup_{\lambda \in I}\norm{u_\lambda}$
	\end{enumerate}
\end{proposition}
\begin{proof}
	Define the projection $q:H\to \overline{\pi(\A)H}$ and the projection $p_*:H\to \overline{\pi(\A)^*H}$. We may define the self-adjoint operator $t=p+(1-p_*)$. We wish to show that this is injective with dense image. To see that it is injective, consider $px=(p_*-1)x$, which implies that $px\in \overline{\pi(\A)H}\cap \overline{\pi(\A)H}^\perp=\{0\}$. Thus $px=0$, and thereby we get we get that $x=(1-p)x=p_*x$, thereby that $x\in \Nil(\pi(\A)^*)\cap \overline{\pi(\A)^*H}=\{0\}$. By self-adjointness we have that $\overline{((p+(1-p_*)))H}=\ker(p+(1-p_*))^\perp=H$, giving that it has dense image. 
	
	Thus we have that $\overline{\pi(\A)H}+\Nil\pi(\A)$ is dense. Pick an arbitrary $\xi\in H$ and $\epsilon>0$. Then we have $\eta_0\in \overline{\pi(\A)H}$ and $\eta_1\in \Nil\pi(\A)$ such that $\norm{\xi-(\eta_0+\eta_1)}\leq \epsilon/4C$ for $C=\sup_{\lambda \in I}\norm{\pi(u_{\lambda})}$. Pick $\lambda\in I$ such that for all $\mu>\lambda$ we have that $\norm{\pi(u_\lambda-u_\mu)\eta_0}<\epsilon/2$. Then we may perform the following estimate to show that $\pi(u_\lambda)$ is strongly Cauchy. 
	\begin{align*}
		&\norm{\pi(u_{\lambda}-u_\mu)\xi} \\
		&\leq \norm{\pi(u_\lambda-u_\mu)(\eta_0+\eta_1)}+\norm{\pi(u_\lambda-u_\mu)(\xi-(\eta_0+\eta_1))} \\
		&\leq \norm{\pi(u_\lambda-u_\mu)\eta_0}+\norm{\pi(u_\lambda-u_\mu)}\norm{\xi-(\eta_0+\eta_1)}\leq \frac{\epsilon}{2}+\frac{\epsilon}{2}
	\end{align*}
	Therefore we may define the limit $q$ of $\pi(u_\lambda)$ as the strong topology is complete on bounded sets. We see that $q$ commutes with the representation, and that $q$ is an idempotent from the construction. As it is an idempotent it has closed range. Thus $\im(q)=\overline{\pi(\A)H}$. If we consider $(1-q)$, we see that $(1-q)\pi(a)=0$, thus $\im(1-q)=\Nil(\pi(\A))$.
	To see that the norm of $q$ is bounded as claimed, consider the following 
	\begin{align*}
		\norm{q}=\sup_{h,\norm{h}=1}\norm{\lim_{\lambda}\pi(u_\lambda)h}\leq \norm{\pi}\sup\norm{u_\lambda}. 
	\end{align*}
\end{proof}
If we wish to drop the assumption of our approximate unit being bounded, it is necessary to find a different method of proof. Fundamentally, we are only working on non-commutative analogues of complete manifolds, as expanded upon in \cite[Section 2]{mesrennie} and in order to generalize to non-complete manifolds with symmetric operators, we need an entirely new framework as introduced in for example \cite{kaad}. 
As a geometric corollary of the previous propositions, we get that $H\cong \overline{\pi(\A)H}+\overline{\pi(\A)^*H}$. 
The following lemma for Kasparov modules shows the fundamental relationship between $\D$ being self-adjoint and there being a bounded approximate unit for $\A$. 
\begin{theorem}\label{mesrennie19}
	Let $(\A,E_B,\D)$ be a Kasparov module, where we only require that $\D$ is symmetric, such that $\overline{\pi(\A)E_B}$ is a complemented sub-module and $p$ the corresponding projection. Assume that $(\D u_ne)$ converges for all $e\in \dom(\D^*)$ and $p\in \lip(\D^*)$ and that the operator $[\D^*,p]$ is the strict limit of the sequence $([\D^*,u_n])_{n\in \N}$. 
	\begin{enumerate}
	\item
		Then $p[\D^*,u_n]p\to 0$ in the strict topology,  
	\item
		$p$ is the strict limit of $u_n$.  
	\item
		If $\A E_B$ is dense in $E_B$, $\D$ is self-adjoint. 
	\end{enumerate}
\end{theorem}
\begin{proof}
	By assumption, $\overline{\pi(A)E_B}$ is complemented so we may consider the projection $p$ onto this submodule. Let $e\in \overline{\pi(A)E_B}$, then $u_ne\to e$ as for $ae\in \pi(A)E_B$ we have $ae-u_nae=(a-u_na)e\to 0$, so by continuity we have the same on all of $\overline{\pi(A)E_B}$

	We have $pa=ap=a$, so $(1-p)a=a(1-p)=0$, implying 
	\begin{align*}
		\lim_{n\to \infty} u_n e=\lim_{n\in \N} u_n pe+u_n(1-p)e=\lim_{n\in \N} u_n pe=pe
	\end{align*}
	giving first part of the statement. As $(u_n)_{n\in \N}$ is a bounded approximate for $\A$, the sequence of operators $[\D^*,u_n]$ is uniformly bounded. Letting $a\in \A$ and $e\in \dom(\D)^*$ we get the identity 
	\begin{align*}
		[\D^*,u_n]ae=[\D^*,u_na]e-u_n[\D^*,a]e
	\end{align*}
	As $ae=pae=ape$, multiplying from the left by $p$ we get
	\begin{align*}
		p[\D^*,u_n]pae=p[\D^*,u_na]pe-pu_n[\D^*,a]pe
	\end{align*}
	This will converge to zero, as each term on the right hand side converges to $p[\D,a]pe$. As $\pi$ is assumed to be essential on $pE_B$, linear combinations of terms of the form $ae$ are dense in $pE_B$, it follows that $p[\D,u_n]p$ converges pointwise to zero. We have assumed that we have a symmetric Kasparov module, along with $(u_n)_{n\in \N}$ being even, so we may infer that $(p[\D^*,u_n]p)^*=-p[\D^*,u_n^*]p$. Thereby $(u_n^*)_{n\in \N}$ also becomes a bounded approximate unit for $\A$. Replacing $u_n$ with $u_n^*$ throughou the previous argument then gives that $p[\D,u_n^*]p$ also converges strictly to zero. 
	To show the second point we first show that $p$ maps $\dom(\D)^*$ to $\dom(\D)$ and then use this to show that $[\D^*,u_n]$ converges to $[\D^*,p]$ on $\dom(\D)^*$. 
	
	We have assumed that the following sequence is convergent
	\begin{align*}
		\pi_\D(u_n)\begin{pmatrix} e \\ \D^* e \end{pmatrix}=\begin{pmatrix} u_n e\\  \D u_n e\end{pmatrix}
	\end{align*}
	By the first part of the theorem $p$ is the strict limit of $u_n$, so 
	\begin{align*}
		\lim_{n\to \infty} u_n\begin{pmatrix} e \\ \D^* e \end{pmatrix}=\begin{pmatrix} pe\\  x \end{pmatrix}
	\end{align*}
	By closedness of $\D$, $pe\in \dom(\D)$ and $x=\lim_{n\in \N} \D u_n e=\D pe$. This shows that $p\dom(\D)^*\subset \dom(\D)$, so for every $e \in \dom(\D)^*$ we get the identities 
	\begin{align*}
		[\D^*,u_n]e&=[\D^*,u_n]pe+[\D^*,u_n](1-p)e \\
		&=\D u_npe-u_n \D pe+\D u_n(1-p)e-u_n \D^* (1-p)e \\
		&=\D u_n e-u_n \D pe -u_n \D^*(1-p)e 
	\end{align*}
	which converges to $\D pe-p\D pe -p \D^* (1-p)e=[\D^*,p]e$. As $[\D^*,u_n]$ is bounded, the convergence is strict on all of $E_B$ and thereby the operator $[\D^*,p]$ is bounded on $\dom(\D)^*$. 
	
	To show the final part of the theorem, note that $[\D^*,u_n]\to 0$ since in this case $p=1$. For every $e\in \dom(\D)^*$ we get $\D^*u_n e=[\D^*,u_n]e+u_n\D^*e\to \D^*e$. As $u_ne\to e$ and $u_n e\in \dom(\D)$, it follows that $e$ lies in the graph-norm completion of $\D$. Since $e$ was arbitrary in $\dom(\D)^*$, it follows that $\dom(\D)^*\subset \dom(\D)$, giving self-adjointness. 
	
\end{proof}
We gather up a plethora of results illustrating the geometric control imposed by the presence of a bounded approximate unit in the following theorems. 
\begin{theorem}\label{mesrennie112}
	Let $\A$ be an operator algebra with a bounded approximate unit and an essential cb. representation $\pi:\A\to B(H)$. Then the norm on $M_n(\A)$ is equivalent to the norm $\norm{\cdot}_{op,n}$, $\norm{a}_{op,n}=\sup_{\norm {b}_n\leq 1}\norm{ab}_n$ where $\norm{\cdot}_n$ is the matrix norm. 
\end{theorem}
\begin{proof}
	We start by noting that $(u_\lambda)_{\lambda\in \Lambda}$ gives rise to bounded approximate units $1_n \cdot u_\lambda$ on $M_n(\A)$, and as such we may simply run the argument for $u_\lambda$.
	It is clear that $\norm{a}_{op,n}\leq \norm{a}$. Letting $u$ be a bounded approximate unit, we clearly have $\frac{1}{c}\norm{u_{\lambda}}\leq 1$ for some fixed $c$. For all $\epsilon>0$ there exists a $\lambda$ such that $\norm{b-bu_\lambda}<\epsilon$. This gives us the following string of inequalities
	\begin{align*}
		\frac{1}{c}(\norm{b}_{op,n}-\epsilon)<\frac{1}{c}\norm{b}-\norm{b-bu_\lambda}\leq \frac{1}{c} \norm{bu_\lambda} \leq \norm{b}_{op,n}
	\end{align*}
	Showing the desired. 
\end{proof}
\begin{definition}
	Let $T:\A\to \A$ where $\A$ is an operator algebra. Define $\norm{T}_{op}=\sup_{n \in \N} \norm{T\tens 1_n}_{op,n}$. 
\end{definition}
An operator $T:\A\to \A$ is cb. if and only $\norm{T}_{op}<\infty$. Thus, by \Cref{mesrennie112} we can use $\norm{\cdot}_{op}$ to define the completely bounded version of the strict topology on an operator algebra: 
\begin{definition}
	Let $\A$ be an operator algebra with a bounded approximate unit. We define the multiplier algebra of $\A$ as the strict closure of $\A$, ie. by
	\begin{align*}
		M(\A)=\{T:\A\to \A\mid \exists (b_\lambda)_{\lambda \in \Lambda}\subset \A, \lim_{\lambda \in \Lambda} \norm{b_\lambda a-Ta}_{op}=\lim_{\lambda \in \Lambda}\norm{ab_\lambda-T a}_{op}=0, \forall a \in A\}
	\end{align*}
	Where we define $\norm{T}=\norm{T}_{op}$. 
\end{definition}
\begin{theorem}\label{mesrennie114}
\begin{enumerate}
	\item Let $\A$ be an operator algebra with a bounded approximate unit and an essential cb. representation $\pi:\A\to B(H)$. The cb. representation extends to a representation of the multiplier algebra of $\A$, such that $\pi(1)=1$. 
	\item If we further assume that $\pi$ is a cb-isomorphic representation, then as for \Cstar-algebras we get the strict closure of $\A$, ie. the multiplier algebra of $\A$ is cb-isomorphic to the idealiser of $\pi(\A)$. Furthermore, every element in $M(\A)$ is the strict limit of a bounded net in $\A$. Finally, closed ideals of $\A$ descend to closed ideals of $M(\A)$. 
	\item Assume that $\pi:\A\to B(H)$ is a cb. representation of an operator with bounded approximate unit. Then $\pi$ extends to a representation $M(\A)\to B(H)$ such that $\pi(1)$ is an idempotent and $\overline{\pi(\A)H}=\pi(1)H$ and $(1-\pi(1))H=\Nil(\pi(\A))$. 
	\end{enumerate}
\end{theorem}
\begin{proof}
	\begin{enumerate}
	\item
		We have assumed that $H=\overline{\pi(\A)H}$, so for all $h\in H$ we have that $u_\lambda h$ converges to $h$. Picking $b\in M(\A)$ and utilizing that $M(\A)$ is the strict closure of $\A$, we get $\sup_{\lambda}\norm{bu_\lambda}<\infty$ and that $(bu_\lambda a)_{\lambda\in \Lambda}$ is norm-Cauchy in $\A$ for all $a\in \A$. Defining 
		\begin{align*}
			\pi(b)\pi(a)h=\lim_{\lambda \in \Lambda} \pi(bu_\lambda a)h
		\end{align*}
		we see that we get a Cauchy net. Therefore $\pi(bu_\lambda)$ converges for every $h\in \pi(\A)H$. As this is a dense subspace and the net $\pi(bu_\lambda )$ is uniformly bounded, we may infer that the net is strongly Cauchy on $H$. Thus $h\mapsto \lim_{\lambda \in \Lambda} \pi(bu_\lambda)h$ gives us a bounded operator on $H$. 
		By definition we have $\pi(ab)=\pi(a)\pi(b)$ for $a\in \A,b\in M(\A)$. Letting $a,b\in M(\A)$ we have the following
		\begin{align*}
			\pi(a)\pi(b)h=\pi(a)\lim_{\lambda \in \Lambda}\pi(bu_\lambda)h=\lim_{\lambda\in \Lambda} \pi(abu_\lambda)h
		\end{align*}
		Where we have used that $bu_\lambda \in \A$, so that the extension defines a homomorphism. 
		To show uniqueness of this extension, we remark that for all $a\in \A$ and $b\in M(\A)$ we have that $bu_\lambda a\to ba$ in $\A$, it follows by essentiality that this extension is unique. 
	\item
		By what we have just shown, $\pi$ extends to representation of $M(\A)$. Let $T\in \pi(M(\A))$, and let $(b_\lambda)_{\lambda \in I}\subset \A$ be a net satisfying 
		\begin{align*}
			\lim_{\lambda \in I}\norm{b_\lambda a-Ta}=\lim_{\lambda \in I}\norm{ab-aT}=0
		\end{align*}
		where we have suppressed $\pi$. It follows that $T\pi(a)\in \pi(\A),\pi(a)T\in \pi(\A)$. Thus $T\in \pi(M(\A))$ idealizes $\pi(\A)$. For the other inclusion, let $T\in B(H)$ such that $T\pi(\A)\subset \A,\pi(\A)T\subset \A$. Considering the net $T\pi(u_\lambda)$, we get 
		\begin{align*}
			&\norm{T(\pi(u_\lambda a))-T\pi(a)}\leq \norm{T}\norm{\pi(u_\lambda a-a)}\to 0 \\
			&\norm{\pi(a)T\pi(u_\lambda)-\pi(a)T}\to 0
		\end{align*}
		As $\pi$ is assumed to be a cb. isomorphism and essential, it follows that $T=\pi(b)$ for some $b\in M(\A)$. To see that $T$ is the strict limit of a bounded net, we simply apply \Cref{mesrennie112}. To see that if $J\subset \A$ is a closed ideal in $\A$, it will also be a closed ideal in $M(\A)$, consider $T\in  M(\A)$ and $b\in J$. The net $u_\lambda Tb$ will converge to $Tb$ in norm, however $u_\lambda T\in \A$ so the net actually lies in $J$. As $J$ is closed $Tb\in J$ and likewise for $bT$. 
	\item/
		We have the cb. isomorphism $H\cong qH\osum (1-q)H$ where $q$ is given as in \Cref{mesrennie17}. Since $\pi$ is essential on $qH$ and $0$ on $(1-q)H$, by the first part of the theorem we get a representation $\pi:M(\A)\to B(qH)$ which is zero $(1-q)H$, thereby giving the desired representation. By construction, $\pi(1)=q$, showing the desired.
	\end{enumerate}
\end{proof}
In case $\A$ is a differentiable algebra, we can characterize the multiplier algebra of $\A$ in a more concrete fashion. The idea is that if we think of $\A$ as $C_0^1(M)$ on a complete manifold, then $M(\A)=C_b^1(M)$. 
\begin{proposition}\label{mesrennie117}
	Let $\D:\dom(\D)\to E_B$ be self-adjoint and regular and $\A\subset \lip(\D)$ be closed with bounded approximate unit, i.e. a differentiable algebra with bounded approximate unit. If the representation of $A$ is essential, then the multiplier algebra $M(\A)$ is cb-isomorphic to:
	\begin{align*}
		\tilde{M}(\A)=\{T\in M(A) : T\dom(\D)\subset \dom(\D),T\A,\A T\subset \A,[\D,T]\in L(E_B)\}
	\end{align*}
	Where the isomorphism preserves spectra.
\end{proposition}
\begin{proof}
	The algebra $\tilde{M}$ is clearly a subalgebra of $M(A)$ and included in the idealizer of $\pi_\D(\A)$ in $L_B(E\osum E)$. To see the other inclusion, start by considering
	\begin{align*}
		T\pi_\D(a)&=\begin{pmatrix} T_{11} & T_{12} \\ T_{21} & T_{22} \end{pmatrix} \begin{pmatrix}a & 0 \\ [\D,a] & \gamma(a) \end{pmatrix}
		=\begin{pmatrix} T_{11}a+T_{12}[\D,a] & T_{12}\gamma(a) \\ T_{21}a+T_{22}[\D,a] & T_{22}\gamma(a) \end{pmatrix}
	\end{align*}
	For this to be an element of $\pi_{\D}(\A)$, we must have $T_{12}\gamma(a)=0$, so $T_{12}=0$ by essentiality of $A$. This implies that $\gamma(T_{11}a)=T_{22}\gamma(a)$, so $\gamma(T_{11})=T_{22}$ by essentiality. 
	As $\A$ is essential, we see that $T_{11}$ must preserve the domain of $\D$. We can infer that $T_{21}$ must satisfy the equation \begin{align*} T_{21}a+\gamma(T)[\D,a]=[\D,T_{11}a]\end{align*} By essentiality we again infer that $T_{21}=[\D,T_{11}]$, and as such is bounded. This shows that the algebra $\tilde{M}(\A)$ we have defined also contains the idealizer of $\pi_{\D}(\A)$ as desired, giving the desired equality. 
	\newline
	For the second part of the theorem, note that $\overline{AE_B}=E_B$ so $\pi_\D(1)=1$ through \Cref{mesrennie114}. Recycling the arguments of the first and second parts of \Cref{mesrennie114} then gives that $M(\A)$ maps to $L_B(E_B)$ and that the idealizer of $\pi_\D(\A)$ equals $\pi_\D(M(\A))$. By our equivalent characterizations of the norm on $M(\A)$ in \Cref{mesrennie112} we get the desired cb-isomorphism. 
	The spectral invariance follows from a result of Mesland, given in the complex case by the equivalence of complex and real spectra it follows through in the real case, \cite[Theorem B.3]{mesland}. 
\end{proof}
\begin{definition}
	Let $\D:\dom(\D)\to E_B$ be a self-adjoint and regular, and $\A\subset \lip (\D)$ be a differentiable algebra with bounded approximate unit, with $\A$ a dense subset of the \Cstar-algebra $A$. If the representation of $A$ is essential we define the unitization of $\A$, $\A^+ \subset M(\A)$ as the algebra generated by $\A$ and $\pi_\D(1)$.
\end{definition}
One can show in that the unitization exists in general, \cite{meyer}.  
\begin{assumption}
	From now on we shall assume that all representations appearing in Kasparov modules are essential. 
\end{assumption}
\begin{remark}
	That the representations featured in the Kasparov modules is essential is not unreasonable, as one may show that any Kasparov module is equivalent to one where the representation is essential,\cite{kasparov}.
\end{remark}
\begin{definition}
	Let $\A$ be a differentiable algebra with a dense right ideal $\dom c$. A linear operator $c:\dom c\to \A$ is an unbounded multiplier if $c(ab)=(ca)b$ for $b,c\in \dom c$. If in addition $c$ satisfies:
	\begin{enumerate}
	\item
		$c$ is closed.
	\item
		It is formally symmetric in the inner product $\ip{a}{b}=a^*b$, i.e. $(ca)^*b=a^*(cb)$ 
	\item
		The operators $c^2+1$ is surjective and $(c^2+1)^{-1}$ is a bounded multiplier for $\A$. 
	\end{enumerate}
	$c$ is an unbounded self-adjoint multiplier.  
	Further, we say that $c$ is positive if $(ca)^*a\geq 0$ in $A$. 
\end{definition}

%\todo{We need to show first that $c^2$ is well-defined}
\begin{lemma}
	Let $c$ be an unbounded multiplier, then the operator $(c^2+\lambda)$ is invertible for all positive $\lambda$, likewise if $c$ is positive then $c+\lambda$ is invertible. 
\end{lemma}
\begin{proof}
	The operator $(c^2+1)^{-1}$ is bounded, and thus the operator $(c^2+\lambda)(c^2+1)^{-1}$ is invertible in $M(A)$, giving that $c^2+\lambda$ is bijective on $A$. Thus we may conclude by \Cref{mesrennie117} that $(c^2+\lambda)(c^2+1)^{-1}$ is invertible in $M(\A)$, giving that $c+\lambda:\dom c\to \A$ is bijective in $\A$ as well. We see that the positive case is proved in an entirely analogous manner. 
\end{proof}
It turns out the existence of a bounded approximate unit is enough to ensure a symmetric multiplier is also closed. 
\begin{lemma}
	For a differentiable algebra with a bounded approximate unit it is sufficient for a multiplier $c:\dom c \to \A$ to satisfy that $(ca)^*b=a^*(cb)$ for every $a,b\in \dom c$ for it to be closable. 
\end{lemma}
\begin{proof}
	As $\A$ has a bounded approximate unit, the norm on $\A$ can be equivalently characterized as $\norm{a}=\norm{a}_{op}$. Letting $a_n$ be a sequence converging to zero in $\dom c$, and $ca_n\to b$. By symmetry of $c$ and the identity $\norm{(ca_n)}=\norm{(ca_n)^*}$, we have 
	\begin{align*}
		b^*a=\lim_{n\to \infty} (ca_n)^* a=\lim_{n\to \infty} a^*_n(ca)=0
	\end{align*}
	Thus $\norm{b^*}_{op}=0$, implying $\norm{b^*}=0$ as desired. 
\end{proof}
It follows that if $(c^2+1)^{-1}$ is densely defined and bounded, then then the closure of $c$ is a self-adjoint unbounded multiplier with domain $(c^2+1)^{-1}\A$. We define the notion of a complete multiplier as follows 
\begin{definition}
	Let $(\A,E_B,\D)$ be an unbounded Kasparov module and $c$ self-adjoint multiplier of $\A$. Then we shall say that $c$ is complete if 
\begin{enumerate}
\item 
	$	(c^2+1)^{-1}\in \A$ 
\item
		$\im((\D^2+1)^{-1/2}(c^2+1)^{-1/2})=\im((c^2+1)^{-1/2}(\D^2+1)^{-1/2})$
\item 
		The operator $[\D,c]$ is bounded on $\im((c^2+1)^{-1/2}(\D^2+1)^{-1})$. 
\end{enumerate}
\end{definition}
These sets are natural to consider when working with unbounded operators, as they are the natural domains for $c\D$ and $\D c$.
We can now prove the essential theorem to expanding the expanded technical theorem of \cite{mesrennie} to the real setting, relating approximate units, positive self-adjoint multipliers and and the differential operator $\D$. This is one of the results where the proof requires the most modification to fit into the real setting. 
\begin{theorem}
	Let $\D:\dom(\D)\to E_B$ be a self-adjoint and regular operator, and $\A\subset \lip(\D)$ such that $AE_B=E_B$. Then the following are equivalent
	\begin{enumerate}
	\item
		There is an approximate commutative unit $(u_n)_{n\in \N}$ for $\A$ such that $\norm{[u_n,\D]}\to 0$ 
	\item
		There exists a positive self-adjoint multiplier $c$ for $\A$. 
	\item
		We have a strictly positive element $h\in \A$ such that $\im((\D^2+1)^{-1/2}h=\im(h(\D^2+1)^{-1/2}$, along with a constant $c$ such that $i[\D,h]\leq ch^2$ in $A\tens \C$.
	\end{enumerate}
\end{theorem}
\begin{proof} 
\begin{itemize}
	\item[$1\Rightarrow 2$]
		Let $\epsilon<1$. Pick a countable subset with dense linear span $(a_i)_{i\in \N}\subset \A$, without loss of generality we may assume that $\norm{(u_{n+1}-u_n)a_i}\leq \epsilon^{2n}$,as well as $\norm{[\D,u_n]}\leq \epsilon^{2n}$. Define $d_n=u_{n+1}-u_n$ and define the candidate self-adjoint multiplier 
		\begin{align*}
			c=\sum_{n=1}^\infty \epsilon^{-n}d_n
		\end{align*}
		This is densely defined, as may be readily verified for any fixed $a_i$, with $i<k<l$, and $c_j=\sum_{n=1}^j \epsilon^{-n}d_n$. The estimates: 
		\begin{align*}
			\norm{\sum_{n=k}^l \epsilon^{-n}d_na_i}&\leq \sum_{n=k}^l \epsilon^{-n}\norm{(u_{n+1}-u_n)a_i} \\
			&\leq \sum_{n=k}^l \epsilon^n
		\end{align*}
		show that $c_k a_i$ is a Cauchy sequence, so $ca_i\in \A$. The operator $c$ is clearly symmetric, so it suffices for us to show that $(c^2+1)^{-1/2}$ is densely defined and bounded. Consider the increasing sequence 
		\begin{align*}
			\norm{[\D,c_k]}&=\norm{\sum_{n=1}^k \epsilon^{-n}([\D,u_{n+1}]-[\D,u_n]}\leq \sum_{n=1}^k \epsilon^{-n} \pa{\norm{[\D,u_n]}+\norm{[\D,u_n]}} \\
			&\leq 2\sum_{n=1}^k \epsilon^{n}
		\end{align*}
		from which it follows that $\norm{[\D,c]}=\sup_k\norm{[\D,c_k]}<\infty$. We can now proceed with our calculations through the holomorphic functional calculus
		\begin{align*}
			&\sup_k \norm{[\D,(c_k^2+1)^{-1/2}}=\sup_{k}\norm{\int_{\Re{z}=1/2} (z^2+1)^{-1/2} [\D,(z-c_k)^{-1}] dz} \\ 
			&\sup_{k}\norm{\int_{\Re{z}=1/2} (z^2+1)^{-1/2} (z-c_k)^{-1} [\D,c_k](z-c_k)^{-1} dz} \\
			&\leq \sup_{k}\norm{\int_{\Re{z}=1/2} (z^2+1)^{-1/2} \norm{(z-c_k)^{-1}}^2 dz} \norm{[\D,c_k]} \\
			&\leq \sup_{k}\norm{\int_{\Re{z}=1/2} (z^2+1)^{-1/2} \norm{(z-c_1)^{-1}}^2 dz} \norm{[\D,c]}<\infty
		\end{align*}
		This shows that the sequence $(c_k^2+1)^{-1/2}$ is  bounded, so we need only check that it is strictly Cauchy for the limit to exist. 
		\begin{align*}
			&\norm{((c_l^2+1)^{-1/2}-(c_m^2+1)^{-1/2})a_i} \\
			&\leq \norm{\pa{1+\sum_{n=1}^l \epsilon^{2n}d^2_n}^{-1/2}-\pa{1+\sum_{n=1}^l \epsilon^{2n}d^2_n+\sum_{k=l+1}^m\epsilon^{2k}d^2_k}^{-1/2}}\norm{a_i}
		\end{align*}
		Picking $l$ sufficently large, we see that this expression may be bounded by $\epsilon$ independent of $m$, giving that the sequence is strictly Cauchy. Thus we have the existence of $(c^2+1)^{-1/2}\in M(\A)$, and we have shown that $c$ is a positive self-adjoint unbounded multiplier. We need to check the remainder of the criteria of $c$ being a complete multiplier. At first we verify the equality 
		\begin{align*}
			\im((c_k^2+1)^{-1/2}(\D^2+1)^{-1/2})=\im((\D^2+1)^{-1/2}(c_k^2+1)^{-1/2})
		\end{align*}
		As $(c^2+1)^{-1/2}$ is the strict limit of $(c_k^2+1)^{-1/2}$, we may write 
		\begin{align*}
			y=\lim_k(c_k^2+1)^{-1/2}(\D^2+1)^{-1/2}x
		\end{align*}
		for every element $y$ in the image  of $(c^2+1)^{-1/2}$.
		We may then write up the following identity
		\begin{align*}
			&(c_k^2+1)^{-1/2}(\D^2+1)^{-1/2}x \\
			&=-(\D^2+1)^{-1/2}(c_k^2+1)^{-1/2} \\
			&+(\D^2+1)^{-1/2}(c_k^2+1)^{-1/2}[(\D^2+1)^{1/2},(c_k^2+1)^{1/2}](c_k^2+1)^{-1/2}(\D^2+1)^{-1/2}
		\end{align*}
		We now need to show that the sequence $[(\D^2+1)^{1/2},(c_k^2+1)^{1/2}]$ is uniformly bounded in operator norm. To do this, consider the following calculations
		\begin{align*}
			&[(\D^2+1)^{1/2},(c_k^2+1)^{1/2}] \\
			&=\int_{z=1/2+it}(z^2+1)^{1/2}[(\D^2+1)^{1/2},(c_k-z)^{-1}] dz\\
			&=\int_{z=1/2+it}(z^2+1)^{1/2}(c_k-z)^{-1}[(\D^2+1)^{1/2},c_k](c_k-z)^{-1} dz \\
			&=\int_{w=i(t^2+1)+t}\int_{z=1/2+it}(z^2+1)^{1/2}(w^2+1)^{1/2}(c_k-z)^{-1}(\D-w)^{-1}[\D,c_k](\D-w)^{-1}(c_k-z)^{-1} dzdw 
		\end{align*}
		This sequence is clearly uniformly bounded in operator norm, so we may perform the limiting procedure for $c_k$ as follows
		\begin{align*}
			&\lim_k(c_k^2+1)^{-1/2}[\D,c_k](c_k^2+1)^{-1/2}(\D^2+1)^{-1/2}x \\
			&=(c^2+1)^{-1/2}[\D,c](c^2+1)^{-1/2}(\D^2+1)^{-1/2}x
		\end{align*}
		thereby showing the desired. The other inclusion is entirely analogous. 
		To see that $[\D,c]$ is bounded on $\im((c^2+1)^{-1}(\D^2+1)^{-1})$ we simply remark that it is the strong limit of the uniformly bounded family of operators $([c_k,\D])_{k\in \N}$.
		Finally, we need to show that $(c^2+1)^{-1/2}\in \A$. We start by showing that it is an element of $A$, then proceeding to show that it actually lies in $\A$. To see this is the case, consider the commutative subalgebra $B=C_0(X)$ of $A$ generated by $(u_n)_{n\in \N}$. We utilize that every unbounded multiplier is specified by its Gelfand transform, see eg. \cite[Theorem 2.1,Theorem 2.3]{wood}.%, which can be seen to be true as the result holds in the Real case and thus also in the real case.
		Fixing $0<t<1$ define the family of sets $X_n=\{x\in X| u_n(x)\geq t\}$. Let $x\in X,k\in \N$ and consider $m\geq k$. Then we may calculate as follows for $x\in X_k$
		\begin{align*}
			\sum_{n=0}^\infty \epsilon^{-n}d_n(x)&\geq \sum_{n=k}^\infty \epsilon^{-n}d_n(x)  \\
			&=\sum_{n=k}^m \epsilon^{-n}d_n(x)+\sum_{j=m+1}^\infty \epsilon^{-j}d_j(x) \\ 
			&\geq \sum_{n=k}^\infty \epsilon^{-k}d_n(x) + \sum_{j=m+1}^\infty \epsilon^{-j}d_j(x)\\
			&=\epsilon^{-k}(u_{m+1}-u_k)(x)+\sum_{j=m+1}^\infty \epsilon^{-j}d_j(x) \\ 
			&\geq \epsilon^{-k}(u_{m+1}-t)(x)+\sum_{j=m+1}^\infty \epsilon^{-j}d_j(x) 
		\end{align*}
		As $u_n$ is an approximate unit and $\sum_{j=m+1}^\infty \epsilon^{-j}d_j(x) $ converges pointwise to zero, we get the estimate 
		\begin{align*}
			\sum_{n\in \N} \epsilon^{-n} d_n(x)\geq (1-t)\epsilon^{-k}
		\end{align*}
		To see that $(c^2+1)^{-1/2}$ actually lies in $\A$, note that $u_n(c^2+1)^{-1/2}$ converges to $(c^2+1)^{-1/2}$ in $A$-norm. We have that $[\D,u_n]$ is bounded and $[\D,u_n]$ goes to zero, so we may derive:
		\begin{align*}
			[\D,u_n(c^2+1)^{-1/2}]&=u_n[\D,(c^2+1)^{-1/2}]+[\D,u_n](c^2+1)^{-1/2} \\
			&=\int_{z=(/2+ti} u_n (z^2+1)^{-1/2} (c+z)^{-1}[c,\D](c+z)^{-1} dz -[\D,u_n](c^2+1)^{-1/2} \\
			&=u_n\int_{z=1/2+ti} (z^2+1)^{-1/2} (c+z)^{-1}[c,\D](c+z)^{-1} dz -[\D,u_n](c^2+1)^{-1/2} 
		\end{align*}
		which converges to $[\D,(c^2+1)^{-1/2}]$ as desired. As such $\pi_\D(u_n(c^2+1)^{-1/2})$ converges to $\pi_\D((c^2+1)^{-1/2})$. As all $u_n(c^2+1)^{-1/2}$ lie in $\A$, we may infer that $(c^2+1)^{-1/2}\in \A$. 
	\item[$2\Rightarrow 1$]
		To see the claim, consider $f_n(x)=\exp(-x/n)$. Then we may write 
		\begin{align*}
			[\D,f_n(c)]y&=\int_0^1 \part_s (\exp(-c(1-s)/n)\D \exp(-cs/n)y)ds \\
			&=-\frac{1}{n}\int_0^1 \exp(-c(1-s)/n)[\D,c]\exp(-cs/n)yds 
		\end{align*}
		As everything is bounded, we may extend the equality to the entirety of $E_B$, as well as getting the inequality $\norm{[\D,f_n(c)]}\leq \frac{1}{n}\norm{[\D,c]}$
		To see that we actually have an approximate unit, note that $(c^2+1)^{-1/2}\A$ is dense, thereby $(c^2+1)^{-1/2}$ generates an essential ideal wherein $f_n(c)$ is clearly an approximate unit. 
	\item[$2\Lr 3$]
		Pick an unbounded multiplier $c$ on $\A$ and remark that $h=(1+c^2)^{-1/2}$ has dense range and is positive. If $h\in \A$ is positive with dense range, define the operator $c=h^{-1}$, which is densely defined on $\im(h)$. We can infer the domain relation from the identity
		\begin{align*}
			(h^{-2}+1)^{-1/2}=h(1+h^2)^{-1/2}
		\end{align*}
		and $1+h^2$ is invertible, so $(1+h^2)^{-1/2}$ is a bijection on $\dom(\D)=\im( (\D^2+1)^{-1/2})$. Then we have the equalities
		\begin{align*}
			&\im(h(h^2+1)^{-1/2}(\D^2+1)^{-1/2})=\im (h(1+\D^2)^{-1/2}) \\
			&=\im ((\D^2+1)^{-1/2}) h=\im ((\D^2+1)^{-1/2}h(1+h^2)^{-1/2})
		\end{align*}
		Finally, from the assumption that $i[\D,h]\leq ch^2$ for some $c\in \R^+$, it can be inferred for $e\in \im(h(\D^2+1)^{-1/2}h)$ that
		\begin{align*}
			&\ip{i[\D,h^{-1}]e}{e}_B=-i\ip{h^{-1}i[\D,h]h^{-1}e}{e}_B \\
			&=\ip{i[\D,h]h^{-1}e}{h^{-1}e}_B\leq C\ip{he}{h^{-1}e}_B=C\ip{e}{e}_B
		\end{align*}
		Then by taking a sequence $hy_n\to y\in E_B$ we can infer boundedness on the entirety of $\im(h(\D^2+1)^{-1/2})$
\end{itemize}
\end{proof}
\begin{remark}
This theorem allows us to assume that every approximate unit for a differentiable algebra is even, self-adjoint commutative and increasing.
\end{remark} This completes our study of differential algebras as independent objects, and we will henceforth be considering these in the context of constructing the unbounded Kasparov product.
\subsection{Connections}
In this section we establish the differential framework needed for the construction of the unbounded Kasparov product, the essential ingredient in the construction of the unbounded Kasparov product in \cite{mesland}. The most important part of the framework is notion of a projector operator module. This allows the construction of a connection, serving to make the naive product $1\tens \D_2$ well-defined on the interior tensor product. The modification is much more direct than in the bounded case, as it is based on the idea of the covariant derivative for a vector bundle. As such it is instructive to think of the product $1\tens_{\nabla}\D_2$ exactly as a covariant derivative with $\D_2$ being the standard derivative and $\nabla$ encoding the derivative on the bundle. That is, we think of $E_B\tens_B F_C$ as the tensor product of two vector bundles, with $E_B$ serving the role of the tangent bundle.
\begin{definition}
%Define the set $\Zred=\Z\setminus \{0\}$, and define the graded Hilbert space $\ell^2(\Zred)$ with grading given by $\gamma(e_i)=\sign(e_i)$.
Let $\B$ be an operator algebra with a bounded approximate unit $u_\lambda$. Then as for \Cstar modules, define the standard right $\B$-module $H_{\B}=H\optens \B$.
\end{definition}
\begin{assumption}
	Let $B$ and $C$ be graded $\sigma$-unital \Cstar-algebras.
	For the remainder of this section, we fix an unbounded essential $(B,C)$-module $(\B,F_C,\D)$ where $\B$ is assumed to have a bounded approximate unit, with $\B^+$ denoting the unitization of $\B$.
	We define the corresponding differential representation of $\B$ as 
	\begin{align*}
		\pi_{\D}(b)=\begin{pmatrix} b & 0 \\ [\D,b] & \gamma(b) \end{pmatrix} \in L_C(F_C\osum F_C)
	\end{align*}
\end{assumption}
We see that the grading operator $\Gamma$ on $H_{\B^+}$ can be written as $\Gamma(b_i)_{i\in \Zred}=(\sign(i)\gamma(b_i))_{i\in \Zred}$. Defining the self-adjoint unitary 
\begin{align*}
	&\epsilon:H_{\B^+}\to H_{\B^+} \\
	&\epsilon((b_i)_{i\in \Zred})=(\sign(i)b_i)_{i\in \Zred}
\end{align*}
the grading operator on $H_{\B^+}$ may be given as $\epsilon \diag(\gamma_{\B^+})$, a factorization which we can already see simplifies our constructions as we may represent $H_{\B^+}$ as an operator $\B^+$ module via. the representation 
\begin{align*}
	(b_i)_{i\in \Z\setminus \{0\}}\mapsto &\begin{pmatrix} b_i & 0 \\ \sign(i)[\D,b_i]_{\B^+} & \Gamma(b_i) \end{pmatrix}_{i\in \Zred} \\
	&=\begin{pmatrix} 1 & 0 \\ 0 & \epsilon \end{pmatrix} \begin{pmatrix} b_i & 0 \\ [\D,b_i]_{\B^+} & \gamma(b_i) \end{pmatrix}_{i\in \Zred}
\end{align*}
An immediate question which one must ask is then whether the inner product on $H_{\B^+}$ derived from $H_{B^+}$ is actually $\B^+$-valued. This is true as shown in \cite{mesland}, and we state it here without proof. 
\begin{proposition}
	The standard $B$-valued inner product on $H_{\B}$, ie. 
	\begin{align*}
		\ip{x}{y}=\sum_{i\in \Zred} x_i^* y_i
	\end{align*}
	takes values in $\B^+$.
\end{proposition}
As the class of a Kasparov module is invariant under compact perturbations, we need a good operator-algebraic notion of the compact operators on $H_{\B^+}$. Likewise, we would like an analogue of the adjointable operators on $H_{\B^+}$. 
\begin{definition}
	We define the compacts on $H_{\B^+}$ as  $\K(H_{\B^+})=\K\optens \B^+$, we define $L(H_{\B^+})$ as the subset of completely bounded maps $T:H_{\B^+}\to H_{\B^+}$ that have an adjoint with respect to the standard inner product. 
\end{definition}
Working only in the context of the standard module turns out to be too inflexible in applications, as shown for instance in \cite{suijlekom} where the concept of an unbounded projection was introduced in order to be able to handle a \Cstar-module stemming from the non-commutative Hopf fibration. Recalling the Serre-Swan theorem, one may be tempted to think of projective operator modules as generalized vector bundles, and it is exactly this we shall use as the base point of our intuition.  
\begin{definition}
	Let $\B$ be an operator $^*$-algebra. A projective operator module $\E$ over $\B$ is a $\B$-module which is completely isometrically unitarily isomorphic to $p\dom(p)$ for some even projection on $H_{\B^+}$ with the additional requirement that $(e_i)_{i\in \Zred}\subset \dom(p)$ 
\end{definition}
We can characterize which modules are of this form with the aid of the concept of a frame, the $^*$-module analogue of a frame in a vector bundle ie. a collection of sections giving a basis for every fiber. 
As for vector bundles, our definition of a section is local in the sense that it is a coordinate-wise definition, but as we shall shortly show having a frame has global consequences for a module.
\begin{definition}
	Let $E_B$ be a \Cstar module over $B$. The sequence $(x_i)_{i\in \Zred}\subset E_B$ is a frame for $E_B$ if:
	\begin{enumerate}
	
	\item  $\gamma_{E_B}(x_i)=\sign_i x_i$. 
	\item 
	The sequence of finite rank operators 
	\begin{align*}
		\chi_n=\sum_{1\leq |i|\leq n} \ket{x_i}\bra{x_i}
	\end{align*}
	is an approximate unit for the finite-rank operators, with $\norm{\chi_n}\leq 1$. $(x_i)_{i\in \Zred}$  
	\end{enumerate}
	We shall refer to $\chi_n$ as the canonical approximate identity corresponding to the frame. 
\end{definition}
The mental model of $\chi_n$ is the strongly convergent symmetric projection in $B(\ell^2(\Zred))$ 

To illustrate the analogy with the frame of a vector bundle, we have the following theorem showing that a $\B$-module has a frame with uniformly bounded correlations if and only if the corresponding \Cstar module is the completion of a projective $\B$ module. 
\begin{proposition}\label{columnfin}
	Let $\B$ be a differentiable algebra with $E_B$ a graded \Cstar module over $B$. Then $E_B$ is the completion of a projective operator module $\E_{\B}$ if and only if there is a frame $(x_i)_{i\in \Zred}$ such that $(\ip{x_i}{x_j})_{i \in \Zred}$ has finite norm in $H_{\B^+}$ for each $j$. 
\end{proposition}
\begin{proof}
	Start by assuming that $\E_{\B}$ is projective, then $\E_{\B}\subset H_{B^+}$ and therefore we may define the frame $x_i=pe_i$ and consider the inner product
	\begin{align*}
		\ip{pe_i}{\sum_{1\leq |j|\leq n} pe_n\ip{pe_n}{pe_j}}=\ip{pe_i}{\sum_{1\leq |j|\leq n} e_n\ip{e_n}{pe_j}}
	\end{align*}
	This converges for $n\to \infty$ since $pe_i$ and all $pe_j$ are in $H_{\B^+}$. This implies that $\ket{pe_i}\bra{pe_k}$ is a frame as $\chi_n$ is a column finite approximate unit for the finite rank operators. 
	To show the converse, we wish to show that the operator $p$ induced on $H_{\B^+}$ by the matrix $(\ip{x_i}{x_j})_{ij\in \Zred}$ is a projection. Start by observing that $p$ has the following domain 
	\begin{align*}
		\dom(p)=\left \{(b_i)_{i\in \Zred} : \forall j \in \Zred \lim_{k\to \infty} \pa{\sum_{1\leq |i|\leq k} \ip{x_i}{x_j}b_j}\in \B \right \}
	\end{align*}
	We see that $p$ is densely defined as all $e_i$ lie in $\dom(p)$ by column finiteness. In order to see that $p$ is closed, let $q_i$ be the projection onto the submodule spanned by $e_i$. As we have assumed that $p$ is column finite, $q_ip\in L(H_{\B^+})$. Let $(z_n)_{n\in \N}$ be a sequence in $H_{\B^+}$ converging to $z$ and assume that $pz_n\to h$. Then $q_ipz_n\to q_ih$. By continuity we also get that $q_ipz_n\to q_iz$ for all $i$.
	This implies that $q_ipz=q_ih$ and that $pz=h\in H_{\B^+}$, so $p$ is closed. 
	We wish to show that $p$ is self-adjoint, so pick $z\in \dom (p^*)$, ie. $\ip{pw}{z}=\ip{w}{x}$ for all $w\in \dom (p)$. As the basis vectors $e_i$ are in the domain of $p$ and using the definition of the adjoint, we may compute:
	\begin{align*}
		\lim_{n\in \N} \sum_{1\leq |i|\leq n} q_ipzn&=\lim_{n\to \infty} \sum_{1\leq |i|\leq n} e_i \ip{e_i}{q_ipz} \\
		&=\lim_{n\in \N} \sum_{1\leq |i|\leq n} e_i \ip{pe_i}{z} \\
		&=\lim_{n\in \N} \sum_{1\leq |i|\leq n} e_i \ip{e_i}{x} \\
		&=x=p^*z
	\end{align*}
	This implies that $pz=x$, giving the desired. 
	Finalizing the proof, we may define $\E_\B$ by
	\begin{align*}
		\E_\B=\{e \in E_B : (\ip{x_i}{e})_{i\in \Zred}\in \dom (p)\}
	\end{align*}
	We clearly have that $x_i\in \E_\B$, and that $\E_\B$ is closed in $H_{\B^+}$ follows from the observation that a convergent net in $\E_\B$ will also be convergent in $E_B$ thus must of the form $(\ip{x_i}{e})_{i\in \Zred}$. 
\end{proof}
\begin{definition}
	Let $\E_\B$ be a projective \Cstar-module. Define the canonical column-finite frame associated to $\E_\B$ as the frame in \Cref{columnfin}.
\end{definition}

We have now constructed our geometric setup, showing that our modules with frames behave as generalized vector bundles. As a next logical step, we wish to construct the analogue of the differentiable sections of our bundle. For this purpose we start by defining the set of universal 1-forms, from which we shall eventually construct our analogue of the differentiable sections.  
\begin{definition}
	We define the universal 1-forms $\Omega^1(B,\B)$ as the kernel of the map $B\optens \B\to B$ given by $b_1\tens b_2\mapsto \gamma(b_1)b_2$ when $B,\B$ are unital. In the non-unital case, we consider $\Omega(B^+,\B^+)$ such that the universal derivation:
	\begin{align*}
		&db:B\to \Omega^1(B^+,\B^+) \\
		&b\mapsto 1\tens b+\gamma(b)\tens 1
	\end{align*}
	is well-defined. 
	Associated to this we have the universal short exact sequence where $m$ is the multiplication map:
	\begin{align*}
	\xymatrix{
		0 \ar[r] & \E_B \optens_{\B^+}\Omega^1(B^+,\B^+) \ar[r] & E_B\optens_{\B} \B^+ \ar[r]^m & E_B \ar[r] & 0 
	}
	\end{align*}
	A split of this is a map $s:\E_\B\to E_B\optens \B^+$ such that $m(s)=\iota_{\E_\B}$, where $\iota_{\E_\B}$ is the inclusion of $\E_\B$ into $E_B$, \cite[Proposition 2.22]{suijlekom}. 
\end{definition}
We are interested in splittings the universal exact sequence for the universal $1$-forms, as we may use such a split to construct an analogue the covariant derivative based on the universal derivative, which we then use to construct an explicit form for a covariant derivation associated to an operator $\D$. 
We may construct a split using our column finite frames. 
\begin{lemma}
	Let $(x_i)_{i\in \Zred}$ be a column finite frame defining a projective $\B$ submodule $\E_\B \subset E_B$. Then the map 
	\begin{align*}
		&s:\E_\B\to E\optens \B^+ \\
		&e\mapsto \sum_{i\in \Zred} \gamma(x_i)\tens \ip{x_i}{e}, \quad \for e\in \E_\B
	\end{align*}
	defines a contractive $\B^+$-linear split of the universal exact sequence. 
\end{lemma}
\begin{proof}
	We start by checking that $s$ is well-defined. Let $\epsilon>0$ and $e\in \E_\B$ and pick $n,m\in \N$ such that 
	\begin{align*}
		\norm{\sum_{n\leq |i|\leq m} \pi_{\D}(\ip{x_i}{e})^*\pi_{\D}(\ip{x_i}{e})}_{L(E_B\osum E_B)}<\epsilon
	\end{align*}
	which we may do as $e\in \E_\B$. 
	We can perform the following estimate
	\begin{align*}
		&\norm{\sum_{n\leq \abs{i} \leq m} \gamma(x_i)\tens \ip{x_i}{e}}^2_{\optens} \\
		&\leq \norm{\sum_{n\leq |i|\leq m} \ket{x_i}\bra{x_i}}_{\K(E_B)}\norm{\sum_{n\leq |i|\leq m}\pi_{\D}(\ip{x_i}{e})^*\pi_{\D}(\ip{x_i}{e})}_{\B^+} \\
		&\leq \norm{\sum_{n\leq |i|\leq m} \pi_{\D}(\ip{x_i}{e})^*\pi_{\D}(\ip{x_i}{e})}_{{L(E_B\osum E_B)}}<\epsilon
	\end{align*}
	Thus the partial sums from the definition of $s$ give rise to a Cauchy sequence in $\optens$-norm. To show continuity, consider the following estimate
	\begin{align*}
		&\norm{s(e)}^2_{\optens} \\
		&\leq \lim_{k\to\infty} \norm{\sum_{1\leq |i|\leq k} \ket{x_i}\bra{x_i}}_{\K(E_B)} \norm{\sum_{1\leq |i|\leq k} \pi_{\D}(\ip{x_i}{e})^*\pi_{\D}(\ip{x_i}{e})}_{\B} \\
		&\leq \norm{e}_{\E_{\B}}^2
	\end{align*}
	This shows that our split is well-defined and contractive as desired. 
\end{proof}

We can now use this split to define our connection. 
%The motivation for working with splits of the universal exact sequence is that these turn out to correspond to generalized connections as required to construct the unbounded Kasparov product, while at the same giving convenient maps from the universal property of the deri%vations. 

\begin{definition}
	 Let $\E_\B$ be a projective operator module, then any completely bounded linear operator $\nabla: \E_\B\to E\optens \Omega^1(B^+,\B^+)$  satisfying the Leibniz rule for the universal derivation $d$
	\begin{align*}
		\nabla(eb)=\nabla(e)b+\gamma(e)\tens db
	\end{align*}
	is a connection. 
\end{definition}
The motivating example for this definition is the following  
\begin{example}
	Given a splitting $s$ we may define a connection associated to this split.
	\begin{align*}
		\nabla_s(e)=s(e)-\gamma(e)\tens 1
	\end{align*}
	For the module $H_{\B^+}$ we may define the connection $(\epsilon d)((b_i)_{i\in \Zred})=(\epsilon((d b_i)))_{i\in \Zred}$. 
\end{example}

Before proceeding, we define an isometry $v$ implementing the stablization map $\E_\B \to H_{\B^+}$, providing a framework into which to place the split $s$. 	Given a column finite frame $(x_i)_{i\in \Zred}$ and a projective module $\E_\B$ we get an induced stabilization map, $v:\E_\B\to H_{\B^+}$ extending to $E_B$ and $H_{B^+}$. 
\begin{definition}
	For a projective operator module $\E_{\B}$ define the isometry $v$ implementing the stabilization $\E_{\B}\to H_{\B^+}$. 
	\begin{align*}
		v(e)=(\ip{e}{x_i})_{i\in \Zred}
	\end{align*}
	The adjoint $v^*$ of $v$ is 
	\begin{align*}
		v^*((b_i)_{i\in \Zred})=\sum_{i \in \Zred} x_i b_i
	\end{align*}
	Define the associated projection $p=vv^*$. 
\end{definition}
In order to understand the utility of the example given above, we have the following lemma illuminating the concrete form of the covariant derivative in terms of the frame. 
\begin{lemma}
	The operator $v$ is an even isometry, and the connection associated to the splitting may be characterized through $v$ as follows $\nabla_s=v^*\epsilon dv$. 
\end{lemma}
\begin{proof}
	We have 
	\begin{align*}
		\nabla_s(e)&=s(e)-\gamma(e)\tens 1\\
		&=\sum_{i\in \Zred} \gamma(x_i)\tens \ip{x_i}{e}-\gamma(e)\tens 1 \\
		&=\sum_{i\in \Zred} (\gamma(x_i)\tens 1)(1\tens \ip{x_i}{e}-\gamma(\ip{x_i}{e})\tens 1) \\
		&=\sum_{i\in \Zred} \gamma(x_i)\tens d(\ip{x_i}{e})
	\end{align*}
	By the calculations we have just performed, we see that $\nabla_s=v^*\circ \epsilon d \circ v$. 
\end{proof}
As we desired that our framework should also work for unbounded, ie. non-regular projections, we need to consider a concrete derivation constructed from $\D$ which we may use to construct a suitable connection. 
\begin{definition}
	Define the derivation $\delta_\D$ as $[\D,\cdot]$. We define the set of $\delta_{\D}$-1-forms as 
	\begin{align*}
		\Omega^1_\D=\overline{\{\pi(b_i)[\D,b_i']: b_i,b_i'\in \B \}}\subset L_C(F_C)
	\end{align*}
	By the universality of $\Omega^1(B^+,\B^+)$ there is a map 
	\begin{align*}
		&j_\D:\Omega^1(B^+,\B^+)\to \Omega^1_\D & 
		&db\mapsto [\D,\pi(b)]
	\end{align*}
	Thus we get a connection $\nabla_\D=(1\tens j_\D) \circ (\nabla_s)$. 
\end{definition}
We might be in the situation that the space of differentiable elements in $\E_\B\optens_{\B^+} F_C$ with respect to $\nabla$ is not complete, and as such we need to enlarge our module to remedy this malady. We start by considering the free case, and use this as our reference. 
\begin{definition}
	We define the space 
	\begin{align*}
		H_{\Omega^1_\D}=H_{\B^+}\optens_{\B^+} \Omega^1_\D
	\end{align*}
	consisting of sequences $(\omega)_{j\in \Zred}$ such that the sum $\sum_{j \in \Zred} \omega_j^*\omega_j$ converges in $L_C(F_C)$
\end{definition}
This allows us to define our operator module $\E^\nabla$ which can see both the action of $\D$ on $\B$ and $F_C$.
\begin{definition}
	Given a column-finite frame $(x_i)_{i\in \Zred}$ for $\E_\B$ define the space $\E^\nabla_\B\subset E_B$: 
	\begin{align*}
		\E^\nabla_\B=\left \{e\in E_B : \lim_{n\to \infty} \pa{\sum_{1\leq |k| \leq n} \ip{x_i}{\gamma(x_k)}[\D,\ip{x_k}{e}]}_{i\in \Zred} \in H_{\Omega^1_\D}\right \}
	\end{align*}
	This is an operator $\B$ module in the representation: 
	\begin{align*}
		\pi_\nabla(e)=\begin{pmatrix} v(e) & 0 \\ vv^*\epsilon[\D,v(e)] & v(\gamma(e)) \end{pmatrix} \in \bigoplus_{i\in \Zred} L_C(F_C\osum F_C), \quad \norm{e}_{\E^\nabla_\B}=\norm{\pi_\nabla(e)}
	\end{align*}
	where $\gamma=\diag(\gamma_{\B^+})$. We shall use the notation $\D_\epsilon=\epsilon \diag(\D)$ on $H_{B^+}\optens_{B^+} F_C$.   
\end{definition}
\begin{remark}\label{remark36}
	The two closed graded derivations defined below 
	\begin{align*}
		[\diag(\D),T]_\gamma&=\diag(\D)T-\gamma T \gamma \diag \D 
		[\D_\epsilon,T]_\Gamma=\D_\epsilon T-\Gamma T\Gamma \D_\epsilon
	\end{align*}
	are related as $[\D_\epsilon,T]_\Gamma=[\diag \D,\epsilon T]_\gamma=\epsilon [\diag(\D),T]_\gamma$. 
	Thus they have the same domain. One should also remark that with the gradings defined on $\E^\nabla_\B$ as before, we have the identity:
	\begin{align*}
		\pa{\sum_{1\leq |k| \leq n} \ip{x_i}{x_k}[\D,\ip{x_k}{e}]}^*&=\sum_{1\leq |k| \leq n} -[\D,\gamma(\ip{e}{x_k})]\ip{\gamma(x_k)}{x_i} \\
		&=\sum_{1\leq |k| \leq n} \gamma([\D,\ip{e}{x_k}]\ip{x_k}{\gamma(x_i)})
	\end{align*}
	So that the sequence of row vectors
	\begin{align}
		\pa{\sum_{1\leq |k|\leq n}[\D,\ip{e}{x_k}]\ip{x_k}{\gamma(x_i)}}^t_{i\in \Zred} \label{310}
	\end{align}
	converges.
\end{remark}

As we shall see in the following lemma, the module $\E_{\B}^\nabla$ is isomorphic to $\E_{\B}$ in the case where $\E_{\B}$ is projective, reinforcing the analogy of projective $\B$-modules as being unbounded $C^1$-vector bundles in a suitable sense. We also show that $\E_{\B}$ is an inner product $\B$-module, allowing us to use our results on these. 
\begin{theorem}
	The operator module $\E_\B^\nabla$ has the following properties. 
	\begin{enumerate}
		\item
			The inner product $\E_\B\times \E_\B \to \B$ extends to $\E_\B^\nabla\times \E_\B^\nabla\to \B$. 
		\item
			For every $e\in \E_\B^\nabla$ the operator $e^*:\E_\B^\nabla:\to \B$ given as $e^*(f)=\ip{e}{f}$ is completely bounded and adjointable, with adjoint given as $(e^*)^*(b)=eb$, satisfying the estimate $\norm{e^*}_{cb}\leq 2\norm{e}_{\E_\B^\nabla}$.
		\item
			For every projection $p:\dom (p)\to H_{\B^+}$ such $\E_\B=p\dom (p)$ is a projective module, there is a completely contractive dense inclusion $\iota:\E_\B \to \E_\B^\nabla$, if $p\in L(H_{\B^+})$, $\iota$ is a $cb$-isomorphism. 
	\end{enumerate}
	\begin{proof}
	\begin{enumerate}
	\item
		Let $f,e\in \E^\nabla_\B$ we wish to show that $\ip{e}{f}$ lies in $\B$. Let $(x_i)_{i\in \Zred}$ be the canonical frame for $\E_\B$. Consider the series of column vectors   
		\begin{align*}
			\sum_{j\in \Zred} \pa{\ip{x_i}{\gamma(x_j)}[\D,\ip{x_j}{e}]}_{i\in \Zred}
		\end{align*}
		which for $e\in \E^\nabla$ is norm-convergent in $H_{\B^+}\optens_{\B^+} F_C$ by definition of $\E_{\B}^\nabla$. 
		Then we may consider the partial sums 
		\begin{align*}
			\comm{\D,\sum_{1\leq |j|\leq n} \ip{e}{x_j}\ip{x_j}{f}}=\sum_{1\leq |j|\leq n} \gamma(\ip{e}{x_j})[\D,\ip{x_j}{f}]+[\D,\ip{e}{x_j}]\ip{x_j}{f}
		\end{align*}
		In order to see that both terms on the right hand side are convergent, consider the following where we use the pairing between row and column vectors, ie. the standard inner product product, we get 
		\begin{align*}
			\norm{\sum_{1\leq |j|\leq n} \gamma(\ip{e}{x_j})[\D,\ip{x_j}{f}]}&=\norm{\sum_{1\leq |j|\leq n}\sum_{i\in \Zred} \ip{\gamma(e)}{x_i}\ip{x_i}{\gamma(x_j)}[\D,\ip{x_j}{f}]}  \\
			&=\norm{\sum_{1\leq |j|\leq n}(\ip{\gamma(e)}{x_i})^*_{i\in \Zred} \cdot \pa{\ip{x_i}{\gamma(x_j)}[\D,\ip{x_j}{f}]}_{i\in \Zred}} \\
			&\leq \norm{e}_{E_B} \norm{\sum_{1\leq |j|\leq n} (\ip{x_i}{\gamma(x_j)}[\D,\ip{x_j}{f}])_{i\in \Zred}}
		\end{align*}
		by our initial considerations this is finite. Thus we have shown the desired since $\delta_\D$ is a closed derivation and $x_i$ is a frame for $\E_\B$, so $\sum_{i\in \Zred}(\ip{e}{x_i})\ip{x_i}{f}$ converges to $\ip{e}{f}$.
	\item
		We have the following equalities
		\begin{align*}
			\begin{pmatrix} \ip{e}{f} & 0 \\ [\D,\ip{e}{f}] & \gamma(\ip{e}{f}) \end{pmatrix}=\sum_{i\in \Zred}\begin{pmatrix} \ip{e}{x_i} & 0 \\ 0 & \ip{\gamma(e)}{x_i} \end{pmatrix}\begin{pmatrix} \ip{x_i}{f} & 0 \\ \sum_{j\in \Zred} \ip{x_i}{\gamma(x_j)} & [\D,\ip{x_i}{f}]\end{pmatrix} \\
			+\begin{pmatrix} 0 & 0 \\ \sum_{j\in \Zred} [\D,\ip{e}{x_j}] & \ip{x_j}{\gamma(x_i)}\end{pmatrix}\begin{pmatrix} \ip{\gamma(x_i)}{f} & 0 \\ 0 & \ip{x_i}{f}\end{pmatrix}
		\end{align*}
		All series converge by \Cref{310} and the definition via. limits of $\E^\nabla_\B$.
		These equalities also hold for matrices of elements in $\E^\nabla_\B$, giving rise to the estimate 
		\begin{align*}
			\norm{e^*(f)}_{\B}\leq \norm{e}_E\norm{f}_{\E^\nabla_\B}+\norm{e}_{\E^\nabla_\B}+\norm{f}_E\leq 2\norm{e}_{\E^\nabla_\B}\norm{f}_{\E^\nabla_\B}
		\end{align*}
		so we infer 
		\begin{align*}
			\norm{e^*}_{cb}\leq 2 \norm{e}_{\E^\nabla}
		\end{align*}
		Which proves the claim.
	\item
		To see the final point we start by using that $v$ is a partial isometry
		\begin{align*}
			&\norm{\begin{pmatrix}v(e) & 0 \\ vv^*\epsilon[\D,v(e)] & \gamma(v(e)) \end{pmatrix}}\\
			&=\norm{\begin{pmatrix} p & 0 \\ 0 & p \end{pmatrix} \begin{pmatrix} v(e)& 0 \\ \epsilon[\D,v(e)] & v(\gamma(e))  \end{pmatrix}} \\
			&\leq \norm{\begin{pmatrix} v(e)& 0 \\ \epsilon[\D,v(e)] & v(\gamma(e))  \end{pmatrix}}
		\end{align*}
		which shows the first of the statement. Recall that $\Gamma$ is the grading operator on $H_{\B^+}$ and that as $p$ is even, we have $\Gamma p=p\Gamma$. 
		We may also deduce the following relations 
		\begin{align*}
			p\epsilon&=p\gamma \epsilon \gamma \\
			&=p\Gamma \gamma=\Gamma p \gamma=\epsilon \gamma p \gamma \\
			&=\epsilon \gamma(p)\\ 
			&[\D_\epsilon,p]_\Gamma v(e)=[\diag(\D),\epsilon p]_{\gamma}v(e) \\
			&=[\D,\epsilon v(e)]-\epsilon \gamma p \gamma [\D,v(e)]=[\D,\epsilon v(e)]-p\epsilon[\D,v(e)]
		\end{align*}
		These two considerations taken together allow us to write the following equality
		\begin{align*}
			\begin{pmatrix} v(e) & 0 \\ vv^* \epsilon [\D,v(e)] & \gamma(v(e))\end{pmatrix}=\begin{pmatrix} 1 & 0 \\ -[\D_\epsilon,p]_\Gamma & 1\end{pmatrix}\begin{pmatrix} v(e) & 0 \\ \epsilon [\D,v(e)] & v(\gamma(e))\end{pmatrix}
		\end{align*}
		By boundedness of the commutator in the matrix $\begin{pmatrix} 1 & 0 \\ -[\D_\epsilon,p] & 1 ] \end{pmatrix}$, it is invertible. This gives the desired inverse. 
	\end{enumerate}
	\end{proof}
\end{theorem}
In order to construct the unbounded Kasparov product, we need some analogues of the compact operators and the adjointables. Further, we need to show that these have the desired differential structure. Doing this construction and proving that the algebras have the appropriate structure for our purposes is the goal of the next couple of pages. 
\begin{definition}
	We may view $\E_\B$ as a proper submodule of $\E_\B^\nabla$. We consider the finite rank operators $Fin(\E_\B)$ as an algebra of operators on $\E_\B^\nabla$ via. the following representation, for $K\in Fin(\E_{\B})$. 
	\begin{align*}
		\pi_\nabla(K)=\begin{pmatrix} vKv^* & 0 \\ p[\D_\epsilon,vKv^*]p & vKv^* \end{pmatrix}
	\end{align*}
	where $P$ comes from a defining column frame, and $v$ is the isometry $\E_\B\to H_{B^+}$. 
\end{definition}
\begin{lemma}
	The representation above is well-defined on the finite rank operators. 
\end{lemma}
\begin{proof}
For $e,f\in \E_\B$ we consider the column and row vectors 
\begin{align*}
	&v\ket{e}=v(e)=(\ip{x_i}{e})_{i\in \Zred} \\
	&\bra{f}v^*=v(f)^*=(\ip{f}{x_i})_{i\in \Zred}^t
\end{align*}
where $^t$ is the transpose. These are elements of $H_{B^+}$ and $H_{B^+}^t$ respectively. Therefore the rank one operator given as $\ket{e}\bra{f}$ satisfies that 
\begin{align*}
	[\D_\epsilon,v\ket{e} \bra{f} v^*]
\end{align*}
is a bounded matrix. This shows that the representation is well-defined.
\end{proof}
This allows us to define the compact operators $\K(\E^\nabla_\B)$ as the closure of $\pi_\nabla(Fin_{\B}(\E_\B))$ in operator space norm. We are now in a situation where we may show existence of suitable approximate units. 
\begin{lemma}
	Let $(\B,F_C,\D)$ be a defining Kasparov module for $\B$ and $\E_\B$ a projective operator module. Let  $(u_n)_{n\in \N}$ be the canonical approximate unit associated to the column finite frame $(x_i)_{i\in \Zred}$, and let $K\in Fin_\B(\E_\B)$. Then $[\D_\epsilon,vKv^*]$ extends to a bounded adjointable operator in $L_C(H_{B^+}\optens_{B^+} F_C)$ and 
	\begin{align}
		&vKv^* \dom (\D_\epsilon) \subset \dom (\D_\epsilon) \\
		&\lim_{n\to \infty} v\chi_n v^* [\D_\epsilon,vKv^*]=vv^*[\D_\epsilon,vKv^*] \\
		&\lim_{n\to \infty}[\D_\epsilon,vKv^*]v\chi_nv^*=[\D_\epsilon,vKv^*]vv^*
	\end{align}
	where all limits are in operator norm. 
\end{lemma}
\begin{proof}
	By linearity and continuity it is sufficient to show this for rank one operators, ie. for $K=\ket{e}\bra{f}$. Then
	\begin{align*}
		vKv^*=(\ip{x_i}{e}\ip{f}{x_i})_{ij\in \Zred}\in \K\optens \B
	\end{align*}
	and thus lies in the domain of $[\D_\epsilon,\cdot]$. We handle only one of the limits, as the proof of the other equality is identical in form. Here we tacitly use the identities for $\D_\epsilon$ established in \Cref{remark36}.
	\begin{align*}
		&\lim_{n\to \infty} v\chi_nv^* [\D_\epsilon,vKv^*]=\lim_{n\to \infty} \pa{\sum_{1\leq |k|\leq n}\ip{x_i}{\gamma(x_k)}[\D,\ip{x_k}{e}\ip{f}{x_j}]}_{i,j\in \Zred} \\ 
		&=\lim_{n\to \infty} \pa{\sum_{1\leq |k|\leq n } \ip{x_i}{\gamma(x_k)}\gamma(\ip{x_k}{e})[\D,\ip{f}{x_j}]+\ip{x_i}{\gamma(x_k)}[\D,\ip{x_k}{e}]\ip{f}{x_j}}_{i,j\in \Zred} \\		
		&=(\ip{x_i}{\gamma(e)}[\D,\ip{f}{x_j})_{i,j \in \Zred} +\lim_{n\to \infty} \pa{\sum_{1\leq |k|\leq n} \ip{x_i}{\gamma(x_k)}[\D,\ip{x_k}{e}]\ip{f}{x_j}}_{i,j\in \Zred}
	\end{align*}
	The first term is well-defined as $f\in \E_\B$ and the second is well-defined as $e\in \E_\B \subset \E_\B^\nabla$
\end{proof}
We may now define the non-commutative notion of completeness of our non-commutative analogues of vector bundles. 
\begin{definition}
	As usual, we let $(\B,F_C,\D)$ be an unbounded Kasparov module and let $\E_\B$ be a projective operator module with canonical approximate unit $(\chi_n)_{n\in \N}$. 
	If there is an approximate unit $(u_n)_{n\in \N}\subset \conv\{\chi_n: n\in \N \}$ such that for all $K\in \K(E_B)$ the sequence 
	\begin{align*}
		p[\D_\epsilon,vu_nv^*]p
	\end{align*}
	converges to zero strictly, $\E_\B$ is a complete projective operator module. 
	Note this is a sequence of operators $H_{B^+}\optens_{B^+} F_C\to H_{B^+}\optens_{B^+} F_C$. In light of \Cref{mesrennie19} this implies that $u_n\to p$.
\end{definition}
\begin{lemma}
	Let $\E_\B$ be a complete projective operator module over $\B$. Then $\K(\E^\nabla_\B)$ has a bounded approximate unit consisting of elements in $\conv(\chi_n)$. 
\end{lemma}
\begin{proof}
	 Let $\chi_n=\sum_{1\leq |i|\leq n}\ket{x_i}\bra{x_i}$ be the canonical approximate unit associated to a column finite frame. Let $(u_n)_{n\in \N}$ be an approximate unit as in the assumptions. 
	 For each $x\in H_{B^+}\optens_{B^+} F$ the sequence $p[\D,vu_nv^*]px$ converges, implying that $\sup_{n\in \N} \norm{p[\D_\epsilon,vu_nv^*]px}<\infty$ hence $\sup_{n\in \N} \norm{p[\D_\epsilon,vu_nv^*]p}<\infty$ 	 
	 It follows from the uniform boundedness principle that $\sup_{n}\norm{\pi_\nabla(u_n)}<\infty$.
	 Picking $K\in Fin_{\B}(\E_\B)$ and using that $vKv^*$ is domain preserving for $\D_\epsilon$, we calculate:
	 \begin{align*}
		p[\D_\epsilon,vu_nKv^*]p=p[\D_\epsilon,vu_nv^*]vKv^*+vu_nv^*[\D_\epsilon,vKv^*]p
	 \end{align*}
	 The final term converges to $p[\D_\epsilon,vKv^*]p$. By continuity and unfirom boundedness, it follows $\pi_\nabla(u_nK)\to \pi_\nabla(K)$ for all $K\in \K(\E^\nabla_\B)$. 
\end{proof}
\begin{proposition}
	Let $\E_\B=p\dom (p)$ be a projective operator module with defining column finite frame $(x_i)_{i\in \Zred}$, and canonical approximate unit $\chi_n$. Then either of the conditions:
	\begin{enumerate}
		\item
			There exists an approximate unit $u_n\in \conv\{\chi_n : n\in \N\}$ for $\K(E_B)$ such that $p[\D_\epsilon,u_n]p\to 0$ in norm on $H_{\B^+}\optens_{\B^+} F_C$. 
		\item
			The projection $p$ is a countable direct sum of finite even projections $p_k\in M_{2m_k}(\B^+)$. 
		\item
			The projection $p$ lies in $L(H_{\B^+})$. 
	\end{enumerate}
	is sufficient to infer completeness of the module.
\end{proposition}
\begin{proof}
\begin{enumerate}
	\item As norm convergence implies strict convergence, we get completeness immediately. 
	\item As such it suffices to show that the second condition implies the first. 
	Given a countable family of finite projections with $[\D_\epsilon,p_i]$ bounded, it holds that $p_i[\D_\epsilon,p_i]p_i=0$, and we have
	\begin{align*}
		p_k=\sum_{1\leq |i|\leq m_k} \ket{pe_i^k}\bra{pe_i^k}
	\end{align*}
	We may identify $\bigoplus_{k=0}^\infty (\B^+)^{2m_k}$ with $H_{\B^+}$ and define $p=\osum_{i=1}^\infty p_i$, thus getting the approximate unit $u_n=\osum_{i=1}^n p_i$. As $p_i$ is defined explicitly we see that $u_n$ is a subsequence of the approximate unit associated to the frame $(pe_k^i)$. Hence $u_n$ lies in the convex hull of the canonical approximate unit.  
	To finalize the argument, observe that $p[\D,u_n]p=\sum_{i=1}^n p[\D,p_i]p=\sum_{i=1}^n p_i [\D,p_i]p_i=0$. 
	\item In order to see that the third condition implies completeness of the module, we remark that $p\in L_{\B^+}(H_{\B^+})$ if and only if $p \tens Id_F$ preserves the domain of $\D_\epsilon$ and the commutator $[\D_\epsilon, p\tens Id_F]$ is adjointable.
	Letting $q_k$ be the increasing family of symmetric projections onto $e_i,1\leq |i|\leq k$. Define $x_i=pe_i$ and the approximate unit $\chi_n=\sum_{i=1}^n \ket{x_i}\bra{x_i}$. 
	We see that for all $y=px$ we have $\chi_n y=pq_npy=pq_n y$. As such we can calculate as below
	\begin{align*}
		p[\D_\epsilon,\chi_n]p=p[\D_\epsilon,pq_n]p=p[\D_\epsilon,p]q_np
	\end{align*}
	As $q_n$ converges strongly to the identity and the commutator is bounded, we see that $p[\D_\epsilon,\chi_n]p x\to p[\D_\epsilon,p]px=0$. 
	\end{enumerate}
\end{proof}
This establishes the second leg of the tripod on which the construction of the Kasparov product rests, and we can now construct the third leg based on the method of localization.
\begin{assumption}
	From here $\E_\B$ is a projective operator module. We shall be working with the fixed Kasparov module $(\B,F_C,\D)$. 
\end{assumption}
\begin{definition}
	Define the operator 
	\begin{align*}
		(1\tens_\nabla \D)(e\tens f)=\gamma(e)\tens \D(f)+\nabla_\D(e)f
	\end{align*}
	on elementary tensors. We shall consider minimal closure of $1\tens_\nabla \D$ on $\E\tens_\B \dom \D$.
\end{definition}
This operator shall turn out take the on the role of covariant derivative on the interior tensor product, where $\nabla_\D$ serves to symmetrize the operator. 
	We may define the stabilized version of $1\tens_\nabla \D$ operator, which is more amenable to calculations. 
\begin{definition}
	Recall that $p=vv^*$ and define $\dom (\part) = v\dom (1\tens_\nabla \D)\osum (1-p)H_\B \optens_{B^+} F_C$ with the operator $\part$ defined as 
	\begin{align*} 
		\part&=v(1\tens_\nabla \D)v^* \\
		\part(vy+(1-p)z)&=v(1\tens_\nabla \D)y
	\end{align*}
	We may thus denote it as $v(1\tens_\nabla \D)v^*$. 
\end{definition}
%\todo{why does v makes sense on the tensor product? Shorthand for $v^*\tens 1$?}
This transformation preserves the geometric information of $1\tens_\nabla \D$, as evidenced by the proposition below. 
\begin{lemma}\label{mesrennie313}
	The operator $\part$ is self-adjoint and regular if and only if $1\tens_\nabla \D$ is self-adjoint and regular. 
\end{lemma}
\begin{proof}
	A closed densely defined symmetric operator $T$ is self-adjoint and regular if and only $T\pm i:\dom T\to E$ have dense range. Assume that $(1\tens_\nabla \D)\pm i$ both have dense range. Picking $x=vy+(1-p)z\in \dom 1\tens_\nabla \D$ we get 
	\begin{align*}
		&(\part \pm i)x=v(1\tens_\nabla \D \pm i)\pm i(1-p)z \\
		&(1\tens_\nabla \pm i)y=v^*(\part \pm i )x
	\end{align*}
	As $\ran v$ and $\ran (1-p)$ are orthogonal, it can be seen that $(\part \pm i)$ has dense range in $H_{B^+}\optens_{B^+} F_C$ if and only if $(1\tens_\nabla \D)\pm i$ has dense range in $(E_B \optens_B F_C$
\end{proof}
This reduces our problem to showing self-adjointness and regularity of $1\tens_\nabla \D$ to considering $\part$. As $B$ is represented essentially on $F_C$ we have the canonical isomorphism 
\begin{align*}
	H_{B^+}\optens_{B^+} F_C \cong \bigoplus_{i\in \Zred} F_C
\end{align*}
%given by mapping $e_i \tens x$ to the sequence with $x$ in the $i'th$ position. 
Thus $\D_\epsilon$ is equivalent to $1\tens_d \epsilon \D$ where $\epsilon d$ is the trivial connection. We can now show that the operator $1\tens_\nabla \D$ is well-defined on the interior tensor product through the following lemma giving an explicit formula on elementary tensors. More importantly, we can evantualyl use the lemma to show that $\part$ is self-adjoint and regular by the continuity of the map $g$ defined therein. 
\begin{lemma}\label{mesrennie314}
	Let $(\B,F_C,\D)$ be the essential unbounded Kasparov module defining $\B$, and let $\E_\B\subset \E^\nabla_\B$ be a graded complete projective module with defining frame $(x_i)_{i\in \Zred}$. We may express $1\tens_\nabla \D$ on elementary tensors $e\tens f\in \E\tens_{\B^+} \dom \D$
	\begin{align}
		\gamma(e)\tens \D f+\nabla(e) f&=\sum_{i\in \Zred} x_i \tens \ip{x_i}{\gamma(e)}\D f+\gamma(x_i)\tens [\D,\ip{x_i}{e}]f  \\
		&=\sum_{i\in \Zred} x_i \tens \ip{x_i}{\gamma(e)}\D f+\sum_{i,j \in \Zred} x_i\tens \ip{x_i}{\gamma(x_j)}[\D,\ip{x_j}{e}]f \label{317} \\ 
		&=\sum_{i\in \Zred} x_i \tens \D\ip{\gamma(x_i)}{e}f
	\end{align}
	In particular, this entails that $1\tens_\nabla \D=v^* \part v=v^* \D_\epsilon v$ on $\E \tens_\B \dom \D$ and that $\part=p\D_\epsilon p$ on $v\E \tens_{\B^+} \dom \D$. 
	The map 
	\begin{align*}
		&g:\E^{\nabla}_{\B}\optens_{\B^+}G(\D)\to G(1\tens_\nabla \D) \\
		&e\tens \begin{pmatrix} f \\ \D f\end{pmatrix} \mapsto \begin{pmatrix} e\tens f  \\ (1\tens_\nabla \D)(e\tens f) \end{pmatrix}
	\end{align*}
	is a completely bounded operator with dense range. 
	Thus $1\tens_\nabla \D$ is continuous in graph norm, allowing us to expand the result by continuity. 
\end{lemma}
\begin{proof}
	Our first goal will be to show that the sum in \Cref{317} is convergent, so that $g$ is well-defined. 
	The first term converges as $\chi_n$ is an approximate unit by assumption. To see the second term converges, let $z\in \E^\nabla_{\B}\optens_{\B^+} G(\D)$.  Let $\sum_{k\in \Zred} e_k\tens f_k \in \E^{\nabla}_{\B}\odot_{\B^+}G(\D)$, be a representation of $z$. Then consider the estimate below, where we repeatedly use the canonical approximate unit stemming from the defining frame. 
	\begin{align}
		&~\norm{\sum_{i,j,k \in \Zred} x_i\tens\ip{x_i}{\gamma(x_j)}[\D,\ip{x_j}{e_k}]f_k}^2_{\optens} \\
		&\leq \norm{\sum_{i\in \Zred} \ket{x_i}\bra{x_i} }_{\K(E_B)}\norm{\pa{\sum_{j,k\in \Zred} \ip{x_i}{\gamma(x_j)}[\D,\ip{x_j}{e_k}]f_k}_{i\in \Zred} }_{E_B\osum E_B}^2 \label{completelyboundedeq}
	\end{align}
	We wish to estimate \Cref{completelyboundedeq}, to this end note that 
	\begin{align*}
		\ip{\sum_{j,k\in \Zred} \ip{x_i}{\gamma(x_j)}[\D,\ip{x_j}{e_k}]f_k}{\sum_{j,k\in \Zred} \ip{x_i}{\gamma(x_j)}[\D,\ip{x_j}{e_k}]f_k}&=\ip{(f_k)_{k\in \Zred}}{\pi_{\nabla}((e_k)_{k\in \Zred})\pi_{\nabla}((e_k)_{k\in \Zred})^*(f_k)_{k\in \Zred}}
	\end{align*}
	Thus, by complete boundedness of the representation $\pi_\nabla$ we can continue our estimates as
	\begin{align*}
	&\norm{\pa{\sum_{j,k\in \Zred} \ip{x_i}{\gamma(x_j)}[\D,\ip{x_j}{e_k}]f_k}_{i\in \Zred} }_{E_B\osum E_B}^2 \\
		&\leq \norm{\sum_{k\in \Zred} \pi_{\nabla}(e_k)\pi_\nabla(e_k)^*}_{L(F_C\osum F_C)}\norm{\sum_{k\in \Zred } \ip{f_k}{f_k}}_{E_B} \\
		&\leq \norm{\sum_{k\in \Zred} \pi_{\nabla}(e_k)\pi_\nabla(e_k)^*}_{L(F_C\osum F_C)}\norm{\sum_{k\in \Zred } \ip{\begin{pmatrix}f_k \\ \D f_k\end{pmatrix}}{\begin{pmatrix}f_k \\ \D f_k\end{pmatrix}} }_{E_B\osum E_B}
	\end{align*}
	which shows that \Cref{317} and the following sums are convergent. 
	In order to show continuity of $g$, we still need to control $e\tens f \mapsto \gamma(e)\tens \D f$. We start by recalling the result, that $E\optens_B F$ is isometrically isomorphic to $E\tens_B F$ if $E,F$ are Hilbert modules, \cite{blechernew}. For this purpose, we have the following estimates, where again $e_k$ and $f_k$ are non-zero only for finitely many $k$.
	\begin{align*}
		\norm{\sum_{k\in \Zred} \gamma(e_k)\tens \D f_k}^2 \leq \norm{\sum_{k\in \Zred} \ket{\gamma(e_k)}\bra{\gamma(e_k)}}_{\K(E)}\norm{\sum_{k\in \Zred} \ip{\D f_k}{\D f_k}} \\
		\leq \norm{\sum_{k\in \Zred} \pi_{\nabla}(e_k)\pi_\nabla(e_k)^*}\norm{\sum_{k\in \Zred } \ip{\begin{pmatrix}f_k \\ \D f_k\end{pmatrix}}{\begin{pmatrix}f_k \\ \D f_k\end{pmatrix}} }
	\end{align*}
	Drawing the above estimates together, we get the estimate:
	\begin{align*}
		\norm{g\pa{\sum_{k\in \Zred} e_k \tens \begin{pmatrix} f_k \\ \D f_k \end{pmatrix}}}_{G(1\tens_\nabla \D)} &\leq 2\norm{\sum_{k\in \Zred} \pi_{\nabla}(e_k)\pi_\nabla(e_k)^*}\norm{\sum_{k\in \Zred } \ip{\begin{pmatrix}f_k \\ \D f_k\end{pmatrix}}{\begin{pmatrix}f_k \\ \D f_k\end{pmatrix}} } \\
		&\leq 2\norm{\sum_{k\in \Zred} e_ke_k^*}_{\E^\nabla_{\B^+}}\norm{f_k^*f_k}_{G(\D)}
	\end{align*}
	%Letting $z\in \E^{\nabla}_\B \odot G(\D)$, $z= \sum_{k\in \Zred} e_k\tens f_k$, by the above we have that $g(z)$ is bounded for every presentation of $z$, 
	%thus $g$ is continuous in the Haagerup norm. 
	%Given an element $z\in \E^\nabla_\B \optens_{\B^+} G(\D)$ presented as a finite sum
	%\begin{align*}
	%	z=\sum_{k\in \Zred} e_k \tens f_k 
	%\end{align*}
	recall that the Haagerup norm is defined as the infimum 
	%\begin{align*}
	%	\norm{z}^2=\inf\{ \norm{ \sum_{k\in \Zred} x_ix_i^*}_{\E^\nabla_\B}\norm{ \sum_{k\in \Zred} y_i^* y_i}_{G(\D)} \mid z=\sum_{k\in \Zred} x_i\tens f_i  \}
	%\end{align*}
	%Thus letting $z\in \E^\nabla_\B \optens_{\B^+} G(\D)$, 
	we have shown for an arbitrary representation $\sum_{k\in \Zred} e_k \tens \begin{pmatrix} f_k \\ \D f_k \end{pmatrix}$ of $z$ that 
	\begin{align*}
		\norm{g\pa{\sum_{k\in \Zred} e_k \tens \begin{pmatrix} f_k \\ \D f_k \end{pmatrix}}}\leq 2\norm{\sum_{k\in \Zred} e_ke_k^*}_{\E^\nabla_{\B^+}}\norm{f_k^*f_k}_{G(\D)}
	\end{align*}
	Thus taking the infimum over representations of $z$, we get:
	\begin{align*}
		\norm{g(z)}&\leq 2\inf\pa{\norm{\sum_{k\in \Zred} e_ke_k^*}_{\E^\nabla_{\B^+}}\norm{f_k^*f_k}_{G(\D)} \mid z=\sum_{k\in \Zred} e_k \tens \begin{pmatrix} f_k \\ \D f_k \end{pmatrix}} \\
		&\leq 2\norm{z}_{\optens}
	\end{align*}
	as desired. 
\end{proof}
As promised the lemma shows well-definedness of $1\tens_\nabla \D$ on the balanced tensor product, if we write out the representation explicitly:
\begin{align*}
	(1\tens_{\nabla} \D)(eb \tens f)&=\sum_{i\in \Zred} x_i \tens \D\pi(\ip{\gamma(x_i)}{eb})f \\
	&=\sum_{i\in \Zred} x_i \tens \D\pi(\ip{\gamma(x_i)}{e}) \pi(b)f \\
	&=(1\tens_{\nabla} \D)(e\tens \pi(b) f)
\end{align*}
The complete boundedness of $g$ will come in useful later, as the operator taking the graph of $\D$ to the graph of $1\tens_\nabla \D$ is now known to be continuous with respect to the operator space norm. 
\begin{lemma}\label{mesrennie315}
	Let $\E_\B$ be a projective operator module with a column finite frame $(x_i)_{i\in \Zred}$ and $R\in \conv\{\chi_n:n\in \N\}$. Then $R$ satisfies:
	\begin{enumerate}
	\item
		$vRv^*:\dom(\D_\epsilon)\to \dom (\part)$.
	\item
		$vRv^*:\dom (\part)^*\to \dom (\D_\epsilon)$. 
	\item
		If $\E_\B$ is complete, $vRv^*: \dom (\part)^* \to \dom (\D_\epsilon) \cap \dom (\part)\subset \dom \part$ 
	\end{enumerate}
\end{lemma}
\begin{proof}
\begin{enumerate}
\item
	It suffices to consider $R=\chi_n$, as then the result follows by linearity. For the first part of the statement, we consider the family of adjointable operators $(H_B\optens_B F)^2\to (vE\optens_B F)^2$.
	\begin{align*}
		\pi_{\D}^p(\chi_k)=\begin{pmatrix} v\chi_k v^* & 0 \\ p[\D_\epsilon,v\chi_k v^*] & v\chi_k v^* \end{pmatrix}
	\end{align*}
	Letting $x=h\tens f \in H_{\B} \tens_{\B} \dom \D\subset \dom (\D_\epsilon)$ we get the following identity 
	\begin{align*}
		v\chi_k v^*(x)=\sum_{1\leq |i| \leq k}  v x_i \tens \ip{x_i}{v^*(h)}f=\sum_{1 \leq |i| \leq k} v x_i \tens \ip{vx_i}{h}f 
	\end{align*}
	As both $v(x_i)$ and $h$ lie in $H_{\B^+}$ we get that the inner product $\ip{vx_i}{h}$ takes values $\B$ as well, thereby allowing us to conclude that $\ip{v x_i}{h}f$ lies in the domain of $\D$. This shows that $v\chi_k v^*(x)$ lies in $v\E\tens_{\B} \dom \D\subset \dom (\part)$. Applying the explicit formula we have derived for $(1\tens_\nabla \D)(e\tens f)$ we arrive at the following expression:
	\begin{align*}
		\pi_\D^p(\chi_k)\begin{pmatrix} x \\ \D_\epsilon x \end{pmatrix}=\begin{pmatrix}v\chi_k v^* x \\ p(\D_\epsilon)v\chi_k v^* x \end{pmatrix} = \begin{pmatrix}v\chi_k v^* x \\  \part v\chi_k v^* x \end{pmatrix}
	\end{align*}
	This implies that $H_\B \tens_\B \dom \D$ contains the direct sum $\bigoplus_{i\in\Zred} \dom \D$ and as such is a core for $\D_\epsilon$. 
	Thereby we have that every element in the family $\pi_{\D}^p(\chi_k)$ maps a dense subspace of $G(\D_\epsilon)$ to $G(\part)$, showing the first part of the lemma. 
	\item
	To show the second part of the lemma, consider $\pi_\D^p(\chi_k)^*$. By what we have just shown, this is a map $G(\part)^\perp \to G(\D_\epsilon)^\perp$. Defining the unitary $U$ as 
	\begin{align*}
		&U:(E_B \optens_{B^+} F_C)^2\to (E_B \optens_{B^+} F_C) \\
		&(x,y)\mapsto (-y,x)
	\end{align*}
	we have the standard equalities $G(\part)^\perp=UG(\part^*)$ and $G(\D_\epsilon)^\perp=UG(\D_\epsilon)$. This allows us to compute the following for every $x\in \dom (\part)^*$. 
	\begin{align*}
		\pi_\D^p(\chi_k)^*\begin{pmatrix} -\part^* x \\ x \end{pmatrix}&=\begin{pmatrix} v\chi_k v^* & -[\D_\epsilon,v\chi_k v^*]p \\ 0 v\chi_k v^* \end{pmatrix}\begin{pmatrix}-\part^* x \\ x \end{pmatrix} \\
		&\begin{pmatrix} -v\chi_k\part^* x -[\D_\epsilon,v\chi_k v^* ]x\\ v\chi_k v^* x \end{pmatrix}
	\end{align*}
	%\todo{why is chi adjoined in the article??-assume mistake}
	Thus $v\chi_k v^*$ lies in the dmoain of $\D_\epsilon$ as long as $x$ is in the domain of $\part^*$. 
	\item
	To show that $vRv^*$ has the claimed properties in case that $\E_\B$ is complete, by Part (2) of the lemma it suffices to show that the range of $vRv^*$ is in $\dom (\part)$. It again suffices to consider $v\chi_n v^*$. By completeness of $\E_\B$ we may pick an approximate unit $(u_n)\subset \conv(\chi_n)$, and let $x\in \dom (\part)^*$. By Part (1) of the lemma, we have $v u_nv^*v\chi v^* x\in \dom (\part)$, and the following norm limit in $H_{\B^+} \optens F_C$. 
	\begin{align*}
		\lim_{n\to\infty} vu_n v^* v\chi_n  v^*x =v\chi_n v^* x
	\end{align*}
	Thus, by applying the second part of the lemma and this limit we get that the operator
	\begin{align*}
		p[\D_\epsilon,vu_kkv^*]p
	\end{align*}
	is well-defined and bounded on the space $v\E \tens_{\B^+} \dom \D$. Thus it extends by continuity to the entirety of $H_{\B+}\optens_{B^+} F_C$. The identity 
	\begin{align*}
		(\part v u_k v^*-vu_k v^* \D_\epsilon)p=p[\D_\epsilon,vu_k v^*]p \quad \for x\in \dom (\part) \cap \dom (\D_\epsilon) \cap (p H_{\B^+}\optens F_C)
	\end{align*}
	together with the continuity of $p[\D_\epsilon,vu_k v^*]p$ gives that $(\part v u_k v^*-vu_k v^* \D_\epsilon)p$ is bounded. 
	As we also have the strict convergences
	\begin{align*}
		p[\D_\epsilon,vu_kv^*]p&\to 0 \\
		vu_kv^* &\to p 
	\end{align*}
	we may deduce that the following limit exists
	\begin{align*}
		\lim_{k\to \infty} \part v u_k v^* v \chi_n v^* x&=\lim_{k\to \infty} vu_k v^*\D_\epsilon \chi_n v^*x+ \part v u_k v^* v \chi_n v^* x-vu_k v^*\D_\epsilon \chi_n v^*x \\
		&=\lim_{k\to\infty} vu_k v^*\D_\epsilon \chi_n v^*x +p[\D_\epsilon,vu_kv^*]pv \chi_n v^* x
	\end{align*}
	By closedness of $\part$, we get that $v\chi_n v^*$ lies in $\dom (\part)$. 
	\end{enumerate}
\end{proof}
This ends our study of connections and projective $\B$-modules as objects in themselves, as we are now sufficiently equipped to use them to show existence of the unbounded Kasparov product.  We proceed with showing that $(1\tens_\nabla \D)$ is a self-adjoint and regular operator and that $\K(\E^\nabla_\B)$ is a differentiable algebra. 
\subsection{Localization and sums of self-adjoint operators}

In this section we return to the local viewpoint of the section on continuous trace algebras, where we view Hilbert $C$-modules as bundles of Hilbert spaces over the state space of $C$, through the formalism of localizations. We start by using the formalism to show that the covariant version of $\D_2$ is actually self-adjoint and regular. Using this result and modifying the techniques of \cite{locglob} to the graded case, we show that the sum of weakly anti-commuting operators is locally self-adjoint, and thereby globally self-adjoint and regular. 

The method which lets us study regular self-adjoint operators through local considerations is the results of \cite{pierre} and \cite{locglob}: An operator is self-adjoint and regular if and only if the operator on each Hilbert localized space is itself a self-adjoint operator. Though the result is stated in the complex case, it readily carries over to the real case by \Cref{complextoreal}.
\begin{definition}
	Given a state $\phi$ on $B$ we can construct the localization $E_B^\phi$ of $E_B$ with respect to this state via. the (pre)-inner product $\phi(\ip{x}{y})$, where we take the quotient by the nullifier of the inner product and complete as usual. 
	Alternatively, given a cyclic representation $\pi$, we can consider $(\pi,H_\pi,\xi_pi)$ and define $E_B^\pi=E_B\tens_\pi H_\pi$. These two definitions are equivalent.  
	Denote the embedding map $\iota_\phi$. 
\end{definition}
\begin{definition}
	Given a regular self-adjoint operator $T$ on a Hilbert $B$-module $E_B$ as well as a representation $\pi$ of $B$ on the Hilbert space $H_\pi$ we define the localization $T_0^\pi$ on $\D(T)\tens_{\pi} H_\pi\subset E_B \tens_{\pi} H_\pi$ via. the formula $T_0^\pi(x\tens h)=Tx\tens h$. The closure of this operator is denoted $T^\pi$ and is called the localization of $T$ with respect to $\pi$.
\end{definition}
\begin{theorem}[The Local-Global Principle]
	A closed densely defined operator $T$ is self-adjoint and regular if and only if all localizations are self-adjoint. 
\end{theorem}
As we are working in the setting of differential algebras rather than Hilbert modules, we need to show that the representation associated with the localization is a morphism in the category of spaces. 
\begin{proposition}
Given an unbounded Kasparov module and a state $\phi$ we get a completely contractive map $\pi_{\D^\phi}:\B\to \lip(\D^\phi)$ by localizing the map $\pi_{\D}$.
\end{proposition}
\begin{proof}
	By definition, $\iota_\phi(\dom \D)$ is a core for $\D^\phi$. For every $b\in \B$ and $f\in \dom \D$ we have $\pi_\phi(b)\iota_\phi(f)=\iota_\phi(bf)\in \iota_\phi(\dom \D)$. Thereby $\pi_\phi(b)$ preserves the core $\iota_\phi(\D)$ for $\D^\phi$. For the commutator we calculate:
	\begin{align*}
		\norm{[\D^\phi,\pi_\phi(b)]\iota_\phi(f)}^2&=\norm{\iota_\phi([\D,b]f)} \\
		&=\phi(\ip{[\D,b]f}{[\D,b]f})\leq \norm{[\D,b]}^2\phi(\ip{f}{f}) \\
		&=\norm{[\D,b]}^2\norm{\iota_\phi(f)}^2
	\end{align*}
	Giving boundedness of the commutator on the core. Therefore $\pi_\phi([\D,b])$ is well-defined and equals $[\D^\phi,\pi_\phi(b)]$, so we write 
	\begin{align*}
		\pi_{\D^\phi}(b)=\pi_\phi(\pi_\D(b))
	\end{align*}
	giving complete contractivity of the map $\pi_\D(b)\mapsto \pi_{\D^\phi}(b)$, showing that it is a morphism in the category of operator spaces. We define $\B^\phi$ as the completion of $\B$ in the norm induced by $\pi_{\D^\phi}$ and as such we may also define the localized module $\E_{\B^\phi}$ over $\B^\phi$ through the mapping $H_{\B^+}\to H_{\B^{+,\phi}}$.
\end{proof}
We proceed to directly apply the results for localizations, showing that all regularity properties pass to localizations, thereby setting the stage for the proof of the self-adjointness and regularity of the covariant  product operator.
\begin{lemma}\label{mesrennie317}
	Let $\E_\B$ be a complete projective operator module for $(\B,F_C,\D)$ with column finite frame $(x_i)_{i\in \Zred}$. Then the localized module $\E_{\B}^\phi$ is a complete projective module for the localized Kasparov module and $(1\tens_\nabla \D^\phi)=(1\tens_\nabla \D)^{\phi}$. Thus we can infer:
	\begin{enumerate}
		\item
			We have the mapping $v\pi_\phi(\chi_n) v^*: \dom(\part^\phi)^*\to \dom \part^\phi$.
		\item
			There is an approximate in $\conv(\chi_n)$ such that $p[\D_\epsilon^\phi, v\pi_\phi(u_n)v^*]p$ converges to $0$ in the $^*$-strong sense on the Hilbert space defined as $H_{B^+}\optens_{B^+} F^\phi$. 
	\end{enumerate}
\end{lemma}
\begin{proof}
	We need to check that the defining frame of $\E_\B$ passes down and the accompanying approximate unit in the convex hull of the canonical approximate unit. The column finiteness of the localized frame is immediate from the previous proposition, since it shows that $\norm{\pi_{\D^\phi}(\ip{x_i}{e})_{i\in \Zred}}\leq \norm{\pi_\D(\ip{x_i}{e})_{i\in \Zred}}$. The sequence $p[\D_\epsilon,vu_nv^*]p$ is uniformly bounded and converges strictly 0, hence the localized sequence does so as well. This shows that $\E_{\B^\phi}$ is a complete projective module for $(\B^\phi,F^\phi,\D^\phi)$. 
	Consider the operator $(1\tens_\nabla \D^\phi)$, which is defined on the core $(\E\tens_{\B^+} \dom \D^\phi)$, while the operator $(1\tens_\nabla \D)^\phi$ is defined on $\iota_\phi(\dom (1\tens_\nabla \D))$.
	Thus we would like to determine a common core for $(1\tens_\nabla \D)^\phi$ and $1\tens_\nabla \D^\phi$. Our candidate is the space: $V=\iota_\phi(\E\tens_\B \dom \D)$. It is clear that this is a core for $(1\tens_\nabla \D)^\phi$, since $\E_\B\tens_\B \dom \D$ is a core for $1\tens_\nabla \D$. In order to show that is is also a core for $1\tens_\nabla \D^\phi$, we unravel the definition of the operator on $e\tens f_k \in \iota_{\phi}(\E_\B\tens_\B \dom \D)$, where $f_k$ is a sequence converging to $f\in \dom \D^\phi$ in graph norm. 
	\begin{align*}
		1\tens_\nabla \D^\phi (e\tens f_k)&=\gamma(e)\tens \D^\phi f_k+\nabla_{\D^\phi}(e)f_k \\
		&=\gamma(e)\tens \D^\phi f_k+\sum_{i\in \Zred}  \gamma(x_i) \tens [\D^\phi,\ip{x_i}{e}]f_k 
	\end{align*}
	The first term will clearly converge to $\gamma(e) \tens \D^\phi f$, by definition of the graph norm. To show convergence of the second term, we apply the norm estimates stemming from the Haagerup norm to show that it is a Cauchy sequence. We have implicitly used that the localization map is contractive to perform these estimates. 
	\begin{align*}
		\norm{\sum_{i\in \Zred}  \gamma(x_i) \tens [\D^\phi,\ip{x_i}{e}](f_k-f_l)}_{\optens}&\leq \norm{\sum_{i\in \Zred} \ket{x_i}\bra{x_i}}_{\K(E_B)}\norm{[\D^\phi,\ip{x_i}{e}](f_k-f_l)} \\
		&\leq \norm{[\D,\ip{x_i}{e}]}^2\norm{(f_k-f_l)}^2
	\end{align*}
	The norm of the first term is finite, as $e$ is assumed to lie in $\E_\B$. 
	This tells us that we may approximate any element $y\in \E\optens_\B \dom \D^\phi$ by elements of $V$ in graph norm of $1\tens_\nabla \D^\phi$. This implies that the closure of $(1\tens_\nabla \D)^\phi$ over $V$ contains the defining domain of $1\tens_\nabla \D^\phi$, showing that $V$ is a common core for the  operators. Since they coincide on the core, we may infer that 
	\begin{align*}
		1\tens_\nabla \D^\phi=(1\tens_\nabla \D)^\phi
	\end{align*}
	as desired.  To see the first claim, simply apply \Cref{mesrennie315} to the defining frame of localized module $\E_{\B^\phi}$. 
\end{proof}
\begin{theorem}\label{mesrennie318}
	Letting $\E_\B$ be a complete projective operator module for $(\B,F_C,\D)$. Then $(1\tens_\nabla \D)$ is self-adjoint and regular. 
\end{theorem}
\begin{proof}
	We have reduced our problem to showing that for every state $\phi$ on $C$, the operator $\part^\phi$ is self-adjoint and regular on the Hilbert space $(E_B\tens_B F_C)^\phi\cong E_B\tens_B F_C^\phi$. Letting $(u_n)_{n\in \N}\subset \conv\{\chi_n: n\in \N \}$ be an approximate unit for the complete projective operator module associated to $(\B,F_C,\D)$, then by \Cref{mesrennie317}: 
	\begin{align*}
		&v\pi_\phi(u_n)v^*:\dom(\part^\phi)^*\to \dom \part^\phi \\
		&p[\D^\phi_\epsilon,v\pi_\phi(u_n)v^*]p\to 0,\quad ^*\text{ strongly on }H_{\B^+}\optens_{B^+}F^\phi. 
	\end{align*}
	We may also apply \Cref{mesrennie314} to conclude that $\part^\phi x=p\D^\phi_\epsilon px$ on the core $H_{\B^+}\optens_{\B^+} \dom \D^\phi$ of $\part^\phi$. Drawing these two results together, let $x\in H_{\B^+}\optens_{\B^+} \dom \D^\phi$:
	\begin{align*}
		[\part^\phi,vu_kv^*]x &=\part^\phi vu_kv^*x-vu_kv^*\part^\phi x \\
		&=p\D^\phi_\epsilon pvu_kv^* x-vu_kv^*p\D^\phi_\epsilon px \\
		&=p[\D^\phi_\epsilon,vu_kv^*]px\to 0
	\end{align*}
	which converges to zero in norm. By the uniform boundedness of $p[\D^\phi_\epsilon,vu_kv^*]p$ the result may be expanded to the entirety of $H_{\B}\optens_{\B} F_C^\phi$. Since the closure of $[\part^\phi,vu_k v^*]$ is equal to the closure of $[(\part^\phi)^*,vu_kv^*]$, it may be inferred that the latter converges strictly to zero on $H_{\B^+}\optens_{\B^+} F_C$. As such, for $y\in \dom (\part^\phi)^*$, we get $vu_kv^* y\in \dom \part^\phi$ through application of \Cref{mesrennie317}, and $vu_kv^* y\to y$. We conclude that $\part^\phi vu_k v^* y$ is convergent to $\part^\phi y$ as below since
	\begin{align*}
		(\part^\phi)^*y&=\lim_{k\to\infty} vu_kv^* (\part^\phi)^*y=\lim_{k\to \infty} \part^\phi vu_k v^* y-[(\part^\phi)^*,vu_k v^*]y \\
		&=\lim_{k\to \infty} \part^\phi vu_k v^* y
	\end{align*}
	Thus $\dom \part^\phi$ is a core for the operator $(\part^\phi)^*$. This lets us conclude that $\part^\phi$ is self-adjoint since it is closed and symmetric, and thus, by the Local-Global principle, it is self-adjoint and regular, and finally by \Cref{mesrennie313}, that $1\tens_\nabla \D$ is self-adjoint and regular. 
\end{proof}

\begin{definition}
	The algebra of adjointable operators on $\E^\nabla$ is the idealiser of $\pi_\nabla(\K(\E^\nabla))$ inside the algebra $L_C((pH_{\B^+}\optens_{B^+} F)^2)$. We shall denote it by $L_\B(\E^\nabla)$. 
\end{definition}
To see that our definitions for differentiable modules interact in the same fashion as for Hilbert modules, we have the following proposition.
\begin{proposition}\label{differentiablealgebra}
	If $\E_\B$ is a complete projective module then $L_\B(\E^\nabla)$ is an operator $^*$-algebra, which is isometrically isomorphic to $M(\K(\E^\nabla))$. This algebra coincides with a closed subalgebra of the Lipschitz algebra of $(1\tens_\nabla \D)$, and as such $\K(\E^\nabla)$ is a differentiable algebra. 
\end{proposition}
\begin{proof}
	The operator $1\tens_\nabla \D$ is self-adjoint and regular, so for finite rank $K$ we have the following equality
	\begin{align*}
		[1\tens_\nabla \D,vKv^*]=p[\D_\epsilon,vKv^*]p
	\end{align*}
	Let $T\in L_{\B}(\E^\nabla)$ then there is a sequence $T_n$ satisfying that $T_nK$ is finite rank for $K$ finite rank, and that $T_n K$ and $p[\D_\epsilon,vT_nKv^*]p$ are both convergent with $T_nK$ to $TK$. By definition of $\part$ we have the equality
	\begin{align*}
		p[\D_\epsilon,vT_nKv^*]p=[\part,vT_nKv^*]
	\end{align*}
	Giving that 
	\begin{align*}
		\part(vT_nKv^*x)=[\part,vT_nKv^*]+vT_nKv^*x 
	\end{align*}
	is convergent for every $x\in \dom \part$. Hence $TK$ preserves the domain of $\part$ for every $K$, and $vFin_B(\E_\B)v^*\dom \part$ is dense in $p\dom \part$ in graph norm. It follows that $T$ preserves the core of $\part$ and on this core the operator
	\begin{align*}
		[\part,vTv^*]vKv^*x=[\part, vTKv^*]x-v\gamma(T)v^*[\part,vKv^*]x
	\end{align*}
	is bounded. This implies that $vTv^*$ is in $\lip(\part)$, or equivalently in $T\in \lip(1\tens_\nabla \D)$ as desired. The remaining part may be shown as in \Cref{mesrennie117}.
\end{proof}
%\todo{import the rest that is needed from section 3, about 7 pages worth!}
\begin{definition}
	Let $E$ be a graded complex \Cstar module. Assume two odd self-adjoint regular operators $S$ and $T$ on $E$ satisfy 
	\begin{enumerate}
	\item
		There is a core $X$ for $T$ such that $(S\pm i\lambda)^{-1}X\subset \dom T$. 
	\item
		We have the inclusions $T(S\pm i\lambda)^{-1}X\subset \dom S$. 
	\item
		The operator $[S,T](S\pm i\lambda)^{-1}$ is bounded on $X$. 
	\end{enumerate}
	for all $\lambda>0$. Then $S$ and $T$ weakly anti-commute.
\end{definition}
We have the following result 
\begin{lemma}\label{boundedness}
	If $S$ and $T$ weakly anticommute then $(S\pm \lambda i)^{-1}$ preserves the domain of $T$, and the commutator $[S,T](S\pm \lambda i)^{-1}$ is bounded on $\dom T$. Hence: 
	\begin{align*}
		&S((S-\lambda i)^{-1}\dom T)\subset \dom T \\ 
		&T(\im(S-\lambda i)\dom T)\subset \dom s
	\end{align*}
	Thus the commutator is defined on $\im (S\pm \lambda i)^{-1}(T\pm ic)^{-1}$.  
\end{lemma}
\begin{proof}
	We may expand the commutator: 
	\begin{align*}
		[T,(S+i\lambda)^{-1}]x=(S\mp \lambda i)^{-1}[S,T](S\pm i\lambda)^{-1}x
	\end{align*}
	This operator is bounded by definition which implies that $(S\pm i\lambda)^{-1}$ preserves the domain of $T$ by \cite[Proposition 2.1]{forsyth}. Given $x\in \dom T$ we may pick a sequence $(x_n)_{n\in \N}\subset X$ converging to $x$, satisfying $Tx_n\to Tx$ as $X$ is a core for $T$. We then calculate:
	\begin{align*}
		T(S-i\lambda)x_n=-(S+i\lambda)^{-1}Tx_n+(S+i\lambda)^{-1}[S,T](S-i\lambda)^{-1}x_n \\
	\end{align*}
	As the resolvent preserves the domain, we may take the limit of the above sequence to get $T(S-i\lambda)x=-(S+i\lambda)^{-1}Tx+(S+i\lambda)^{-1}[S,T](S-i\lambda)^{-1}x $.
	As this sequence is convergent, the limit is $T(S-i\lambda)^{-1}x\in \dom S$. In order to see that we have the inclusion $S((S-i\lambda)^{-1}\dom T)\subset \dom T$ consider the equality 
	\begin{align*}
		S(S-i\lambda)^{-1}(T-i\mu)^{-1}=(T-i\mu)^{-1}+i\lambda(S-i\lambda)^{-1}(T-i\mu)^{-1}
	\end{align*}
\end{proof}
%We have reached the point where we define the notion of an unbounded correspondence, which is the appropriate setting to put the construction of the unbounded Kasparov in. 
We can draw all our work together into a unified framework into which we place the construction of the unbounded Kasparov product, namely the unbounded correspondence. 
\begin{definition}
	Given an unbounded Kasparov module $(\B,F_C,\D_2)$ with bounded approximate unit for $\B$, an $\A-\B$ correspondence for $(\B,F_C,\D_2)$ is a quadruple $(\A,\E_\B,\D_1,\nabla)$ such that:
	\begin{enumerate}
		\item
			The module $\E_\B$ is a projective $\B$ operator module. 
		\item
			The algebra $\A$ is a $^*$-algebra satisfying $\A\subset L_{\B}(\E^\nabla_\B)\cap \lip(\D_1)$. 
		\item
			The operator $\D_1$ is self-adjoint regular operator satisfying that $(\D_1^2+1)^{-1/2}\in L_{\B}(\E^\nabla_\B)$ with $a(\D_1^2+1)^{-1/2}\in \K_\B(\E^\nabla_\B)$ for all $a\in \A$. 
		\item
			The operator $\nabla:\E^\nabla_\B\to E\optens_B \Omega^1_\D$ is a connection  such that $\nabla((\D_1^2+1)^{-1/2}\epsilon)\subset \dom (\D_1 \tens 1)$,and the operator $[\nabla, \D_1](\D_1^2+1)^{-1/2}:\E_\B^\nabla \to E_B\optens_{B^+} \Omega^1_{\D_2}$ is completely bounded. 
	\end{enumerate}
	such a correspondence is strongly complete if there exists an approximate unit for $A$ such that $[\D_1,u_n]\to 0$ and $[1\tens_\nabla \D_2,u_n\tens Id_{F_C}]\to 0$ in norm.
\end{definition}
To see that the notion of an unbounded correspondence is the correct one, we show that $D_1=\D_1\tens 1$ and $D_2=1\tens_\nabla D_2$  weakly anticommute. One should remark that weakly anticommuting is defined only on the complexifications, as the property is only required to prove self-adjointness and regularity of the sum operator. 
\begin{lemma}\label{mesrennie42}
	Let $(\A,\E_\B,\D_1,\nabla)$ be an $\A-\B$ correspondence. Then the self-adjoint regular operators $D_1=\D_1\tens 1$ and $D_2=1\tens_\nabla \D_2$ weakly anticommute, with core $X=\E_\B^\nabla \optens_{\B} \D_2$. 
\end{lemma}
\begin{proof}
	By \Cref{mesrennie314} the map $g:\E_\B^\nabla\optens_{\B^+} G(\D_2)\to G(1\tens_\nabla \D_2)$ given as 
	\begin{align*}
		e\tens \begin{pmatrix} f \\ \D_2 f\end{pmatrix} \mapsto \begin{pmatrix} e\tens f  \\ 1\tens_\nabla \D_2(e\tens f) \end{pmatrix}
	\end{align*}
	has dense range. Thus $X=\E_\B^\nabla \tens_\B \dom \D_2$ is a core for $D_2$. As \begin{align*} (\D_1^2+1)^{-1/2}: \E_\B^\nabla \to \E_\B^\nabla, \end{align*} so will $(D_1\pm i)^{-1}$ and thereby we get the first part of the criterion for being weakly anticommuting. 
	To check the second part, remark that by definition \begin{align*} D_1((D_2+i)^{-1}(D_2-i)^{-1})^{1/2}X\subset \dom D_2, \end{align*} and thus $D_1(D_2\pm i)^{-1} X\subset \dom D_2$. On $X$ we may calculate the graded commutator 
	\begin{align*}
		[D_1,(D_2\pm i)^{-1}]=(D_2\mp i)^{-1}[D_2,D_1](D_2\pm i)^{-1}
	\end{align*}
	which can be seen to be bounded from: 
	\begin{align*}
		[D_2,D_1](e\tens f)&=(D_1\tens 1)(\gamma(e)\tens \D_2 f+\nabla(e)f)+\gamma(D_1e)\tens \D_2 f+\gamma(D_1e)\tens f+\nabla(D_2 e)\tens f\\
		&=[\nabla,D_2](e)f
	\end{align*}
	By definition $[\nabla,D_1](D_1\pm i)^{-1}$ is bounded, finishing the claim. 
\end{proof}
We can now show local compactness of the resolvents. 
\begin{lemma}\label{mesrennie43}
	Assume we are in the setup of \Cref{mesrennie42}, then the operators $(D_2^2+1)^{-1/2}K$, and $K(D_2^2+1)^{-1/2}$ are compact for $K\in \K(E_B)$. 
\end{lemma}
\begin{proof}
	For every $e\in \E$ the following series is norm-convergent.  
	\begin{align*}
		\sum_{i \in \Zred}[\D_{2,\epsilon},\ip{x_i}{e}]^*[\D_{2,\epsilon},\ip{x_i}{e}]
	\end{align*}
	Thus we may conclude that the operator given as $D_2\ket{e}-\ket{\gamma(e)}\D_2$ acting on $f$ as:
	\begin{align*}
		\sum_{i\in \Zred} x_i \tens \D_2 \ip{x_i}{e}-\gamma(e)\tens \D_2 f&=\sum_{i\in \Zred} x_i\tens (\D_{2,\epsilon}\ip{x_i}{e}-\ip{x_i}{\gamma(e)}\D_{2,\epsilon})f \\
		&=\sum_{i\in\Zred} x_i \tens[\D_{2,\epsilon},\ip{x_i}{e}]f
	\end{align*}
	is bounded. 
	Let $(u_n)_{n\in \N}$ be an approximate unit for $B$ and consider the operator $\ket{e}(\D_2^2+1)^{-1/2}:F_C\to E_B\optens_B F_C$. We have the equality:
	\begin{align*}
		\ket{e}(\D_2^2+1)^{-1/2}&=\lim_{n\to\infty} \ket{e u_n}(\D_2^2+1)^{-1/2} \\
		&=\lim_{n\to \infty}\ket{e}u_n(\D_2^2+1)^{-1/2}
	\end{align*}
	Giving that our operator is the norm limit of compact operators, and thus is compact by the assumption of locally compact resolvent of $\D_2$. To expand this to all compacts, consider the calculation below
	\begin{align*}
		&((1\tens_\nabla \D_2)\pm i)^{-1} \ket{e_1}\bra{e_2} \\
		&=\ket{\gamma(e_1)}(\D_2)\pm i)^{-1} \bra{e_2} \\
		&+((1\tens_\nabla \D_2)\pm i)^{-1}(\ket{e_1}\D_2-(1\tens_\nabla \D_2)\ket{\gamma(e_1)}\pm i\ket{e_1-\gamma(e_1)})(\D_2\pm i)^{-1} \ket{e_2}
	\end{align*}
Thus the operator $(1\tens_\nabla \D_2 \pm i)^{-1}K$ is compact for every $K\in \K(E)\tens 1$, so we have shown the desired by \Cref{complextoreal}. 
\end{proof}
After this brief aside demonstrating the consequences of the existence of an unbounded correspondence for whether two unbounded self-adjoint operators weakly anti-commute, we proceed by showing the applications of weak anti-commutativity. In particular, we shall show that if two self-adjoint regular operators weakly anti-commute, their sum is self-adjoint and regular. 
\begin{lemma}\label{strongconvsa}
	Let $W$ be a self-adjoint regular operator on $E$. Let $(f_n)_{n\in\N}$ be a uniformly bounded sequence of functions, converging uniformly on compact subsets of $\R$ to $f\in C_0(\R)$. 
	Then $f_n(W)$ converges strongly to $f(W)$. 
\end{lemma}
\begin{proof}
	Let $x\in \D(W)$. Define $\phi(t)=(t+i)^{-1}$, and $y=(W+i)x$. This gives the identity $x=\phi(W)y$, as the regularity of $W$ implies that the function is well-defined. As $\phi$ vanishes at infinity, $f_n\phi\to f$ uniformly, not just on compact subsets. Thus
	\begin{align*}
		f_n(W)x=((f_n\phi)(W))y\to f(W)x
	\end{align*}
	Therefore $f_n(W)$ converges strongly to $f(W)$ on a dense subset of $E$. As the sequence is uniformly bounded, we get strong convergence globally. 
\end{proof}
Define the family of operators $Y_\mu=[S,T](S-i\mu)^{-1}$, for $\mu \in \R$. This family is adjointable as the domain of $Y_\mu$ includes the dense submodule $(S-i)^{-1}(E)$. Thus we get
\begin{align*}
	Y_\mu^*\xi=-(S+i\mu)^{-1}[S,T]\xi
\end{align*}
for every $\xi\in \dom([S,T])$. 
\begin{lemma}\label{convergestozero1}
	The sequence of operators 
	\begin{align*}
		R_n=-\frac{i}{n}\pa{\frac{-i}{n}S+1}^{-1}[S,T]\pa{\frac{i}{n}S-1}^{-1}
	\end{align*}
	converges strongly to the zero operator. Furthermore, $R_n$ is adjointable and the adjoint converges to the zero operator as well.
\end{lemma}
\begin{proof}
	Write $R_n$ in the form
	\begin{align*}
		R_n=\pa{\frac{i}{n}S+1}^{-1}Y_{-1}(S-i)(S-in)^{-1} 
	\end{align*}
	Every factor is adjointable, so $R_n$ is adjointable. By \Cref{strongconvsa} $(S-i)(S-in)^{-1}$ converges strongly to 0. Further, $\pa{\frac{i}{n}S+1}^{-1}X_{-1}$ is a uniformly bounded sequence. This proves the desired, as the proof for the adjoint proceeds in an analogous fashion. 
\end{proof}
In order to control the sum of $S$ and $T$ it is necessary to control their commutators. 
\begin{lemma}\label{estimatelemma1}
	Let $S$ and $T$ be self-adjoint regular weakly anti-commuting operators, then there exists a $C>0$ such that:
	\begin{align*}
		\ip{[S,T]\xi}{\xi}\leq \frac{1}{2}\ip{S\xi}{S\xi}+C\ip{\xi}{\xi}
	\end{align*}
\end{lemma}
\begin{proof}
	Taking $S$ and $T$'s self-adjointness into account and the assumption that they weakly anticommute, we see that the form $\ip{[S,T]\xi}{\xi}$ is self-adjoint for every $\xi \in \dom([S,T])$. Then by 
	\begin{align*}
		2\ip{[S,T]\xi}{\xi}&=-(\ip{[S,T]\mu\xi}{\mu^{-1}\xi}+\ip{\mu^{-1}\xi}{[S,T]\mu\xi} \\
		&\leq \mu^2 \ip{[S,T]\xi}{[S,T]\xi}+\mu^{-2}\ip{\xi}{\xi} \\
		&\leq \mu^2 ||[S,T](S+\mu i)^{-1}||^2\ip{(S+\mu i)\xi}{(S+\mu i)\xi}-\mu^{-2}\ip{\xi}{\xi} \\
		&\leq \mu^2||X_{-1}||^2 \ip{S\xi}{S\xi}+(\mu^2(||X_{-1}+\mu^{-2}||)\ip{\xi}{\xi}
	\end{align*}
	we see that choosing $\mu=\frac{1}{\norm{Y_{-1}}}$ shows the desired.
\end{proof}
After having achieved control over the commutator of self-adjoint regular weakly anti-commuting operators, we can show we can control the graph norm of the sum via. the sum of the graph norms. 
\begin{lemma}\label{estimatelemma2}
	Let $S$ and $T$ be self-adjoint regular weakly anti-commuting operators. There is a constant $C>0$ such that for every $\xi \in \dom(S)\cap \dom(T)$. 
	\begin{align*}
		\ip{(S+T)\xi}{(S+T)\xi}\geq \frac{1}{2}||S\xi||^2+||T\xi||^2-C||\xi||^2
	\end{align*}
\end{lemma}
\begin{proof}
	We start by showing the inequality on $\dom([S,T])$. Here we can apply \Cref{estimatelemma1} to get: 
	\begin{align*}
		\ip{(S+T)\xi}{(S+T)\xi}&=\ip{S\xi}{S\xi}+\ip{T\xi}{T\xi}+\ip{[S,T]\xi}{\xi} \\
		&\geq \frac{1}{2} ||S\xi||^2+||T\xi||^2-C||\xi||^2
	\end{align*}
	Consider $\xi\in \dom(S)\cap \dom(T)$. Define the sequence
	\begin{align*}
		\xi_n=\pa{\frac{i}{n}S+1}^{-1}\xi \in \dom([S,T])
	\end{align*}
	We have the convergence $\xi_n\to \xi$ and $S\xi_n \to S\xi$ by \Cref{strongconvsa}. Therefore the only thing left to show is that $T\xi_n\to T\xi$. We calculate:
	\begin{align*}
		T\xi_n&=T\pa{\frac{i}{n}S+1}^{-1}\xi \\
		&=\pa{\frac{i}{n}S+1}^{-1}T\xi+\frac{i}{n}\pa{\frac{i}{n}S+1}^{-1}[S,T]\pa{\frac{i}{n}S-1}^{-1}\xi \\
		&=\pa{\frac{i}{n}S+1}^{-1}T\xi+R_n\xi
	\end{align*}
	By the result in \Cref{convergestozero1} this implies that $T\xi_n\to T\xi$ as desired. 
\end{proof}
We can now show the result that the sum of weakly anticommuting self-adjoint regular operators is again self-adjoint. 
\begin{theorem}\label{sumselfadjoint}
	Assume that the self-adjoint regular operators $S$ and $T$ weakly anti-commute. Then the operator $S+T$ with $\dom(S+T)=\dom(S)\cap \dom(T)$ is self-adjoint. Further, $S+T$ has core $\im ((S\pm i)^{-1}\dom T)=\im((S\pm \lambda i)^{-1}(T\pm i)^{-1})$. 
\end{theorem}
\begin{proof}
	We shall proceed by first showing that $S+T$ is closed and symmetric, using this to show self-adjointness. 
	By \Cref{estimatelemma2} we get that the convergence of a sequence in the graph norm of $S+T$ is equivalent to the sequence converging in the graph norms stemming from both operators. Therefore closedness of $S$ and $T$ gives closedness of $S+T$. In order to show that $(S+T)^*=S+T$, we need only consider one inclusion since symmetry is immediate, namely
	\begin{align*}
		\dom((S+T)^*)\subset \dom(S)\cap \dom(T)
	\end{align*}
	To do this, consider an element $\xi \in \D((S+T)^*)$. Consider the sequence
	\begin{align*}
		\xi_n=\pa{-\frac{i}{n}S+1}^{-1}\xi
	\end{align*}
	which will lie in $\dom(S)$, and is norm-convergent to $\xi$. As all operators concerned are closed, we need only show that $\xi_n \in \dom(S)\cap \dom(T)$ and $(S+T)^*\xi_n=(S+T)\xi_n$ is convergent.
	
	To see $\xi_n\in \dom(S)\cap \dom(T)$, let $\eta\in \dom(T)\cap \dom(S)$. Then we can perform the following calculations
	\begin{align*}
		\ip{\xi_n}{T\eta}&=\ip{\pa{\frac{-i}{n}S+1}^{-1}\xi}{T\eta} \\
		&=\ip{\xi}{T\pa{\frac{i}{n}S+1}^{-1}\eta} +\ip{\xi}{\frac{i}{n}\pa{\frac{i}{n}S+1}^{-1}[S,T]\pa{\frac{i}{n}S-1}^{-1}\eta} \\ 
		&=\ip{\xi}{(S+T)\pa{\frac{i}{n}S+1}^{-1}\eta}+\ip{\xi}{S\pa{\frac{i}{n}S+1}^{-1}\eta}+\ip{R_n^*\xi}{\eta} \\
		&=\ip{\pa{\frac{i}{n}S+1}^{-1}(S+T)^*\xi}{\eta}+\ip{S\xi_n}{\eta}+\ip{R_n^*\xi}{\eta}
	\end{align*}
	By self-adjointness of $T$ we may now conclude $\xi\in \dom(T)$, as well as 
	\begin{align*}
		T\xi_n=\pa{-\frac{i}{n}S+1}^{-1}(S+T)^*\xi-S\xi_n+R_n^*\xi
	\end{align*}
	Now we need only show that $(S+T)\xi_n$ converges in $E$. By the expression for $T\xi_n$ we can see that
	\begin{align*}
		(S+T)\xi_n=\pa{\frac{-i}{n}S+1}^{-1}(S+T)\xi+iR_n^*\xi
	\end{align*}
	which converges in $E$ as $R_n^*$ converges strongly to 0. Thus $((S+T)^*\xi_n)_{n\in \N}$ converges, and we have shown the desired result on $\dom(S+T)$ and also that the domain has the claimed core. 
\end{proof}
We now take a brief detour back to localizations, in order to show that the localization procedure preserves weak anticommutativity and sums. We first state a lemma without proof.
\begin{lemma}\label{locanticom}
	Let $S$ and $T$ be self-adjoint regular weakly anti-commuting operators. For any state $\phi$ the operators $S^\phi$ and $T^\phi$ weakly anti-commute, with core $\E^\phi=\iota_\phi(\dom(T))$. 
\end{lemma}
\begin{comment}
\begin{proof}
	The result follows by the since $\pi_{\omega}$ is completely contractive and a homomorphism. Thereby the localization of a shared core is again a shared core, as well as preserving domain inclusions. 
\end{proof}
\end{comment}
The only remaining hurdle is showing that the process of localization is finitely additive. 
\begin{lemma}\label{sumloc}
	The localization of a sum is equal to the sum of the localizations. That is, 
	\begin{align*}
		S^\phi+T^\phi=(S+T)^\phi
	\end{align*}
	for any state $\phi$. 
\end{lemma}
\begin{proof}
	By definition we have $\D(S^\phi+T^\phi)=\D(S^\phi)\cap \D(T^\phi)$. The inclusion $(S+T)^\phi\subset S^\phi+T^\phi$  is immediate as $(S+T)_0^\phi\subset S_0^\phi+T_0^\phi$, and $S^\phi+T^\phi$ is closed. To show the other inclusion, we start by showing that 
\begin{align*}
	(S^\phi+i\mu)^{-1}(\D(T^\phi))\subset \D((S+T)^\phi)
\end{align*}
	Pick $\xi \in \D(T^\phi)$. Then there is a sequence $\eta_n\subset \dom(T)$ such that $\iota_\phi(\eta_n)$ converges to $\xi$ and $\iota_{\phi}(T\eta_n)$ converges to $T^\phi(\xi)$. Applying our assumption that $S$ and $T$ weakly anticommute we get the following
	\begin{align*}
		(S^\phi+i\mu)^{-1}(\iota_\phi(\eta_n))=\iota_\phi((S+i\mu)^{-1}\eta_n)\in \iota_\phi(\dom(S)\cap \dom(T))\subset \D((S+iT)^\phi)
	\end{align*}
	By continuity we may infer that $(S^\phi+i\mu)(\iota_\phi(\eta_n))$ converges to $(S^\phi+i\mu)^{-1}\xi$. Thus it suffices to show that $(S+T)^\phi(S^\phi+i\mu)^{-1}(\eta_n)$ converges in the norm of $E$, which may be done with an argument analogous to the one in \Cref{sumselfadjoint}. To finalize the proof of the inclusion, let $\xi\in \D(S^\phi+T^\phi)$. Define the sequence $\xi_n=\pa{\frac{i}{n}S^\phi+1}^{-1}\xi$. As in \Cref{estimatelemma2} we have that $\xi_n$ converges to $\xi$, and $(S^\phi+T^\phi)\xi_n$ converges to $(S^\phi+T^\phi)\xi$ in $E^\phi$. This shows the desired, as $\xi_n\in \D((S+T)^{\phi})$. 
\end{proof}
Drawing all our work together, we have finally reached the point where we can show that the sum of regular self-adjoint normal operators is once again self-adjoint, thereby the last piece of the puzzle showing the existence of the unbounded Kasparov product. 
\begin{theorem}\label{locglob71}
	Let $S$ and $T$ be weakly anticommuting self-adjoint regular operators. Then the sum operator 
	\begin{align*}
		Z=S+T
	\end{align*}
is self-adjoint and regular.
\end{theorem} 	
\begin{proof}
	Let $\phi$ be a state on $B$. By the Local-Global principle it suffices to show that $Z^\phi$ is self-adjoint. By \Cref{sumloc} we get 
	\begin{align*}
		Z^\phi=S^\phi+T^\phi
	\end{align*}
	Thus $(Z^\phi)^*=(S^\phi+T^\phi)^*$. By \Cref{locanticom} $S^\phi$ and $T^\phi$ weakly anti-commute and by \Cref{sumselfadjoint} we get that $S^\phi+T^\phi$ is again self-adjoint. Thus 
	\begin{align*}
		(Z^\phi)^*=(S^\phi+T^\phi)^*=(S+T)^\phi=S^\phi+T^\phi
	\end{align*}
	which is exactly $Z^\phi$, as desired. 
\end{proof}
\todo{der er noget rod med nogle kommutatorer hist og her, tilfoej metatekst}
\subsection{The Kasparov Product}
We may now show that we have constructed an unbounded version of the Kasparov product, applicable in both real and complex cases. 
We start by stating a result by Kucerovsky, \cite[Theorem 13]{kucerovsky}. 
\begin{theorem}[Kucerovsky's criterion]\label{kucerovskycrit}
	Assume that we have unbounded Kasparov modules $x=(E_B,\pi_1,\D_1)\in \Psi(A,B)$  and $y=(E_C,\pi_2,\D_2)\in \Psi(B,C)$. 			Let $W\subset \pi(\A)E_B$ be a dense subset. For every $e\in W$ define the operator:
	\begin{align*}
		&T_e:E_B\to E_B\tens_B E_C \\
		&f\mapsto e\tens f
	\end{align*}
	The module $z=(E_B\tens_B E_C,\pi_1\tens 1,D)\in \Psi(B,C)$ represents the Kasparov product of $x$ and $y$ if the following conditions are satisfied:
	\begin{enumerate}
		\item
			For $e\in W$ the commutator
			\begin{align*}
			\begin{bmatrix}
				\begin{pmatrix} D & 0 \\ 0 & \D_2 \end{pmatrix}, \begin{pmatrix} 0 & T_e \\ T_e^* & 0\end{pmatrix}	
			\end{bmatrix}
			\end{align*}
			 extends to a bounded operator. 
		\item
			$\dom(D)\subset \dom(D_1\tens 1)$. 
		\item
			There exists $C\in \R$ such that
			\begin{align*}
				\ip{(\D_1\tens 1) x}{D x}+\ip{D x}{(\D_1\tens 1)x}\geq C\ip{x}{x}
			\end{align*}
			for every $x\in \dom(D)$.
\end{enumerate}
\end{theorem}
%As a first application of the unbounded interior Kasparov product, we start by reading off the Bott cycles relating the suspension and Clifford algebras from \cite{kasparov}. 
Bringing together the results we have shown on connections, localization and differential algebras we can show that we have constructed the unbounded Kasparov module representing the product of two composable cycles. The proof is a mixture of the proof given in \cite{unboundkasp} and \cite{mesrennie}, where we have chosen to forego the spectral approach of \cite{mesrennie} and instead work with the localizations of \cite{unboundkasp} in order to make the results more self-contained. 
\begin{theorem}
	Let $(\B,F_C,\D_2)$ be an unbounded Kasparov $B-C$ module and let $(\A,\E_\B,\D_1,\nabla)$ be an $\A-\B$ correspondence for $(\B,F_C,\D_2)$. Then the module $(\A,(E_B\optens_{B^+} F_C,\D_1\tens 1+1\tens_\nabla \D_2)$ is an an unbounded Kasparov module representing the product of $(\A,E_B,\D_1)$ and $(\B,F_C,\D_1)$. 
\end{theorem}
\begin{proof}
	The proof is broken up into two stages. We start by showing that we have an unbounded Kasparov module, and then we check the requirements for Kucerovsky's criterion. 
	By \Cref{mesrennie318} the operators $\D_1\tens 1$ and $(1\tens_\nabla \D_2)$ are both self-adjoint and regular. Appealing to \Cref{mesrennie42} we get that they weakly anti-commute, and by \Cref{locglob71} we thus get that the sum is self-adjoint and regular. 
	
	In order to see that our operator has locally compact resolvent, we follow the method of \cite{unboundkasp}. 
	%By \Cref{differentiablealgebra} $\K(\E_\B^\nabla)$ is a differentiable algebra, so there exists an increasing commutative approximate unit $(u_n)_{n\in \N}$ for $K(\E_\B^\nabla)$. 
	Let $(u_n)_{n\in \N}$ be an approximate unit for $\K(E_B)$. 
	
	Consider the commutative diagram of operator modules, where $\iota_*$ denotes the inclusion operator, as implemented by multiplication with the resolvent.  
	\begin{align*}
		\xymatrix{
			\dom(\D_1\tens 1 +1\tens_{\nabla} \D_2) \ar[r]^{\iota_1} \ar[rd]^{\iota} \ar[d]^{\iota_3} & \dom(1\tens_{\nabla} \D_2) \ar[d]^{\iota_2} \\
			\dom(\D_1\tens 1) \ar[r]^{\iota_4} & (E_{B}\optens_B F_C)
		}
	\end{align*}
	We need to show that $\pi(a)\circ \iota: \dom(\D_1\tens 1 +1\tens_{\nabla} \D_2)$ is compact, where $\pi(a)$ acts by multiplication on the first component of the tensor product. 
	By commutativity of the diagram we get 
	\begin{align*}
		(u_n \tens 1)\circ \pi(a)\circ \iota=(u_n\tens 1)\circ \pi(a) \circ \iota_2\circ \iota_1
	\end{align*}
	The operator $(u_m\tens 1)\circ \pi(a) \circ \iota_1\circ \iota_2$ is compact by \Cref{mesrennie43}, implying that
	\begin{align*}
	(u_m \tens 1)\circ \pi(a)\circ \iota:\E^\nabla_\B \optens_{\B^+} \dom \D_2 \to E_B\optens_{B^+} F_C 
	\end{align*}
	is compact for all $n$. 
	Going back to the diagram, we have the identity 
	\begin{align*}
		&(u_m\tens 1)\circ \pi(a)\circ \iota=(u_m\tens 1)\circ \pi(a) \circ \iota_4\circ \iota_3.
	\end{align*}
	when viewed as operators 
	\begin{align*}
		\E^\nabla_\B\optens_{\B^+} \dom \D_2\to E_B\optens_{B^+} F_C
	\end{align*}
	As we are working over a correspondence, we have $(\pi(a)\circ \iota_4)=\pi(a)\circ (\D_1^2\tens 1+1)^{-1/2}=K\tens 1$ where $K\in \K(\E^\nabla_\B)\subset \K(E_B)$. 
	Hence the sequence $((u_m\tens 1)\circ \pi(a)\circ \iota_4)_{m\in \N}\subset L(\E^\nabla_\B \optens_{\B^+} \dom \D_2,E_B\optens_{B^+} F_C)$ converges to the bounded operator $(\pi(a)\circ \iota_4)\in L(\dom(\D_1\tens 1+1\tens_\nabla \D_2),E_B\tens_B F_C)$ in operator norm. This shows that $\pi(a)\circ \iota_4$ is compact, hence $\pi(a)\circ \iota$ is compact.    
	Consequently, 
	\begin{align*}
		(\A,E_B\tens_B F_C, \D_1\tens 1+1\tens_\nabla \D_2)
	\end{align*}
	is an unbounded Kasparov $A-C$ module.
	
	In order to see that it is the product of the two modules as claimed, we need to invoke Kucerovsky's criterion, see \Cref{kucerovskycrit}. 
	
	Define $W=\dom (1\tens_{\nabla} \D_2)$, and define the operators $T_e$, and $T_e^*$ as in \Cref{kucerovskycrit}. Define the operator $Q$,
	\begin{align*}
		Q=\comm{\begin{pmatrix} \D_1\tens 1+ 1\tens_\nabla \D_2 & 0 \\ 0 & \D_2 \end{pmatrix}, \begin{pmatrix} 0 & T_e\\ T_e^* & 0\end{pmatrix}}
	\end{align*}
	the boundedness of which we need to check.  
	Calculating, for $(e'\tens f',f)\in \dom (\D_1\tens +1 +1\tens_\nabla \D_2)\osum \dom(\D_1)$:
	\begin{align*}
		Q\begin{pmatrix} e'\tens f' \\ f \end{pmatrix}&=\begin{pmatrix} \D_1e\tens f+(-1)^{\part e} \nabla_{\D_2}(e) f \\  \ip{e}{\D_1 e'}f+[\D_2,\ip{e}{e'}]f+(-1)^{\part(e')}\ip{e}{\nabla_{\D_2}(e')}f\end{pmatrix} \\
		&=\begin{pmatrix} \D_1e\tens f +(-1)^{\part e} \nabla(e)f \\ \ip{\D_1e}{e'}f+\ip{\nabla(e)}{e'}f\end{pmatrix}.
	\end{align*}
	For fixed $e \in W$ we see that $Q$ extends to a bounded operator as desired. 
	
	The remaining item we need to check is the semi-boundedness condition. 
	Define $s=\D_1\tens 1,t=1\tens_\nabla \D_2$. We consider $s$ and $t$ on their common core $V=(s+\lambda i)^{-1}\dom(t)\ran(s+\lambda i)^{-1}(t+\lambda i)^{-1}$, see \Cref{boundedness}. Here we can rewrite the commutator in Kucerovsky's criterion from a quadratic form to an operator expression, recalling that $[s,s]=2s^2$:
	\begin{align*}
		\ip{[ \D_1\tens 1+ 1\tens_\nabla \D_2,\D_1\tens 1]x}{x}=2\ip{sx}{sx}+\ip{[s,t]x}{x}\geq -\norm{[s,t]}\ip{x}{x}
	\end{align*}
	To establish semi-boundedness from below of $2\ip{sx}{sx}+\ip{[s,t]x}{x}$, remark that $2\ip{sx}{sx}\geq 0$ and by \Cref{boundedness} the commutator $[s,t]$ is bounded on $V$. Thus:
	\begin{align*}
		2\ip{sx}{sx}+\ip{[s,t]x}{x}\geq -\norm{[s,t]}\ip{x}{x} \quad \for x \in V
	\end{align*}
	By \Cref{sumselfadjoint} $V$ is a core for $s+t$, allowing us to conclude the semi-boundedness on the entirety of $\dom s+t$. 
	Thus our module represents the Kasparov product as claimed.
\end{proof}
One may then inspect the proof of the lifting theorems in \cite{mesrennie} for the Kasparov product, and see that these all readily go through in the case with no caveats or modifications necessary, and as such any Kasparov product can, in theory at least, be calculated in the unbounded theory. As remarked throughout the litterature, \cite{mesland},\cite{kaad},\cite{jensmorita},\cite{suijlekom}, a great loss of geometric data occurs when passing to $KK$-theory as it cannot see differential structures. 
As such it would be beneficial to develop a theory which encapsulates the geometric data, and this is is exactly what is attempted in \cite{jensmorita}. It is however an open problem to develop an appropriate product for this theory, which also works for non-complete manifolds, eg. the half-open interval. This theory is also different in flavor from ordinary unbounded KK-theory, as the usual cycles are replaced with their generalizations, called modular cycles, thereby precluding usage of Kucerovsky's criterion to show that we recover the Kasparov product and leading to the necessity of much greater technical sophistication to show that we reocver the usual product. 
Along with the further work in developing an unbounded version of $KK$-theory encapsulating geometric data, it would also be interesting to see if one could use the methods of unbounded $KK$-theory to find concrete representatives of $K$-homology classes in the cases where we know we have Poincare duality in $KK$-theory. Doing this would in particular be interesting in the real case, as the cycles might be found to have some physical significance as encoding the geometry of the non-commutative space in question. Showing Poincare duality in the real case is, however, a hard problem as will be discussed in the next section. 
\newpage
\appendix
\section{A brief introduction to Clifford Algebras}
%In order to properly understand unboundeed real $KK$-theory, it is necessary that we first take a trip to the world of spin geometry. In this section we shall start by defining the notions spin groups and dirac bundles, as well as exploring the intimate relationship these structures have with the classification of Clifford modules and the groups which one may associate to this. In essence, non-commutative geometry may in some sense be thought of as the natural generalization of Spin geometry motivated by the Atiyah-Singer index theorem. Thus for the fullness of the presentation we start by briefly stating some results on real spin geometry, which shall be essential when developing unbounded representatives of the real $K$-homology of $\R$. 
%Further, these results serve as motivation and intuition for the development of unbounded $KK$-theory. 


%\subsection{The classification of Clifford modules and the Atiyah-Bott-Shapiro construction}
We wish to classify the Clifford modules, and from this construct a group $\hat{A}_*$, which it turns out it is isomorphic to the $KO$-theory of a point, and thereby directly to the $K$-homology of a point. Thus the Clifford modules which arise here will eventually turn out to relate directly to the kernels of suitable Dirac operators. 
We start by defining the Clifford algebras
\begin{definition}
	We define the real Clifford algebras as follows 
	\begin{align*}
		Cl_{p,q}=\spn_{\R}\{\gamma_1,\dots,\gamma_p,\rho_1,\dots,\rho_q| (\gamma_i)^2=1,\gamma_i^*=\gamma_i,(\rho_i)^{2}=-1,(\rho_i)^*=-\rho_i\}
	\end{align*}
	with $x_jx_i=-x_jx_i$ where $x_j,x_i$ are distinct generators. 
\end{definition}
In analogy with the complex situation, we have the result that we may build the higher Clifford algebras from the lower Clifford algebras. 
\begin{lemma}
	We have the isomorphism
	\begin{align*}
		Cl_{p,q}\tensh Cl_{p',q'}\cong Cl_{p+p',q+q'}
	\end{align*}
\end{lemma}
 We start by briefly recalling the classification of Clifford algebras, as given in the following table for $Cl_{0,i},Cl_{i,0}$. 
			\begin{align*}
			\begin{array}{c c c} 
			k  & Cl_{0,i} & Cl_{i,0} \\
			1 & \C & \R\osum \R \\
			2 & \mathbb{H} & M_{2}(\R) \\
			3 & \mathbb{H}\osum \mathbb{H} & M_{2}(\C) \\
			4 & M_2(\mathbb{H}) & M_2(\mathbb{H})  \\
			5 & M_4(\C)) & M_2(\mathbb{H}) \osum M_2(\mathbb{H})  \\
			6 & M_{8}(\R) & M_8(\C) \\ 
			7 & M_8(\R)\osum M_8 (\R) & M_8(\C)\osum M_8(\C)  \\
			8 & M_{16}(\R) & M_{16}(\R)
			\end{array}
			\end{align*}
\begin{proposition}\label{equivalentcliff}
	The categories of ungraded $Cl_{n-1}$-modules and graded $Cl_n$ modules are equivalent, via. the identification of $W^{(0)}\osum W^{(1)} \mapsto W^{(0)}$ and $W^{(0)}\mapsto W^{(0)}\tens_{Cl_n^0} Cl_n$
\end{proposition}
Thus by the classification of Clifford algebras and the result above we get
\begin{definition}\label{mk}
	Define $\hat{\M}_k$ as the group generated by unitary equivalence classes of irreducible representations of $Cl_{0,k}$. 
\end{definition}
We now state the Atiyah-Bott-Shapiro(ABS) theorem.

\begin{theorem}\label{abstheorem}
	Consider the map $\iota:\R^n\to \R^{n+1}$, this induces a morphism $\iota_*:Cl_n\to Cl_{n+1}$. Thus by restriction, we get a morphism $\iota^*:\hat{\M}_{n+1}\to \hat{\M}_n$. Defining the groups $\hat{A}_n=\hat{\M}_{n}/\iota^*(\hat{\M}_{n+1})$, we have an isomorphism $\hat{A}_n\cong KO^{-n}(\{ * \})\cong KO_{n}(\R)$. 
\end{theorem}
We have the accompanying table, in which we also have the groups $\hat{M}_n$ included. We note that the groups are generated by the unique irreducible representations of the simple algebras.
\begin{align*} 
\begin{array}{c c c}
k & \hat{A}_k & \hat{\M}_k \\ 
0 & \Z & \Z \\
1 & \Z_2 & \Z \\
2 & \Z_2 & \Z \\
3 & 0 & \Z \\
4 & \Z & \Z\osum \Z \\
5 & 0 & \Z \\
6 & 0 & \Z \\
7 & 0 & \Z 
\end{array}
\end{align*}
%\end{comment}
\newpage
\section{Bibliography}
\bibliographystyle{alpha}
\bibliography{speciale}
\end{document}	
