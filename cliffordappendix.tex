%In order to properly understand unboundeed real $KK$-theory, it is necessary that we first take a trip to the world of spin geometry. In this section we shall start by defining the notions spin groups and dirac bundles, as well as exploring the intimate relationship these structures have with the classification of Clifford modules and the groups which one may associate to this. In essence, non-commutative geometry may in some sense be thought of as the natural generalization of Spin geometry motivated by the Atiyah-Singer index theorem. Thus for the fullness of the presentation we start by briefly stating some results on real spin geometry, which shall be essential when developing unbounded representatives of the real $K$-homology of $\R$. 
%Further, these results serve as motivation and intuition for the development of unbounded $KK$-theory. 


%\subsection{The classification of Clifford modules and the Atiyah-Bott-Shapiro construction}
We wish to classify the Clifford modules, and from this construct a group $\hat{A}_*$, which it turns out it is isomorphic to the $KO$-theory of a point, and thereby directly to the $K$-homology of a point. Thus the Clifford modules which arise here will eventually turn out to relate directly to the kernels of suitable Dirac operators. 
We start by defining the Clifford algebras
\begin{definition}
	We define the real Clifford algebras as follows 
	\begin{align*}
		Cl_{p,q}=\spn_{\R}\{\gamma_1,\dots,\gamma_p,\rho_1,\dots,\rho_q| (\gamma_i)^2=1,\gamma_i^*=\gamma_i,(\rho_i)^{2}=-1,(\rho_i)^*=-\rho_i\}
	\end{align*}
	with $x_jx_i=-x_jx_i$ where $x_j,x_i$ are distinct generators. 
\end{definition}
In analogy with the complex situation, we have the result that we may build the higher Clifford algebras from the lower Clifford algebras. 
\begin{lemma}
	We have the isomorphism
	\begin{align*}
		Cl_{p,q}\tensh Cl_{p',q'}\cong Cl_{p+p',q+q'}
	\end{align*}
\end{lemma}
 We start by briefly recalling the classification of Clifford algebras, as given in the following table for $Cl_{0,i},Cl_{i,0}$. 
			\begin{align*}
			\begin{array}{c c c} 
			k  & Cl_{0,i} & Cl_{i,0} \\
			1 & \C & \R\osum \R \\
			2 & \mathbb{H} & M_{2}(\R) \\
			3 & \mathbb{H}\osum \mathbb{H} & M_{2}(\C) \\
			4 & M_2(\mathbb{H}) & M_2(\mathbb{H})  \\
			5 & M_4(\C)) & M_2(\mathbb{H}) \osum M_2(\mathbb{H})  \\
			6 & M_{8}(\R) & M_8(\C) \\ 
			7 & M_8(\R)\osum M_8 (\R) & M_8(\C)\osum M_8(\C)  \\
			8 & M_{16}(\R) & M_{16}(\R)
			\end{array}
			\end{align*}
\begin{proposition}\label{equivalentcliff}
	The categories of ungraded $Cl_{n-1}$-modules and graded $Cl_n$ modules are equivalent, via. the identification of $W^{(0)}\osum W^{(1)} \mapsto W^{(0)}$ and $W^{(0)}\mapsto W^{(0)}\tens_{Cl_n^0} Cl_n$
\end{proposition}
Thus by the classification of Clifford algebras and the result above we get
\begin{definition}\label{mk}
	Define $\hat{\M}_k$ as the group generated by unitary equivalence classes of irreducible representations of $Cl_{0,k}$. 
\end{definition}
We now state the Atiyah-Bott-Shapiro(ABS) theorem.

\begin{theorem}\label{abstheorem}
	Consider the map $\iota:\R^n\to \R^{n+1}$, this induces a morphism $\iota_*:Cl_n\to Cl_{n+1}$. Thus by restriction, we get a morphism $\iota^*:\hat{\M}_{n+1}\to \hat{\M}_n$. Defining the groups $\hat{A}_n=\hat{\M}_{n}/\iota^*(\hat{\M}_{n+1})$, we have an isomorphism $\hat{A}_n\cong KO^{-n}(\{ * \})\cong KO_{n}(\R)$. 
\end{theorem}
We have the accompanying table, in which we also have the groups $\hat{M}_n$ included. We note that the groups are generated by the unique irreducible representations of the simple algebras.
\begin{align*} 
\begin{array}{c c c}
k & \hat{A}_k & \hat{\M}_k \\ 
0 & \Z & \Z \\
1 & \Z_2 & \Z \\
2 & \Z_2 & \Z \\
3 & 0 & \Z \\
4 & \Z & \Z\osum \Z \\
5 & 0 & \Z \\
6 & 0 & \Z \\
7 & 0 & \Z 
\end{array}
\end{align*}