The development of unbounded KK-theory is foreshadowed in Kasparov's original paper \cite{kasparov} where all Bott elements are bounded transforms of naturally defined unbounded cycles. The study of unbounded $KK$-theory begins with the paper by Baaj and Julg, \cite{baajjulg} where they define unbounded $KK$-cycles:
\begin{definition}[Unbounded $KK$-cycles]
	We define the set $\Psi(A,B)$ of unbounded Kasparov $A-B$ cycles as the set of quadruples $(E,\pi,\A,\D)$ where $E$ is a graded Hilbert $B$-module and $\A$ is a dense subalgebra, $\pi:A\to L(E)$ is a $^*$-homomorphism, and $\D$ is a degree one unbounded self-adjoint regular densely defined operator on $E$ satisfying that 
	\begin{align*}
		&(1+\D^2)^{-1}\in \{ T\in L(E):\pi(A)T,T\pi(A)\subset K(E)\} \\
		&\pi(\A)\dom(\D)\subset \dom(\D) 
	\end{align*}
	and for all $a\in \A$, the operator $[\D,\pi(a)]$ extends uniquely to an operator in $L(E)$. 
\end{definition}
We recognize the cycles defining the real $K$-homology of $\R$ as unbounded Kasparov cycles.
\begin{remark}
	The commutators and gradings on $A$ and $E_B$ are related: 
	\begin{align*}
		&[\D,\pi(a)]=\D\pi(a)-\pi(\gamma(a))\D \\
		&[\D,\pi(a)]^*=-[\D,\pi(\gamma_A(a^*))] \\
		&[\pi(\gamma_A(a))]=\gamma_{E_B}\pi(a)\gamma_{E_B}
	\end{align*}
\end{remark}
The utility of unbounded Kasparov modules was immediate, as it allowed \cite{baajjulg} to write the product operator in the exterior Kasparov product simply as 
\begin{align*}
	D_1\tens 1+1\tens D_2
\end{align*}
It was an open problem for many years to construct an unbounded version of the interior Kasparov product, but since the work of \cite{mesland} this problem has come ever closer to being solved in full generality through the work of \cite{kaad}, \cite{suijlekom} and various others. In this section, we shall give the hitherto most general construction of the unbounded interior Kasparov product by expanding the results  \cite{mesrennie}  to the real setting. We generally follow their methods closely, only occasionally adapting them to the real case when necessary. 


As it will turn out the natural setting for the unbounded Kasparov product is the category of operator spaces, not \Cstar-algebras so we need to introduce the framework of operator spaces and algebras, in particular complete differential algebras.
\subsection{Operator spaces and algebras}
\begin{definition}
\begin{enumerate}
\item 
	An operator algebra $\A$ is a closed subalgebra of a \Cstar algebra $B$. As we may represent $B$ isometrically on $B(H)$, we may assume $\A\subset B(H)$. 
\item
	An operator $^*$-algebra is an operator algebra $\A\subset B(H)$ with a completely bounded involution $^*:\A\to \A$. This involution will in general not coincide with the involution on $B$.
\item	
	An operator space $X$ is a closed subspace of a \Cstar algebra.
\item 
	Let $\A$ be an operator algebra, and $X$ be an operator space. If there is a continuous (left or right)-action $\A\times X\to X$, $X$ is a (left or right)-$\A$-module. 
\item
	Let $A$ and $B$ be \Cstar-algebras. Suppose that $X\subset A$ and $Y\subset B$ are operator spaces, and let $\psi:X\to Y$ be a linear map. There are unique norms $\norm{\cdot}_n$ on $M_n(X)$ and $M_n(Y)$ stemming from the unique norms on $M_n(A)$ and $M_n(B)$ respectively. 
	Define $\phi_n=\phi \tens 1_n:M_n(X)\to M_n(Y)$. If 
	\begin{align*}
		\sup_{n\in \N} \sup_{\norm{a}_n\leq 1}\norm{\phi_n(a)}_{n}<\infty
	\end{align*}
	Then $\phi$ is said to be completely bounded, with 
	\begin{align*} 
		\norm{\phi}_{cb}=	\sup_{n\in \N} \sup_{\norm{a}_n\leq 1}\norm{\phi_n(a)}_{n}. 
	\end{align*} 
	Likewise, $\phi$ is completely isometric if $\phi_n$ is an isometry for all $n$, and likewise for contractiveness. 
\end{enumerate}
\end{definition}



\begin{remark}
	There are two \Cstar algebras associated to a given operator algebra:
	\begin{enumerate}
	\item
	$C^*(\A)\subset A$, the smallest subalgebra of $A$ containing $\A$. 
	\item
	$A$: The \Cstar-closure of $A$, which stems from viewing $\A$ as a Banach $^*$-algebra and completing in the norm defined from the spectral radius of $a^*a$. 
	\end{enumerate}
\end{remark}
\begin{definition}
Given an unbounded Kasparov module $(\A,E_B,\D)$ with grading $\gamma$ define the algebra $\A_D=\{a\in A| a\dom(\D) \subset \dom(\D), [\D,a]\in L(E_B)\}$. We may equip this with the structure of an operator $^*-$algebra via. the representation 
\begin{align*}
	&\pi_{\D}:\A\to L((E_B)\osum E_B) \\
	&a\mapsto \begin{pmatrix} \pi(a) & 0 \\ [\D,\pi(a)] & \pi(\gamma(a)) \end{pmatrix}
\end{align*} 
With the $^*$-operation given by $\pi_{\D}(a)^*=\pi(\gamma(a))$.  We shall always assume $\A_\D$ to be equipped with the topology coming from $\norm{\pi_\D(a)}_{L(E_B\osum E_B)}$. 
\end{definition}
\begin{assumption}
	The representation $\pi$ will be assumed to be faithful throughout the thesis, as else we would have to consider the the representation 
	\begin{align*}
		a\mapsto a\osum \pi_\D(a) \in A\osum L_B(E_B\osum E_B)
	\end{align*}
	which does not change the results, but clutters the calculations. 
\end{assumption}
As motivation for the definition of the Lipschitz representation, remark that if $a\in C_0^1(\R)$ and $\D$ is the derivative on $\R$, $[\pi(a),\D]=a'$. In general, one should think of the commutators with $\D$ as derivatives with respect to $\D$. 
An important generalization of this is the full Lipschitz algebra associated to a closed symmetric regular operator on a \Cstar module $E_B$, as well as the notion of a differentiable algebra. 
\begin{definition}
We define the Lipschitz algebra associated to a self-adjoint regular operator $\D$:
\begin{align*}
	\lip(\D)=\{T\in L(E_B) \mid T\dom(\D^*)\subset \dom(\D), [\D^*,T]\in L(E_B)\}
\end{align*}
\end{definition}
We can use this to define the general framework of differentiable algebras, which are algebras where the operators behave like differentiable functions. 
\begin{definition}
We define a differentiable algebra $\A$ as a separable operator $^*$-subalgebra of $\lip(\D)$ closed in the topology stemming from $\pi_{\D}$. Projecting onto the first coordinate of $\pi_{\D}(\A)$, the \Cstar closure of $\A$ algebra coincides with the closure of $\A$ as a subalgebra of $L(E_B)$, thereby giving a subalgebra of $A$. 
\end{definition}
We need to define a tensor product on the category of operator spaces, so we may proceed with our constructions in the setting of operator spaces rather than Hilbert \Cstar modules. 
\begin{definition}
	Given two operator spaces $X,Y$ we may define their algebraic tensor product $X\odot Y$. Define the Haagerup norm for $z\in M_n(X\odot Y)$
	\begin{align*}
		\norm{z}^2=\inf \left \{ \norm{\sum_{i=1}^m  x_ix_i^*} \norm{\sum_{i=1}^m y_i^*y_i} : z=\sum_{i=1}^m x_i\tens y_i \right \}
	\end{align*}
	We shall denote the Haagerup norm as $\norm{\cdot}_{\optens}$. Let $\A$ be an operator algebra. If $X$ is a left and $Y$ is a right operator module, define the Haagerup module tensor product:
	\begin{align*}
		X\optens_\A Y
	\end{align*}
	as the quotient of $X\optens Y$ by the closed linear spans of expressions of the form $x\tens ay-xa\tens y$. The Haagerup tensor product serves to make multiplication $X\optens \A\to X$ continuous. 
\end{definition}
\begin{remark}
	In case $E_B$ and $F_C$ are Hilbert modules, with a representation $\pi:B\to F_C$, we have the cb. isomorphism $E_B\optens_B F_C \cong E_B\tens_B F_C$, see \cite{blecher}.
\end{remark}
To see this is well-defined, we refer to \cite{blecher}. 
Likewise, we would like to define the notion of inner product operator modules. 
\begin{definition}
	An inner product operator module $\E$ is a normed right operator module over an operator $^*$-algebra $\mathcal{B}$ with a sesquilinear pairing $\E\times \E\to \mathcal{B}$ such that 
	\begin{align*}
	&\ip{e_1}{e_2b}=\ip{e_1}{e_2}b \\
	&\ip{e_1}{e_2}^*=\ip{e_2}{e_1} \\
	&\ip{e}{e}\geq 0 \text{ in } B\\
	&\ip{e}{e}=0 \Lr e=0
	\end{align*}
	Further we require that $\ip{\cdot}{\cdot}$ satisfies a weak version of the Cauchy-Schwarz inequality for $C>0$: \begin{align*} \norm{\ip{e_1}{e_2}}\leq C \norm{e_1}_{\E}\norm{e_2}_{\E} \end{align*} for all matrix norms. 
\end{definition}
\begin{remark}
	We do not require $\E$ to be complete in the inner product $\ip{}{}_{\E}$, and in general the topology induced from the $\B$-valued inner product will differ from the norm topology on $\E$. 
\end{remark}

The operator-algebraic analogue of having an exhausting net of subspaces in a locally compact Hausdorff space is the following 
\begin{definition}
	Let $\A$ be an operator algebra, then a bounded approximate unit for $\A$ is a net $(u_{\lambda})_{\lambda\in I}$ such that $\sup_{\lambda \in I}\norm{u_{\lambda}}<\infty$ and $\lim_{\lambda\in I}\norm{u_\lambda a-a }=\lim_{\lambda \in I}\norm{au_{\lambda}-a}=0$. We say that it is commutative if $u_{\lambda}u_{\mu}=u_{\mu}u_{\lambda}$. The approximate unit is said to be sequential if the index set is the natural numbers with the usual ordering. 
\end{definition}
As we are working over operator algebras and operator spaces, we slightly need to expand the notion of a representation so we may analyze $\A$ via. its representation theory as for \Cstar-algebras. 
\begin{definition}
	A completely bounded (cb) representation of an operator algebra $\A$ is a completely bounded homomorphism $\pi:\A\to B(H)$. An representation is non-degenerate if $\pi(A)H$ is dense in $H$. 
\end{definition}
Bounded approximate units of an operator algebra converge strongly to idempotents under completely bounded representations. 
\begin{lemma}
	Let $\pi:A\to B(H)$ be a completely bounded representation. Then
	\begin{align*}
		H=\overline{\pi(\A)H}\osum (\pi(\A)H)^\perp=\overline{\pi(A)^*H}\osum (\pi(\A)^* H)^\perp
	\end{align*}
	Defining $\Nil(\pi(\A))=\{h\in H: \pi(a)h=0, \quad \forall a\in A\}$, it satisfies $\Nil(\pi(\A))=\overline{(\pi(\A)^*)}^{\perp}$. 
\end{lemma}
\begin{proof}
	Let $a\in \A$ and $\xi,\eta\in H$, then the decomposition of $H$ follows from the identity: $\ip{\pi(a)\xi}{\eta}=\ip{\xi}{\pi(a)^*\eta}$. 
	For the identity on $\Nil(\pi(A))$, let $h\in \Nil(\pi(\A)),v\in H$ and $a\in \A$. Then 
	\begin{align*}
		\ip{h}{\pi(a)^*v}=\ip{\pi(a)h}{v}=0
	\end{align*}
	So $\Nil(\pi(\A))\subset \overline{\pi(A)^*H}^\perp$. For the converse, let $h\in  \overline{\pi(A)^*H}^\perp$, $v\in H$, and $a\in \A$. Then 
	\begin{align*}
		\ip{\pi(a)h}{v}=\ip{h}{\pi(a)^*v}=0
	\end{align*}
	hence $\pi(a)h=0$, and $h\in \Nil(\pi(\A))$.  
\end{proof}
With this minor remark out of the way, we can show the claimed strong convergence. 
\begin{proposition}\label{mesrennie17}
	Let $\A$ be an operator algebra with bounded approximate unit $(u_{\lambda})_{\lambda \in I}$ and a completely bounded representation $\pi:\A \to B(H)$. Then $\pi(u_\lambda)$ converges in the strong operator topology to an idempotent $q\in B(H)$ satisfying the following 
	\begin{enumerate}
		\item
			$q\pi(a)=\pi(a)q=\pi(a)$
		\item
			$qH=\overline{\pi(\A)H}$. 
		\item
			$(1-q)H=\Nil(\pi(\A))$. 
		\item
			$\norm{q}\leq \norm{\pi}\sup_{\lambda \in I}\norm{u_\lambda}$
	\end{enumerate}
\end{proposition}
\begin{proof}
	Define the projection $q:H\to \overline{\pi(\A)H}$ and the projection $p_*:H\to \overline{\pi(\A)^*H}$. We may define the self-adjoint operator $t=p+(1-p_*)$. We wish to show that this is injective with dense image. To see that it is injective, consider $px=(p_*-1)x$, which implies that $px\in \overline{\pi(\A)H}\cap \overline{\pi(\A)H}^\perp=\{0\}$. Thus $px=0$, and thereby we get we get that $x=(1-p)x=p_*x$, thereby that $x\in \Nil(\pi(\A)^*)\cap \overline{\pi(\A)^*H}=\{0\}$. By self-adjointness we have that $\overline{((p+(1-p_*)))H}=\ker(p+(1-p_*))^\perp=H$, giving that it has dense image. 
	
	Thus we have that $\overline{\pi(\A)H}+\Nil\pi(\A)$ is dense. Pick an arbitrary $\xi\in H$ and $\epsilon>0$. Then we have $\eta_0\in \overline{\pi(\A)H}$ and $\eta_1\in \Nil\pi(\A)$ such that $\norm{\xi-(\eta_0+\eta_1)}\leq \epsilon/4C$ for $C=\sup_{\lambda \in I}\norm{\pi(u_{\lambda})}$. Pick $\lambda\in I$ such that for all $\mu>\lambda$ we have that $\norm{\pi(u_\lambda-u_\mu)\eta_0}<\epsilon/2$. Then we may perform the following estimate to show that $\pi(u_\lambda)$ is strongly Cauchy. 
	\begin{align*}
		&\norm{\pi(u_{\lambda}-u_\mu)\xi} \\
		&\leq \norm{\pi(u_\lambda-u_\mu)(\eta_0+\eta_1)}+\norm{\pi(u_\lambda-u_\mu)(\xi-(\eta_0+\eta_1))} \\
		&\leq \norm{\pi(u_\lambda-u_\mu)\eta_0}+\norm{\pi(u_\lambda-u_\mu)}\norm{\xi-(\eta_0+\eta_1)}\leq \frac{\epsilon}{2}+\frac{\epsilon}{2}
	\end{align*}
	Therefore we may define the limit $q$ of $\pi(u_\lambda)$ as the strong topology is complete on bounded sets. We see that $q$ commutes with the representation, and that $q$ is an idempotent from the construction. As it is an idempotent it has closed range. Thus $\im(q)=\overline{\pi(\A)H}$. If we consider $(1-q)$, we see that $(1-q)\pi(a)=0$, thus $\im(1-q)=\Nil(\pi(\A))$.
	To see that the norm of $q$ is bounded as claimed, consider the following 
	\begin{align*}
		\norm{q}=\sup_{h,\norm{h}=1}\norm{\lim_{\lambda}\pi(u_\lambda)h}\leq \norm{\pi}\sup\norm{u_\lambda}. 
	\end{align*}
\end{proof}
If we wish to drop the assumption of our approximate unit being bounded, it is necessary to find a different method of proof. Fundamentally, we are only working on non-commutative analogues of complete manifolds, as expanded upon in \cite[Section 2]{mesrennie} and in order to generalize to non-complete manifolds with symmetric operators, we need an entirely new framework as introduced in for example \cite{kaad}. 
As a geometric corollary of the previous propositions, we get that $H\cong \overline{\pi(\A)H}+\overline{\pi(\A)^*H}$. 
The following lemma for Kasparov modules shows the fundamental relationship between $\D$ being self-adjoint and there being a bounded approximate unit for $\A$. 
\begin{theorem}\label{mesrennie19}
	Let $(\A,E_B,\D)$ be a Kasparov module, where we only require that $\D$ is symmetric, such that $\overline{\pi(\A)E_B}$ is a complemented sub-module and $p$ the corresponding projection. Assume that $(\D u_ne)$ converges for all $e\in \dom(\D^*)$ and $p\in \lip(\D^*)$ and that the operator $[\D^*,p]$ is the strict limit of the sequence $([\D^*,u_n])_{n\in \N}$. 
	\begin{enumerate}
	\item
		Then $p[\D^*,u_n]p\to 0$ in the strict topology,  
	\item
		$p$ is the strict limit of $u_n$.  
	\item
		If $\A E_B$ is dense in $E_B$, $\D$ is self-adjoint. 
	\end{enumerate}
\end{theorem}
\begin{proof}
	By assumption, $\overline{\pi(A)E_B}$ is complemented so we may consider the projection $p$ onto this submodule. Let $e\in \overline{\pi(A)E_B}$, then $u_ne\to e$ as for $ae\in \pi(A)E_B$ we have $ae-u_nae=(a-u_na)e\to 0$, so by continuity we have the same on all of $\overline{\pi(A)E_B}$

	We have $pa=ap=a$, so $(1-p)a=a(1-p)=0$, implying 
	\begin{align*}
		\lim_{n\to \infty} u_n e=\lim_{n\in \N} u_n pe+u_n(1-p)e=\lim_{n\in \N} u_n pe=pe
	\end{align*}
	giving first part of the statement. As $(u_n)_{n\in \N}$ is a bounded approximate for $\A$, the sequence of operators $[\D^*,u_n]$ is uniformly bounded. Letting $a\in \A$ and $e\in \dom(\D)^*$ we get the identity 
	\begin{align*}
		[\D^*,u_n]ae=[\D^*,u_na]e-u_n[\D^*,a]e
	\end{align*}
	As $ae=pae=ape$, multiplying from the left by $p$ we get
	\begin{align*}
		p[\D^*,u_n]pae=p[\D^*,u_na]pe-pu_n[\D^*,a]pe
	\end{align*}
	This will converge to zero, as each term on the right hand side converges to $p[\D,a]pe$. As $\pi$ is assumed to be essential on $pE_B$, linear combinations of terms of the form $ae$ are dense in $pE_B$, it follows that $p[\D,u_n]p$ converges pointwise to zero. We have assumed that we have a symmetric Kasparov module, along with $(u_n)_{n\in \N}$ being even, so we may infer that $(p[\D^*,u_n]p)^*=-p[\D^*,u_n^*]p$. Thereby $(u_n^*)_{n\in \N}$ also becomes a bounded approximate unit for $\A$. Replacing $u_n$ with $u_n^*$ throughou the previous argument then gives that $p[\D,u_n^*]p$ also converges strictly to zero. 
	To show the second point we first show that $p$ maps $\dom(\D)^*$ to $\dom(\D)$ and then use this to show that $[\D^*,u_n]$ converges to $[\D^*,p]$ on $\dom(\D)^*$. 
	
	We have assumed that the following sequence is convergent
	\begin{align*}
		\pi_\D(u_n)\begin{pmatrix} e \\ \D^* e \end{pmatrix}=\begin{pmatrix} u_n e\\  \D u_n e\end{pmatrix}
	\end{align*}
	By the first part of the theorem $p$ is the strict limit of $u_n$, so 
	\begin{align*}
		\lim_{n\to \infty} u_n\begin{pmatrix} e \\ \D^* e \end{pmatrix}=\begin{pmatrix} pe\\  x \end{pmatrix}
	\end{align*}
	By closedness of $\D$, $pe\in \dom(\D)$ and $x=\lim_{n\in \N} \D u_n e=\D pe$. This shows that $p\dom(\D)^*\subset \dom(\D)$, so for every $e \in \dom(\D)^*$ we get the identities 
	\begin{align*}
		[\D^*,u_n]e&=[\D^*,u_n]pe+[\D^*,u_n](1-p)e \\
		&=\D u_npe-u_n \D pe+\D u_n(1-p)e-u_n \D^* (1-p)e \\
		&=\D u_n e-u_n \D pe -u_n \D^*(1-p)e 
	\end{align*}
	which converges to $\D pe-p\D pe -p \D^* (1-p)e=[\D^*,p]e$. As $[\D^*,u_n]$ is bounded, the convergence is strict on all of $E_B$ and thereby the operator $[\D^*,p]$ is bounded on $\dom(\D)^*$. 
	
	To show the final part of the theorem, note that $[\D^*,u_n]\to 0$ since in this case $p=1$. For every $e\in \dom(\D)^*$ we get $\D^*u_n e=[\D^*,u_n]e+u_n\D^*e\to \D^*e$. As $u_ne\to e$ and $u_n e\in \dom(\D)$, it follows that $e$ lies in the graph-norm completion of $\D$. Since $e$ was arbitrary in $\dom(\D)^*$, it follows that $\dom(\D)^*\subset \dom(\D)$, giving self-adjointness. 
	
\end{proof}
We gather up a plethora of results illustrating the geometric control imposed by the presence of a bounded approximate unit in the following theorems. 
\begin{theorem}\label{mesrennie112}
	Let $\A$ be an operator algebra with a bounded approximate unit and an essential cb. representation $\pi:\A\to B(H)$. Then the norm on $M_n(\A)$ is equivalent to the norm $\norm{\cdot}_{op,n}$, $\norm{a}_{op,n}=\sup_{\norm {b}_n\leq 1}\norm{ab}_n$ where $\norm{\cdot}_n$ is the matrix norm. 
\end{theorem}
\begin{proof}
	We start by noting that $(u_\lambda)_{\lambda\in \Lambda}$ gives rise to bounded approximate units $1_n \cdot u_\lambda$ on $M_n(\A)$, and as such we may simply run the argument for $u_\lambda$.
	It is clear that $\norm{a}_{op,n}\leq \norm{a}$. Letting $u$ be a bounded approximate unit, we clearly have $\frac{1}{c}\norm{u_{\lambda}}\leq 1$ for some fixed $c$. For all $\epsilon>0$ there exists a $\lambda$ such that $\norm{b-bu_\lambda}<\epsilon$. This gives us the following string of inequalities
	\begin{align*}
		\frac{1}{c}(\norm{b}_{op,n}-\epsilon)<\frac{1}{c}\norm{b}-\norm{b-bu_\lambda}\leq \frac{1}{c} \norm{bu_\lambda} \leq \norm{b}_{op,n}
	\end{align*}
	Showing the desired. 
\end{proof}
\begin{definition}
	Let $T:\A\to \A$ where $\A$ is an operator algebra. Define $\norm{T}_{op}=\sup_{n \in \N} \norm{T\tens 1_n}_{op,n}$. 
\end{definition}
An operator $T:\A\to \A$ is cb. if and only $\norm{T}_{op}<\infty$. Thus, by \Cref{mesrennie112} we can use $\norm{\cdot}_{op}$ to define the completely bounded version of the strict topology on an operator algebra: 
\begin{definition}
	Let $\A$ be an operator algebra with a bounded approximate unit. We define the multiplier algebra of $\A$ as the strict closure of $\A$, ie. by
	\begin{align*}
		M(\A)=\{T:\A\to \A\mid \exists (b_\lambda)_{\lambda \in \Lambda}\subset \A, \lim_{\lambda \in \Lambda} \norm{b_\lambda a-Ta}_{op}=\lim_{\lambda \in \Lambda}\norm{ab_\lambda-T a}_{op}=0, \forall a \in A\}
	\end{align*}
	Where we define $\norm{T}=\norm{T}_{op}$. 
\end{definition}
\begin{theorem}\label{mesrennie114}
\begin{enumerate}
	\item Let $\A$ be an operator algebra with a bounded approximate unit and an essential cb. representation $\pi:\A\to B(H)$. The cb. representation extends to a representation of the multiplier algebra of $\A$, such that $\pi(1)=1$. 
	\item If we further assume that $\pi$ is a cb-isomorphic representation, then as for \Cstar-algebras we get the strict closure of $\A$, ie. the multiplier algebra of $\A$ is cb-isomorphic to the idealiser of $\pi(\A)$. Furthermore, every element in $M(\A)$ is the strict limit of a bounded net in $\A$. Finally, closed ideals of $\A$ descend to closed ideals of $M(\A)$. 
	\item Assume that $\pi:\A\to B(H)$ is a cb. representation of an operator with bounded approximate unit. Then $\pi$ extends to a representation $M(\A)\to B(H)$ such that $\pi(1)$ is an idempotent and $\overline{\pi(\A)H}=\pi(1)H$ and $(1-\pi(1))H=\Nil(\pi(\A))$. 
	\end{enumerate}
\end{theorem}
\begin{proof}
	\begin{enumerate}
	\item
		We have assumed that $H=\overline{\pi(\A)H}$, so for all $h\in H$ we have that $u_\lambda h$ converges to $h$. Picking $b\in M(\A)$ and utilizing that $M(\A)$ is the strict closure of $\A$, we get $\sup_{\lambda}\norm{bu_\lambda}<\infty$ and that $(bu_\lambda a)_{\lambda\in \Lambda}$ is norm-Cauchy in $\A$ for all $a\in \A$. Defining 
		\begin{align*}
			\pi(b)\pi(a)h=\lim_{\lambda \in \Lambda} \pi(bu_\lambda a)h
		\end{align*}
		we see that we get a Cauchy net. Therefore $\pi(bu_\lambda)$ converges for every $h\in \pi(\A)H$. As this is a dense subspace and the net $\pi(bu_\lambda )$ is uniformly bounded, we may infer that the net is strongly Cauchy on $H$. Thus $h\mapsto \lim_{\lambda \in \Lambda} \pi(bu_\lambda)h$ gives us a bounded operator on $H$. 
		By definition we have $\pi(ab)=\pi(a)\pi(b)$ for $a\in \A,b\in M(\A)$. Letting $a,b\in M(\A)$ we have the following
		\begin{align*}
			\pi(a)\pi(b)h=\pi(a)\lim_{\lambda \in \Lambda}\pi(bu_\lambda)h=\lim_{\lambda\in \Lambda} \pi(abu_\lambda)h
		\end{align*}
		Where we have used that $bu_\lambda \in \A$, so that the extension defines a homomorphism. 
		To show uniqueness of this extension, we remark that for all $a\in \A$ and $b\in M(\A)$ we have that $bu_\lambda a\to ba$ in $\A$, it follows by essentiality that this extension is unique. 
	\item
		By what we have just shown, $\pi$ extends to representation of $M(\A)$. Let $T\in \pi(M(\A))$, and let $(b_\lambda)_{\lambda \in I}\subset \A$ be a net satisfying 
		\begin{align*}
			\lim_{\lambda \in I}\norm{b_\lambda a-Ta}=\lim_{\lambda \in I}\norm{ab-aT}=0
		\end{align*}
		where we have suppressed $\pi$. It follows that $T\pi(a)\in \pi(\A),\pi(a)T\in \pi(\A)$. Thus $T\in \pi(M(\A))$ idealizes $\pi(\A)$. For the other inclusion, let $T\in B(H)$ such that $T\pi(\A)\subset \A,\pi(\A)T\subset \A$. Considering the net $T\pi(u_\lambda)$, we get 
		\begin{align*}
			&\norm{T(\pi(u_\lambda a))-T\pi(a)}\leq \norm{T}\norm{\pi(u_\lambda a-a)}\to 0 \\
			&\norm{\pi(a)T\pi(u_\lambda)-\pi(a)T}\to 0
		\end{align*}
		As $\pi$ is assumed to be a cb. isomorphism and essential, it follows that $T=\pi(b)$ for some $b\in M(\A)$. To see that $T$ is the strict limit of a bounded net, we simply apply \Cref{mesrennie112}. To see that if $J\subset \A$ is a closed ideal in $\A$, it will also be a closed ideal in $M(\A)$, consider $T\in  M(\A)$ and $b\in J$. The net $u_\lambda Tb$ will converge to $Tb$ in norm, however $u_\lambda T\in \A$ so the net actually lies in $J$. As $J$ is closed $Tb\in J$ and likewise for $bT$. 
	\item/
		We have the cb. isomorphism $H\cong qH\osum (1-q)H$ where $q$ is given as in \Cref{mesrennie17}. Since $\pi$ is essential on $qH$ and $0$ on $(1-q)H$, by the first part of the theorem we get a representation $\pi:M(\A)\to B(qH)$ which is zero $(1-q)H$, thereby giving the desired representation. By construction, $\pi(1)=q$, showing the desired.
	\end{enumerate}
\end{proof}
In case $\A$ is a differentiable algebra, we can characterize the multiplier algebra of $\A$ in a more concrete fashion. The idea is that if we think of $\A$ as $C_0^1(M)$ on a complete manifold, then $M(\A)=C_b^1(M)$. 
\begin{proposition}\label{mesrennie117}
	Let $\D:\dom(\D)\to E_B$ be self-adjoint and regular and $\A\subset \lip(\D)$ be closed with bounded approximate unit, i.e. a differentiable algebra with bounded approximate unit. If the representation of $A$ is essential, then the multiplier algebra $M(\A)$ is cb-isomorphic to:
	\begin{align*}
		\tilde{M}(\A)=\{T\in M(A) : T\dom(\D)\subset \dom(\D),T\A,\A T\subset \A,[\D,T]\in L(E_B)\}
	\end{align*}
	Where the isomorphism preserves spectra.
\end{proposition}
\begin{proof}
	The algebra $\tilde{M}$ is clearly a subalgebra of $M(A)$ and included in the idealizer of $\pi_\D(\A)$ in $L_B(E\osum E)$. To see the other inclusion, start by considering
	\begin{align*}
		T\pi_\D(a)&=\begin{pmatrix} T_{11} & T_{12} \\ T_{21} & T_{22} \end{pmatrix} \begin{pmatrix}a & 0 \\ [\D,a] & \gamma(a) \end{pmatrix}
		=\begin{pmatrix} T_{11}a+T_{12}[\D,a] & T_{12}\gamma(a) \\ T_{21}a+T_{22}[\D,a] & T_{22}\gamma(a) \end{pmatrix}
	\end{align*}
	For this to be an element of $\pi_{\D}(\A)$, we must have $T_{12}\gamma(a)=0$, so $T_{12}=0$ by essentiality of $A$. This implies that $\gamma(T_{11}a)=T_{22}\gamma(a)$, so $\gamma(T_{11})=T_{22}$ by essentiality. 
	As $\A$ is essential, we see that $T_{11}$ must preserve the domain of $\D$. We can infer that $T_{21}$ must satisfy the equation \begin{align*} T_{21}a+\gamma(T)[\D,a]=[\D,T_{11}a]\end{align*} By essentiality we again infer that $T_{21}=[\D,T_{11}]$, and as such is bounded. This shows that the algebra $\tilde{M}(\A)$ we have defined also contains the idealizer of $\pi_{\D}(\A)$ as desired, giving the desired equality. 
	\newline
	For the second part of the theorem, note that $\overline{AE_B}=E_B$ so $\pi_\D(1)=1$ through \Cref{mesrennie114}. Recycling the arguments of the first and second parts of \Cref{mesrennie114} then gives that $M(\A)$ maps to $L_B(E_B)$ and that the idealizer of $\pi_\D(\A)$ equals $\pi_\D(M(\A))$. By our equivalent characterizations of the norm on $M(\A)$ in \Cref{mesrennie112} we get the desired cb-isomorphism. 
	The spectral invariance follows from a result of Mesland, given in the complex case by the equivalence of complex and real spectra it follows through in the real case, \cite[Theorem B.3]{mesland}. 
\end{proof}
\begin{definition}
	Let $\D:\dom(\D)\to E_B$ be a self-adjoint and regular, and $\A\subset \lip (\D)$ be a differentiable algebra with bounded approximate unit, with $\A$ a dense subset of the \Cstar-algebra $A$. If the representation of $A$ is essential we define the unitization of $\A$, $\A^+ \subset M(\A)$ as the algebra generated by $\A$ and $\pi_\D(1)$.
\end{definition}
One can show in that the unitization exists in general, \cite{meyer}.  
\begin{assumption}
	From now on we shall assume that all representations appearing in Kasparov modules are essential. 
\end{assumption}
\begin{remark}
	That the representations featured in the Kasparov modules is essential is not unreasonable, as one may show that any Kasparov module is equivalent to one where the representation is essential,\cite{kasparov}.
\end{remark}
\begin{definition}
	Let $\A$ be a differentiable algebra with a dense right ideal $\dom c$. A linear operator $c:\dom c\to \A$ is an unbounded multiplier if $c(ab)=(ca)b$ for $b,c\in \dom c$. If in addition $c$ satisfies:
	\begin{enumerate}
	\item
		$c$ is closed.
	\item
		It is formally symmetric in the inner product $\ip{a}{b}=a^*b$, i.e. $(ca)^*b=a^*(cb)$ 
	\item
		The operators $c^2+1$ is surjective and $(c^2+1)^{-1}$ is a bounded multiplier for $\A$. 
	\end{enumerate}
	$c$ is an unbounded self-adjoint multiplier.  
	Further, we say that $c$ is positive if $(ca)^*a\geq 0$ in $A$. 
\end{definition}

%\todo{We need to show first that $c^2$ is well-defined}
\begin{lemma}
	Let $c$ be an unbounded multiplier, then the operator $(c^2+\lambda)$ is invertible for all positive $\lambda$, likewise if $c$ is positive then $c+\lambda$ is invertible. 
\end{lemma}
\begin{proof}
	The operator $(c^2+1)^{-1}$ is bounded, and thus the operator $(c^2+\lambda)(c^2+1)^{-1}$ is invertible in $M(A)$, giving that $c^2+\lambda$ is bijective on $A$. Thus we may conclude by \Cref{mesrennie117} that $(c^2+\lambda)(c^2+1)^{-1}$ is invertible in $M(\A)$, giving that $c+\lambda:\dom c\to \A$ is bijective in $\A$ as well. We see that the positive case is proved in an entirely analogous manner. 
\end{proof}
It turns out the existence of a bounded approximate unit is enough to ensure a symmetric multiplier is also closed. 
\begin{lemma}
	For a differentiable algebra with a bounded approximate unit it is sufficient for a multiplier $c:\dom c \to \A$ to satisfy that $(ca)^*b=a^*(cb)$ for every $a,b\in \dom c$ for it to be closable. 
\end{lemma}
\begin{proof}
	As $\A$ has a bounded approximate unit, the norm on $\A$ can be equivalently characterized as $\norm{a}=\norm{a}_{op}$. Letting $a_n$ be a sequence converging to zero in $\dom c$, and $ca_n\to b$. By symmetry of $c$ and the identity $\norm{(ca_n)}=\norm{(ca_n)^*}$, we have 
	\begin{align*}
		b^*a=\lim_{n\to \infty} (ca_n)^* a=\lim_{n\to \infty} a^*_n(ca)=0
	\end{align*}
	Thus $\norm{b^*}_{op}=0$, implying $\norm{b^*}=0$ as desired. 
\end{proof}
It follows that if $(c^2+1)^{-1}$ is densely defined and bounded, then then the closure of $c$ is a self-adjoint unbounded multiplier with domain $(c^2+1)^{-1}\A$. We define the notion of a complete multiplier as follows 
\begin{definition}
	Let $(\A,E_B,\D)$ be an unbounded Kasparov module and $c$ self-adjoint multiplier of $\A$. Then we shall say that $c$ is complete if 
\begin{enumerate}
\item 
	$	(c^2+1)^{-1}\in \A$ 
\item
		$\im((\D^2+1)^{-1/2}(c^2+1)^{-1/2})=\im((c^2+1)^{-1/2}(\D^2+1)^{-1/2})$
\item 
		The operator $[\D,c]$ is bounded on $\im((c^2+1)^{-1/2}(\D^2+1)^{-1})$. 
\end{enumerate}
\end{definition}
These sets are natural to consider when working with unbounded operators, as they are the natural domains for $c\D$ and $\D c$.
We can now prove the essential theorem to expanding the expanded technical theorem of \cite{mesrennie} to the real setting, relating approximate units, positive self-adjoint multipliers and and the differential operator $\D$. This is one of the results where the proof requires the most modification to fit into the real setting. 
\begin{theorem}
	Let $\D:\dom(\D)\to E_B$ be a self-adjoint and regular operator, and $\A\subset \lip(\D)$ such that $AE_B=E_B$. Then the following are equivalent
	\begin{enumerate}
	\item
		There is an approximate commutative unit $(u_n)_{n\in \N}$ for $\A$ such that $\norm{[u_n,\D]}\to 0$ 
	\item
		There exists a positive self-adjoint multiplier $c$ for $\A$. 
	\item
		We have a strictly positive element $h\in \A$ such that $\im((\D^2+1)^{-1/2}h=\im(h(\D^2+1)^{-1/2}$, along with a constant $c$ such that $i[\D,h]\leq ch^2$ in $A\tens \C$.
	\end{enumerate}
\end{theorem}
\begin{proof} 
\begin{itemize}
	\item[$1\Rightarrow 2$]
		Let $\epsilon<1$. Pick a countable subset with dense linear span $(a_i)_{i\in \N}\subset \A$, without loss of generality we may assume that $\norm{(u_{n+1}-u_n)a_i}\leq \epsilon^{2n}$,as well as $\norm{[\D,u_n]}\leq \epsilon^{2n}$. Define $d_n=u_{n+1}-u_n$ and define the candidate self-adjoint multiplier 
		\begin{align*}
			c=\sum_{n=1}^\infty \epsilon^{-n}d_n
		\end{align*}
		This is densely defined, as may be readily verified for any fixed $a_i$, with $i<k<l$, and $c_j=\sum_{n=1}^j \epsilon^{-n}d_n$. The estimates: 
		\begin{align*}
			\norm{\sum_{n=k}^l \epsilon^{-n}d_na_i}&\leq \sum_{n=k}^l \epsilon^{-n}\norm{(u_{n+1}-u_n)a_i} \\
			&\leq \sum_{n=k}^l \epsilon^n
		\end{align*}
		show that $c_k a_i$ is a Cauchy sequence, so $ca_i\in \A$. The operator $c$ is clearly symmetric, so it suffices for us to show that $(c^2+1)^{-1/2}$ is densely defined and bounded. Consider the increasing sequence 
		\begin{align*}
			\norm{[\D,c_k]}&=\norm{\sum_{n=1}^k \epsilon^{-n}([\D,u_{n+1}]-[\D,u_n]}\leq \sum_{n=1}^k \epsilon^{-n} \pa{\norm{[\D,u_n]}+\norm{[\D,u_n]}} \\
			&\leq 2\sum_{n=1}^k \epsilon^{n}
		\end{align*}
		from which it follows that $\norm{[\D,c]}=\sup_k\norm{[\D,c_k]}<\infty$. We can now proceed with our calculations through the holomorphic functional calculus
		\begin{align*}
			&\sup_k \norm{[\D,(c_k^2+1)^{-1/2}}=\sup_{k}\norm{\int_{\Re{z}=1/2} (z^2+1)^{-1/2} [\D,(z-c_k)^{-1}] dz} \\ 
			&\sup_{k}\norm{\int_{\Re{z}=1/2} (z^2+1)^{-1/2} (z-c_k)^{-1} [\D,c_k](z-c_k)^{-1} dz} \\
			&\leq \sup_{k}\norm{\int_{\Re{z}=1/2} (z^2+1)^{-1/2} \norm{(z-c_k)^{-1}}^2 dz} \norm{[\D,c_k]} \\
			&\leq \sup_{k}\norm{\int_{\Re{z}=1/2} (z^2+1)^{-1/2} \norm{(z-c_1)^{-1}}^2 dz} \norm{[\D,c]}<\infty
		\end{align*}
		This shows that the sequence $(c_k^2+1)^{-1/2}$ is  bounded, so we need only check that it is strictly Cauchy for the limit to exist. 
		\begin{align*}
			&\norm{((c_l^2+1)^{-1/2}-(c_m^2+1)^{-1/2})a_i} \\
			&\leq \norm{\pa{1+\sum_{n=1}^l \epsilon^{2n}d^2_n}^{-1/2}-\pa{1+\sum_{n=1}^l \epsilon^{2n}d^2_n+\sum_{k=l+1}^m\epsilon^{2k}d^2_k}^{-1/2}}\norm{a_i}
		\end{align*}
		Picking $l$ sufficently large, we see that this expression may be bounded by $\epsilon$ independent of $m$, giving that the sequence is strictly Cauchy. Thus we have the existence of $(c^2+1)^{-1/2}\in M(\A)$, and we have shown that $c$ is a positive self-adjoint unbounded multiplier. We need to check the remainder of the criteria of $c$ being a complete multiplier. At first we verify the equality 
		\begin{align*}
			\im((c_k^2+1)^{-1/2}(\D^2+1)^{-1/2})=\im((\D^2+1)^{-1/2}(c_k^2+1)^{-1/2})
		\end{align*}
		As $(c^2+1)^{-1/2}$ is the strict limit of $(c_k^2+1)^{-1/2}$, we may write 
		\begin{align*}
			y=\lim_k(c_k^2+1)^{-1/2}(\D^2+1)^{-1/2}x
		\end{align*}
		for every element $y$ in the image  of $(c^2+1)^{-1/2}$.
		We may then write up the following identity
		\begin{align*}
			&(c_k^2+1)^{-1/2}(\D^2+1)^{-1/2}x \\
			&=-(\D^2+1)^{-1/2}(c_k^2+1)^{-1/2} \\
			&+(\D^2+1)^{-1/2}(c_k^2+1)^{-1/2}[(\D^2+1)^{1/2},(c_k^2+1)^{1/2}](c_k^2+1)^{-1/2}(\D^2+1)^{-1/2}
		\end{align*}
		We now need to show that the sequence $[(\D^2+1)^{1/2},(c_k^2+1)^{1/2}]$ is uniformly bounded in operator norm. To do this, consider the following calculations
		\begin{align*}
			&[(\D^2+1)^{1/2},(c_k^2+1)^{1/2}] \\
			&=\int_{z=1/2+it}(z^2+1)^{1/2}[(\D^2+1)^{1/2},(c_k-z)^{-1}] dz\\
			&=\int_{z=1/2+it}(z^2+1)^{1/2}(c_k-z)^{-1}[(\D^2+1)^{1/2},c_k](c_k-z)^{-1} dz \\
			&=\int_{w=i(t^2+1)+t}\int_{z=1/2+it}(z^2+1)^{1/2}(w^2+1)^{1/2}(c_k-z)^{-1}(\D-w)^{-1}[\D,c_k](\D-w)^{-1}(c_k-z)^{-1} dzdw 
		\end{align*}
		This sequence is clearly uniformly bounded in operator norm, so we may perform the limiting procedure for $c_k$ as follows
		\begin{align*}
			&\lim_k(c_k^2+1)^{-1/2}[\D,c_k](c_k^2+1)^{-1/2}(\D^2+1)^{-1/2}x \\
			&=(c^2+1)^{-1/2}[\D,c](c^2+1)^{-1/2}(\D^2+1)^{-1/2}x
		\end{align*}
		thereby showing the desired. The other inclusion is entirely analogous. 
		To see that $[\D,c]$ is bounded on $\im((c^2+1)^{-1}(\D^2+1)^{-1})$ we simply remark that it is the strong limit of the uniformly bounded family of operators $([c_k,\D])_{k\in \N}$.
		Finally, we need to show that $(c^2+1)^{-1/2}\in \A$. We start by showing that it is an element of $A$, then proceeding to show that it actually lies in $\A$. To see this is the case, consider the commutative subalgebra $B=C_0(X)$ of $A$ generated by $(u_n)_{n\in \N}$. We utilize that every unbounded multiplier is specified by its Gelfand transform, see eg. \cite[Theorem 2.1,Theorem 2.3]{wood}.%, which can be seen to be true as the result holds in the Real case and thus also in the real case.
		Fixing $0<t<1$ define the family of sets $X_n=\{x\in X| u_n(x)\geq t\}$. Let $x\in X,k\in \N$ and consider $m\geq k$. Then we may calculate as follows for $x\in X_k$
		\begin{align*}
			\sum_{n=0}^\infty \epsilon^{-n}d_n(x)&\geq \sum_{n=k}^\infty \epsilon^{-n}d_n(x)  \\
			&=\sum_{n=k}^m \epsilon^{-n}d_n(x)+\sum_{j=m+1}^\infty \epsilon^{-j}d_j(x) \\ 
			&\geq \sum_{n=k}^\infty \epsilon^{-k}d_n(x) + \sum_{j=m+1}^\infty \epsilon^{-j}d_j(x)\\
			&=\epsilon^{-k}(u_{m+1}-u_k)(x)+\sum_{j=m+1}^\infty \epsilon^{-j}d_j(x) \\ 
			&\geq \epsilon^{-k}(u_{m+1}-t)(x)+\sum_{j=m+1}^\infty \epsilon^{-j}d_j(x) 
		\end{align*}
		As $u_n$ is an approximate unit and $\sum_{j=m+1}^\infty \epsilon^{-j}d_j(x) $ converges pointwise to zero, we get the estimate 
		\begin{align*}
			\sum_{n\in \N} \epsilon^{-n} d_n(x)\geq (1-t)\epsilon^{-k}
		\end{align*}
		To see that $(c^2+1)^{-1/2}$ actually lies in $\A$, note that $u_n(c^2+1)^{-1/2}$ converges to $(c^2+1)^{-1/2}$ in $A$-norm. We have that $[\D,u_n]$ is bounded and $[\D,u_n]$ goes to zero, so we may derive:
		\begin{align*}
			[\D,u_n(c^2+1)^{-1/2}]&=u_n[\D,(c^2+1)^{-1/2}]+[\D,u_n](c^2+1)^{-1/2} \\
			&=\int_{z=(/2+ti} u_n (z^2+1)^{-1/2} (c+z)^{-1}[c,\D](c+z)^{-1} dz -[\D,u_n](c^2+1)^{-1/2} \\
			&=u_n\int_{z=1/2+ti} (z^2+1)^{-1/2} (c+z)^{-1}[c,\D](c+z)^{-1} dz -[\D,u_n](c^2+1)^{-1/2} 
		\end{align*}
		which converges to $[\D,(c^2+1)^{-1/2}]$ as desired. As such $\pi_\D(u_n(c^2+1)^{-1/2})$ converges to $\pi_\D((c^2+1)^{-1/2})$. As all $u_n(c^2+1)^{-1/2}$ lie in $\A$, we may infer that $(c^2+1)^{-1/2}\in \A$. 
	\item[$2\Rightarrow 1$]
		To see the claim, consider $f_n(x)=\exp(-x/n)$. Then we may write 
		\begin{align*}
			[\D,f_n(c)]y&=\int_0^1 \part_s (\exp(-c(1-s)/n)\D \exp(-cs/n)y)ds \\
			&=-\frac{1}{n}\int_0^1 \exp(-c(1-s)/n)[\D,c]\exp(-cs/n)yds 
		\end{align*}
		As everything is bounded, we may extend the equality to the entirety of $E_B$, as well as getting the inequality $\norm{[\D,f_n(c)]}\leq \frac{1}{n}\norm{[\D,c]}$
		To see that we actually have an approximate unit, note that $(c^2+1)^{-1/2}\A$ is dense, thereby $(c^2+1)^{-1/2}$ generates an essential ideal wherein $f_n(c)$ is clearly an approximate unit. 
	\item[$2\Lr 3$]
		Pick an unbounded multiplier $c$ on $\A$ and remark that $h=(1+c^2)^{-1/2}$ has dense range and is positive. If $h\in \A$ is positive with dense range, define the operator $c=h^{-1}$, which is densely defined on $\im(h)$. We can infer the domain relation from the identity
		\begin{align*}
			(h^{-2}+1)^{-1/2}=h(1+h^2)^{-1/2}
		\end{align*}
		and $1+h^2$ is invertible, so $(1+h^2)^{-1/2}$ is a bijection on $\dom(\D)=\im( (\D^2+1)^{-1/2})$. Then we have the equalities
		\begin{align*}
			&\im(h(h^2+1)^{-1/2}(\D^2+1)^{-1/2})=\im (h(1+\D^2)^{-1/2}) \\
			&=\im ((\D^2+1)^{-1/2}) h=\im ((\D^2+1)^{-1/2}h(1+h^2)^{-1/2})
		\end{align*}
		Finally, from the assumption that $i[\D,h]\leq ch^2$ for some $c\in \R^+$, it can be inferred for $e\in \im(h(\D^2+1)^{-1/2}h)$ that
		\begin{align*}
			&\ip{i[\D,h^{-1}]e}{e}_B=-i\ip{h^{-1}i[\D,h]h^{-1}e}{e}_B \\
			&=\ip{i[\D,h]h^{-1}e}{h^{-1}e}_B\leq C\ip{he}{h^{-1}e}_B=C\ip{e}{e}_B
		\end{align*}
		Then by taking a sequence $hy_n\to y\in E_B$ we can infer boundedness on the entirety of $\im(h(\D^2+1)^{-1/2})$
\end{itemize}
\end{proof}
\begin{remark}
This theorem allows us to assume that every approximate unit for a differentiable algebra is even, self-adjoint commutative and increasing.
\end{remark} This completes our study of differential algebras as independent objects, and we will henceforth be considering these in the context of constructing the unbounded Kasparov product.