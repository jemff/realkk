We can now enter the into a potential application of the unbounded Kasparov product, namely Poincare duality for \Cstar-algebras as developed in \cite{kasparov2} \cite{connes1} \cite{connes2}, and more recently in \cite{rennie2} which also illustrates the non-commutative manifold philosphy.
Before we proceed to define duality 
\begin{definition}
Given an iterated tensor product, define the mapping 
\begin{align*}
	\sigma_{nm}:A_1\tens \dots \tens  A_n \tens \dots A_m \tens \dots \tens A_k\to  A_1\tens \dots \tens  A_m \tens \dots A_n \tens \dots \tens A_k
\end{align*}
swapping the $n$'th and $m$'th entries for any iterated tensor product. Likewise, we define the standard map $\sigma^D:KK(A,B)\to KK(A\tens D,B\tens D)$, and the corresponding map $\sigma_D::KK(A,B)\to KK(D\tens A,D\tens B)$
\end{definition}
We can now define non-commutative duality for \Cstar-algebras. 
\begin{definition}
	Consider two \Cstar-algebras $A,B$. Then $A$ and $B$ are dual with a dimension shift of $i$ if there exist classes $\delta \in KKO^i(\R,A\tens B),\Delta \in KKO^{-i}(A\tens B,\R)$ satisfying that $\delta \tens_A \Delta=(-1)^{i} 1_B\in KK(B,B)$ and $\delta \tens_A \Delta=1_A \in KK(A,A)$ 
\end{definition}
An example of this is that $C_0(\R,\id)$ is dual $Cl_{1,0}$ and $C_0(\R,-\id)$ is dual to $Cl_{0,1}$.
The raison d'etre for this definition is:
\begin{proposition}
	If two \Cstar-algebras $A,B$ are dual, then $KO_*(A)\cong KO^{*+i}(B)$ with the isomorphism implemented by the Kasparov product with the classes $\delta,\Delta$. 
\end{proposition}
\begin{proof}
	We start by defining
	\begin{align*}
		&\delta_j:KO^j(B)\to KO_{j-i}(A) \\
		&\delta_j(x)=\delta\tens_B x \\
		&\Delta_j K_j(A)\to K^{j+i} \\
		&y\tens_A \Delta
	\end{align*}
	Then we show that
	\begin{align*}
		\Delta_{j-i}(\delta_j(x))&=(-1)^{ij}(\delta \tens_A \Delta)\tens_B x \\
		\delta_{j+i}(\Delta_j(y))&=(-1)^{ij} y \tens_A (\delta\tens_B \Delta)
	\end{align*}
	where $(\delta \tens_B \Delta)=\delta\tens_B \sigma^{*}_{12}(\Delta)$ and likewise for $\Delta \tens \delta$, with $\sigma_{12}:A\tens B\to B\tens A$, $\sigma(a\tens b)=b\tens a$. 
	Let $x\in K^{j}(B)$. Then we have 
	\begin{align*}
		\Delta_{j-i}(\delta_j(x))=\sigma_B(\delta)\tens_{A\tens B \tens B} \sigma^A\sigma_B(x) \tens_{A\tens B} \Delta 
	\end{align*}
	Remark that $\sigma^A\sigma_B \tens \Delta =\sigma_{23}^*(x\tens_\R \Delta)$, so 
	\begin{align*}
		\sigma^A\sigma_B\tens_{A\tens B} \Delta=(-1)^{ij}\sigma^*_{23}(\sigma_B(\Delta)\tens_B x)
	\end{align*}
	We still need to check that 
	\begin{align*}
		\sigma_B(\delta)\tens_{A\tens B \tens B} \sigma^*_{23}(\sigma_B(\Delta))=\delta \tens_A \Delta
	\end{align*}
	Remark that 
	\begin{align*}
		&\sigma_{12}^*\sigma_{23}^*(\sigma_B(\Delta))=\sigma^B(\Delta) \\
		\delta\tens_A \Delta =(\sigma_B(\sigma_12)_*(\delta))
	\end{align*}
	So drawing all of this together, we get the equality
	\begin{align*}
		\delta\tens_A \Delta &=\sigma_B(\delta)\tens_{A\tens B \tens B} \sigma^*_{12}\sigma^*_{12}\sigma_{23}^*(\sigma_B(\Delta)) \\
		&=\sigma_B(\delta)\tens_{A\tens B \tens B} \sigma_{23}^*(\sigma_B(\Delta))
	\end{align*}
	Showing the first part of the two statements. Then the claimed proposition follows if $\delta\tens_B \Delta=1_A$ or $\delta \tens_A \Delta  (-1)^{i} 1_B$
\end{proof}
This further specializes to Connes' notion of Poincare duality, introduced in the real case in \cite[P.601]{bookbig} expanded upon in the complex case in \cite{rennie2} and \cite{connes1},\cite{connes2}. 
\begin{definition}
	A complex \Cstar-algebra $A$ is said to satisfy Poincare duality if $A^{\op}$ is dual to $A$.  
\end{definition}
If $A$ is a real algebra, then $A\cong A^{\op}$ and we get that $K^{j}(A\tens \C)\cong K_{j-i}(A\tens \C)$ if $A\tens \C$ can be shown to satisfy Poincare duality. 

It is well-known, \cite{connes2} that the non-commutative torus satisfies Poincare duality in the complex case with $i=2$, as such it is an interesting problem whether the real non-commutative torus satisfies Poincare duality. The failure or succes of this would be a good test of whether the idea of a non-commutative spin manifold is meaningful. We start by computing the $KO$-theory of the non-commutative torus. 
\begin{proposition}
The $K$-theory of the non-commutative Torus given as $C(S^1,id)\rtimes_{\theta} \Z$ is: 
\begin{align*}
	KO_n(A_\theta)=\left \{ \begin{array}{c} (\Z+\theta\Z)\osum \Z_2 \\ \Z\osum (\Z_2)^3 \\ (\Z_2)^3 \\ \Z\osum \Z_2 \\ \Z\osum \theta \Z  \\ \Z \\ 0 \\ \Z \end{array} \right .
\end{align*}
\end{proposition}
\begin{comment}
\begin{proof}
	As we are presently only interested in the $K$-theory classes as abstract groups, we may start by considering the case where the action is trivial. In this case the crossed product reduces to the tensor product $C(S^1,\tau)\tens C^*(\Z)$. Letting $a$ be the generator of $\Z$, we see that we may define the $^*$-isomorphism $a^n \to (z\mapsto z^n)$ so that $C^*(\Z)\cong C(S^1,\tau)$. Thus we get $A_0=C(S_1,id)\rtimes_0 \Z\cong C(S^1,id)\tens C(S^1,\tau_0)$. We may calculate the $K$-theory of this tensor product via. \Cref{ktheorys1} to get the desired table. The $K$-theory groups are independent of the choice of parameter as the group structure of $KO_*(A_\theta)$ depends continuously on $\theta$, showing the desired. 
\end{proof}
\end{comment}
\begin{proof}
	Defining
\begin{align*}
	T_\theta(C(S^1,\id))=\{f\in C([0,1],A) | f(1)=\exp(i\theta)f(0) \}
\end{align*}
By \cite[1.36]{schroder}, there is the isomorphism
\begin{align*}
	KO_n(C(S^1,\id)\rtimes_\theta \Z) \cong KO_{n-1}(T_\theta(C(S^1,\id)))
\end{align*}
There is the associated short exact sequence to $T_\theta(C(S^1,\id))$
\begin{align*}
	\xymatrix{ 
		0 \ar[r] & C_0(\R)\tens C(S^1,\id) \ar[r] &  T_\theta(C(S^1,\id)) \ar[r] & C(S^1,\id) \ar[r] & 0 
	}
\end{align*}
We take $KO_*$-theory of this sequence,
\begin{align*}
\xymatrix{
	\dots \ar[r] & KO_{n}(C(S^1,\id)) \ar[r] &  KO_{n-1}(T_\theta(C(S^1,\id)) )\ar[r] & KO_{n-1}(C(S^1,\id)) \ar[r] & \dots
	}
\end{align*}
We can construct the split $s$ for a projection $p\in M_n(C(S^1,\id)\tens Cl_{0,n})$
\begin{align*}
&[p(x)]\mapsto [p(x+t\theta)] \\
&KO_{n-1}(C(S^1,\id))\to  KO_{n-1}( T_\theta(C(S^1,\id)) )
\end{align*}
Thus the 24-term exact sequence reduces to $KO_{n-1}(T_\theta(C(S^1,\id)))\cong KO_{n-1}(C(S^1,\id)) \osum KO_n(C(S^1,\id))$, showing the desired. 
\end{proof}
\begin{remark}
	Though the $K$-theory groups are independent of the parameter, their generators are most emphatically not. 
\end{remark}
\begin{conjecture}
	The non-commutative torus satisfies real Poincare duality, with K-homological class: 
	\begin{align*}
		\Delta \in KKO(Cl_{0,2} \tens A_\theta \tens A_\theta^{op},\R)
	\end{align*}
	We consider the module $L^2(A_\theta, \tau) \osum L^2(A_\theta,\tau)$, ie. the completion of $A_\theta$ in the inner product given by the trace. 
	\begin{align*}
		\D=\gamma_1\delta_1+\gamma_2 \delta_2
	\end{align*}
	We have the involution $J_0$ on $L^2(A_\theta, \tau)$ given as $J_0a=a^*$. Defining $J=e_{21}J_0-e_{12}J_0$ we represent one generator of $Cl_{0,2}$, the second generator may be represented as $ie_{21}J_0+ie_{12}J_0$.
	The $K$-theoretic cycle is 
	\begin{align*}
		\delta \in KKO(\R,A_\theta \tens A_\theta^{op} \tens Cl_{0,2})
	\end{align*}
	Given as 
	\begin{align*}
		e_0^{op}\cup e_1 - e_1^{op} \cup e_0 +u\cup u^{op}+w
	\end{align*}
	where $e_0,e_1$ are the non-torsion parts of the $KO_0$ theory of $A_\theta$ and $u$ is the non-torsion part of the $K_1$-group of $A_\theta$. The parameter $w$ encodes the third and seventh $K$-groups.
\end{conjecture}
We have been unable to construct $K$-theory elements pairing with the proposed $K$-homology cycle in a non-degenerate fashion, but if this were successful we could use the machinery of the unbounded Kasparov product to find explicit representatives of the $K$-homology of the real non-commutative torus. It would be interesting to see if this were possible, and also see if one could show real duality for the real algebras for which the duality is known for the complex analogue. The definition of Poincare duality is not entirely clear in the real case, as to whether we can only except duality up to $2$-torsion, ie. up to tensoring with $\Z[\frac{1}{2}]$. There are various definitions of a non-commutative manifold, the most recent one being in \cite{rennie2} where it is shown by utilizing the unbounded Kasparov product extensively that this definition recovers the notion of a commutative Riemannian and Spin manifold respectively. In all of these definitions the candidate non-commutative manifold is required to satisfy Poincare duality, in analogy with smooth manifolds equipped and singular (co)homology.


%\begin{proof}
%	\todo{prove this}
%\end{proof}
%It is natural to consider the irrational rotation algebras, and see if we may calculate the $KO$-homology of it by constructing appropriate cycles. The way to do this is by considering the spin structures on it along with the corresponding dirac operators. 
%We start by considering the very special class, or "fundamental class"(checK) stemming from the defining extension. 
%\todo{Use Pimsner-Voiculescu to calculate what we expect it to be, based on what we know of the K-homology of the reals and K-theory of rotation algebra.}. 
